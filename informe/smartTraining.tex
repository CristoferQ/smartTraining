%% inicio, la clase del documento es iccmemoria.cls
\documentclass{smartTraining}
\usepackage{amsmath}
\usepackage{mhchem}
%\usepackage{chemfig}
\usepackage{makecell}
\renewcommand{\arraystretch}{3}
%% datos generales y para la tapa
\titulo{\textit{Smart Training}, Servicio web de Entrenamiento de Modelos}
\author{David Alfredo Medina Ortiz}
\supervisor{�lvaro Olivera Nappa, Ph.D.}
\director{Profesor del curso}
\date{Agosto, 2018}

%% inicio de documento
\begin{document}

%% crea la tapa
\maketitle

%%% dedicatoria
%\begin{dedicatory}
%Dedicado a ...
%\end{dedicatory}
%
%%% agradecimientos
%\begin{acknowledgment}
%Agradecimientos a ...
%\end{acknowledgment}

%% indices
\tableofcontents
\listoffigures
\listoftables

%% resumen
\begin{resumen}

Conceptos como miner�a de datos, machine learning, big data, an�lisis estad�sticos, modelamientos matem�ticos, etc, son mencionados d�a a d�a, ya sea en el �mbito privado como p�blico, involucrando �reas como: comercio, salud, investigaci�n, transporte, etc. Lo cual denota que son tem�ticas que han adquirido mayor relevancia y su ascenso seguir� con el pasar del tiempo.

La manipulaci�n de grandes vol�menes de datos, con el fin de poder extraer informaci�n de ellos, b�squeda de patrones, evaluaciones estad�sticas, etc. Implica por parte del interesado, tener conocimientos en dichas �reas adem�s de comprender herramientas inform�ticas que le permitan dicho procedimiento. Sin embargo, dichas herramientas o son costosas, debido a la licencia que implica, o, se requiere de conocimiento inform�tico para su manipulaci�n, debido a que requiere implementar m�dulos o servicios a medida que permitan ejecutar las tareas de inter�s, lo cual deja a un n�mero importante de entidades que desean involucrarse en dicho mundo, pero no cuentan con las capacidades ni tampoco con las competencias para ello.

Dado a lo anterior y en base a la creciente demanda de desarrollo de metodolog�as que permitan aplicar data mining a procesos de datos, con el fin de extraer informaci�n y conocimiento de la misma, se propone Smart Training, como sistema web, que facilite los procesos de evaluaciones estad�sticas, b�squeda de patrones de comportamiento, desarrollo de modelos de clasificaci�n y evaluaci�n de caracter�sticas o features en el set de datos a estudiar.

			%resumen...
\end{resumen}

\chapter{INTRODUCCI�N}

\section{Data Mining}

Miner�a de datos es el proceso de descubrimiento de patrones en set de datos, involucrando m�todos asociados a Machine Learning, Estad�sticas y sistemas de bases de datos. \cite{intro1}. La miner�a de datos es un subcampo interdisciplinario de la inform�tica, el cual tiene por objetivo general extraer informaci�n (a trav�s de m�todos inteligentes) de un conjunto de datos y transformar la informaci�n en una estructura comprensible para su uso posterior. \cite{intro2, intro3}. La miner�a de datos es el paso de an�lisis del proceso de \textit{descubrimiento de conocimiento en bases de datos}, o KDD. \cite{intro4}. Adem�s del an�lisis en bruto de los datos, tambi�n incluye aspectos de manipulaci�n de bases de datos, pre procesamiento de datos, evaluaciones de modelo e inferencia, m�tricas de inter�s, consideraciones de complejidad, post procesamiento de estructuras descubiertas, visualizaci�n y actualizaci�n de la informaci�n.

En la Figura \ref{intro1}, se exponen las principales ramas que componen la miner�a de datos y los diferentes procesos que se asocian a dichas ramas.

\begin{figure}[!h]
	
	\centering
	\includegraphics[scale=.3]{imagenes/data_mining.jpg}
	\caption{Componentes en la miner�a de datos}
	\label{intro1}
\end{figure}

Son tres las principales �reas que abarca la miner�a de datos: Estad�stica, Inteligencia Artificial y Manipulaci�n de sistemas de informaci�n, mientras que son distintos procesos los que interact�an entre estas ramas, tales como: Modelamiento Matem�tico, reconocimiento de patrones, Sistemas de almacenamiento persistente y machine learning.

Cada �rea en particular tiene un objetivo general y diversos objetivos espec�ficos. Sin embargo, estas �reas interact�an entre s�, con el fin de poder extraer patrones de informaci�n que generen conocimientos a partir de la data de procesada.

La miner�a de datos se utiliza en diferentes campos, tales como: Gen�tica, Evaluaciones prote�micas, Comercio, Sistemas de tr�nsito, Optimizaciones en procesos industriales, reconocimiento de patrones y rasgos cuantificables en enfermedades y m�s recientemente en �reas de din�micas moleculares y par�metros para la generaci�n de pipe lines automatizados de simulaciones cu�nticas en sistemas qu�micos.

El uso de la miner�a de datos y la b�squeda de patrones de comportamientos de datos de inter�s, no s�lo es un �rea que se enfoca en la investigaci�n. Diversas son las entidades privadas que ofrecen servicios de big data y data science, adem�s del sector p�blico, con el fin de detectar puntos cr�ticos de zonas de riesgo, zonas de accidentes, evaluaci�n o perfilaci�n de grupos de estudio, etc. 

\section{Problem�tica}

Actualmente, los sistemas de almacenamiento persistente, permiten disponer de informaci�n que es de inter�s para distintos tipos de entidades, las cuales depender�n del �rea a las que se dediquen. Sin embargo, cada �rea en particular, tiene objetivos similares, tales como:

\begin{itemize}
	
	\item Qu� significa los datos que tengo?
	\item Puedo personalizar y interpretar datos?
	\item Puedo optimizar procesos en base a la informaci�n que tengo sobre estos?
	\item Es factible aumentar la experiencia de usuario en cuanto a procesos de ventas y compras, personalizando sus �reas de cliente?
\end{itemize}

Son muchos los objetivos que se pueden encontrar y muchas las aplicaciones que implica esta t�cnica. No obstante, el hecho de tener data almacenada y no saber c�mo procesarla es un gran problema para muchas peque�as y medianas empresas, as� como para tambi�n ventas, laboratorios de investigaci�n, etc.

El deseo de aplicar miner�a de datos, con el fin de extraer informaci�n, es d�a a d�a, m�s frecuente. Sin embargo, un usuario com�n debe enfrentar la problem�tica de como afrontar el problema, adquirir las competencias, o simplemente, contratar servicios de data science, los cuales cobran altas sumas de dinero y emplean un tiempo elevado con el fin de llegar a una respuesta pronta.

Por otro lado, existen herramientas que facilitan la utilizaci�n de miner�a de datos, pero, el costo por conceptos de licencia es demasiado elevado y no permite abarcar diversas �reas y testear algoritmos variados. Mientras que por otro lado, existen m�dulos o librer�as que han sido implementadas en diversos lenguajes de programaci�n, con el fin de aplicar miner�as de datos, pero esto aumente a�n m�s la complejidad del tema, debido a que para implementar dicha labor, se requiere de conocimientos en programaci�n y en algunos casos, las implementaciones son engorrosas y requieren de un conjunto de arquitecturas que soporten dichas instancias.

Dado a lo anterior, es que se propone el desarrollo de una herramienta web, que permita la aplicaci�n de diversas t�cnicas de miner�a de datos y oriente de manera inteligente al usuario, esto implicar�a que no se requiere de un conocimiento de programaci�n, adem�s que las competencias en miner�a de datos deben ser m�nimas, puesto que la idea contempla la orientaci�n al usuario con respecto al objetivo que desea.

Durante este documento, se expone el dise�o de la herramienta, las metodolog�as a utilizar y los artefactos de software que se crear�n, con el fin de poder implementar esta herramienta en base a una metodolog�a de dise�o, adem�s se expone un resumen de las tecnolog�as actuales, cuales son las ventajas y desventajas que poseen cada una y en qu� se diferencia el software planteado con respecto a los existentes.

\subsection{Estado del Arte}

\section{Smart Training}

Smart Training es un sistema web, que facilita la utilizaci�n de t�cnicas de miner�a de datos, con el fin de evaluar caracter�sticas en set de datos, reconocer patrones, entrenar algoritmos de clasificaci�n, generar evaluaci�n de caracter�sticas, etc. Tiene por finalidad acercar la miner�a de datos a un p�blico que no posee las competencias para implementar modelos mediante utilizaci�n de m�dulos de programaci�n.

Smart Training se compone de 5 m�dulos, los cuales se detallan a continuaci�n.

\subsection{M�dulo de Procesamiento de Datos}

Este m�dulo tiene por objetivo cargar la data entregada en archivos de texto, eval�a los datos existentes, corrobora que no existan problemas, revisa el set de datos, encuentra valores nulos, codificaciones no permitidas, etc, a lo que, finalmente, entrega un resumen del proceso, exponiendo los resultados y de dicha tarea y comentando si es posible trabajar con dicho set cargado, en caso contrario, expone mensajes con recomendaciones a seguir.

\subsection{M�dulo de An�lisis Estad�stico}

Este m�dulo permite la evaluaci�n del set de datos contemplando, correlaciones, box plot, distribuciones mediante histogramas, scatter plot, gr�ficos de densidad de datos, adem�s de res�menes estad�sticos para cada atributo o caracter�stica en el set de datos que se est� trabajando.

\subsection{M�dulo de An�lisis de Features}

Este m�dulo permite evaluar las caracter�sticas en el set de datos y el aporte que �stas entregan, adicional a ello, contempla an�lisis mediante t�cnicas PCA, para generar reducciones de dimensionalidad, Mutual Information y t�cnicas basadas en correlaci�n, todas con el objetivo de explicar los comportamientos de �stas y c�mo influyen en el set de datos.

\subsection{M�dulo de Clustering}

Clustering es una de las t�cnicas m�s conocidas para asociar segmentos en una muestra, es decir, generar grupos o particiones que tengan un alto grado de diferencia entre ellas, pero cuyos integrantes en una partici�n dada, sean altamente similares.

Existen diferentes algoritmos de clustering y par�metros asociados a estos, los cuales tienen formas de encontrar particiones distintas, bas�ndose en distancias, medidas gausianas, generaci�n de hiper planos, etc.

Este m�dulo tiene por objetivo generar exploraci�n de dichas t�cnicas y algoritmos, con el fin de poder entregar particiones en el set de datos, adem�s, permite la evaluaci�n de dichas particiones con el fin de poder determinar si son estad�sticamente significativas o no, adem�s de cumplir con los criterios de similitud y diferenciaci�n mencionados previamente.

Normalmente, la b�squeda de particiones conlleva al hecho de generar modelos de clasificaci�n para nuevos ejemplos y determinar a qu� particiones se encuentran, o tambi�n, generar divisiones para implementar set de datos diferentes con comportamientos diferentes, de tal manera que a la hora de aplicar algoritmos de clasificaci�n, sus comportamientos presenten mayor eficiencia.

\subsection{M�dulo de Entrenamiento de Modelos}

Entrenar un modelo de clasificaci�n, predicci�n, implica tener un conjunto de elementos con su clasificaci�n o valor de predicci�n conocido, con el fin de poder, a partir de �ste, evaluar nuevos ejemplos, ya sea para clasificarlos o para predecir posibles valores de inter�s. Todas estas tareas, aplicando miner�a de datos, se logran mediante la implementaci�n de algoritmos de aprendizaje supervisado.

Existen diferentes algoritmos de aprendizaje supervisado, los cuales contemplan diferentes formulaciones matem�ticas y caracter�sticas, los cuales cumplen con dicha tarea, cada uno de estos, presenta caracter�sticas distintas, en relaci�n al funcionamiento del mismo, la elecci�n de un algoritmo por sobre otro, va de la mano con el hecho de las necesidades que el problema conlleva, ya sea, con el fin de entregar s�lo un resultado, adicionar valores que permitan explicar el porqu� de la clasificaci�n, etc. Normalmente, para un problema desconocido, es necesario implementar fases exploratorias que permitan evaluar diferentes algoritmos y sus par�metros. Con el fin de poder, a partir de dicha instancia y en base a m�tricas que permitan evaluar el desempe�o, seleccionar un algoritmo y sus par�metros.

Lo anterior, es el objetivo del m�dulo de entrenamiento de modelos, la idea de �ste, es recibir un set de datos con ejemplos clasificados o cuya respuesta tenga un valor conocido y aplicarle diversos algoritmos y variaciones de par�metros, entregando un resumen de las medidas de desempe�o obtenidas y efectuando un ranking seg�n medida, para que finalmente se pueda entregar una recomendaci�n de los mejores modelos para un cierto problema.

\subsection{Diagrama soluci�n}

\section{Objetivos y Alcances}

El objetivo general del proyecto contempla la implementaci�n de un sistema web denominado, \textit{Smart Training}, el cual permita la aplicaci�n de distintas t�cnicas de miner�a de datos a set de datos de inter�s del usuario, a partir de los cuales �ste pueda entender comportamientos de datos y generar conocimiento a partir de ellos.

Es importante destacar los objetivos espec�ficos que nacen dentro del desarrollo del proyecto.

\begin{enumerate}
	
	\item Dise�ar metodolog�a de software a utilizar.
	\item Comprender los requerimientos observados a partir del estado del arte.
	\item Evaluar las funcionalidades y atributos que tendr� el sistema.
	\item Comprender los usuarios y actores del software.
	\item Entender las secuencias y flujos de trabajo existentes en la herramienta.
	\item Crear modelos de conceptos, entidades y clases.
	\item Implementar los m�dulos propuestos.
	\item Implantar sistema.
\end{enumerate}

\section{Metodolog�a de Desarrollo de Software}

Existen diversas metodolog�as de desarrollo de software, las cuales contemplan diferentes caracter�sticas y se enfocan en distintos puntos objetivo. Algunos ejemplos son.

\begin{itemize}
	
	\item Metodolog�as �giles.
	\item Dise�o cascada.
	\item Dise�o estreslla.
	\item Iterativas.
\end{itemize}

Metodolog�a �gil, se utiliza cuando el objetivo se basa principalmente en sacar a producci�n el software de manera r�pida, no contempla procesos de dise�o complejos y simplemente se centra en el desarrollo del producto, lo cual permite, por un lado, poseer un producto en poco tiempo. Sin embargo, est� sometida a enmarcar errores debido a que no se consideraron patrones de dise�o.

El dise�o en cascada y tambi�n en estrella, se centran en los requerimientos de usuario y generar sub productos asociados al software final, los cuales cumplen un objetivo en particular del software, fueron muy utilizadas en los a�os 90, debido a la simplicidad que estos pose�an.No obstante, no contempla iteraciones para evaluar los flujos de trabajo, ni tampoco la utilizaci�n de paradigmas complejos de desarrollo de software.

Una de las metodolog�as m�s utilizadas, es la Iterativa, �sta contempla un conjunto de patrones de dise�o, los cuales est�n asociados a la evaluaci�n de las funcionalidades, los atributos, describir secuencias de flujos y asociar conceptos, es la metodolog�a que contempla un mayor conjunto de pasos. No obstante, es la que m�s asegura que a la hora de implementar, dicho proceso sea r�pido. El hecho de ser iterativa, implica que cada etapa entrega un artefacto de software, bajo el cual depende el siguiente, en nuevas etapas se hacen mejoras a los artefactos generados y se est� en constante feedback.

Adicional a las metodolog�as de software, existen diferentes paradigmas de programaci�n que son empleados en conjunto con dichas estrateg�as. Los dos principales son: Estructurado y Orientado a Objetos. En el primero, se sigue un orden secuencial del problema a desarrollar, no est� adaptado para grandes desarrollos, si no que m�s bien, es empleado en scripting y manejo de patrones en archivos de texto. Por otro lado, la programaci�n Orientada a Objetos (POO), cumple con la caracter�sticas de ser m�s cercana a la vida real, debido a que se basa en el dise�o de clases, que representan entidades las cuales pueden ser ficticias o reales. Presenta grandes ventajas debido a que posee las siguientes caracter�sticas.

\begin{itemize}
	
	\item \textit{Encapsulamiento}: Un Objeto es due�o de sus atributos y m�todos.
	\item \textit{Polimorfismo}: Un mismo m�todo, pueden tener significados diferentes para distintas clases.
	\item \textit{Reusabilidad}: Una clase modelada puede utilizarse en diversos proyectos debido a que siempre poseer� los mismos atributos y m�todos.
\end{itemize}

Adem�s de dichas caracter�sticas, la POO asocia conceptos abstractos a la programaci�n, tales como: herencia, dependencias, asociaciones y composiciones, las cuales aumentan m�s a�n, la usabilidad de este paradigma.

En esta ocasi�n, debido a la envergadura del proyecto, a las caracter�sticas que se espera que posea y a las ventajas que entregan las metodolog�as, se utilizar� el dise�o de software iterativo con objeto acoplado. Es decir, se dise�ar� teniendo en consideraci�n distintas iteraciones asociadas a los artefactos de software que se desarrollen, enfocando dicho dise�o a POO.
   %intro general al tema
\chapter{An�lisis}

La etapa de an�lisis del proyecto, contempla la evaluaci�n y determinaci�n de las diferentes funcionalidades que tendr� el sistema, asociado a los atributos que estos presentan y que permiten cuantificar de cierta manera el software. Tambi�n contempla las evaluaciones de los flujos asociados a cada funci�n y determina las secuencias de pasos a seguir para dar respuesta a cada una de �stas, exponi�ndolos en forma narrativa mediante los casos de uso. Finalmente se eval�an los conceptos existentes y que representan entidades dentro del software, los cuales forman parte del dise�o posterior.

\section{Funciones del Sistema}

Las funciones del sistema, hacen referencia a las acciones que se pueden desarrollar en software, las cuales representan flujos completos de trabajo en cuanto a nivel de interacci�n y coexisten al menos un actor y un concepto.

A continuaci�n se listan un conjunto de funciones, las cuales han sido divididas con respecto a la finalidad que �stas cumplen en el software, cada funci�n posee su n�mero de registro, adem�s de presentar su categor�a y su sigla asociada, esto es con el objetivo de poder asociar funciones a m�dulos y agruparas seg�n el objetivo o la relaci�n que �stas tengan.

Un resumen de estas funciones se expone en la Tabla \ref{analisisTab1}.


% Please add the following required packages to your document preamble:
\begin{table}[!h]
	\centering
	\resizebox{\textwidth}{!}{%
		\begin{tabular}{|l|l|l|l|}
			\hline
			\multicolumn{4}{|c|}{\textit{\textbf{Descripci�n de categor�a de funciones}}}                                                                                                                                             \\ \hline
			\multicolumn{1}{|c|}{\textit{\textbf{\#}}} & \multicolumn{1}{c|}{\textit{\textbf{ID}}} & \multicolumn{1}{c|}{\textit{\textbf{Nombre Categor�a}}} & \multicolumn{1}{c|}{\textit{\textbf{Descripci�n}}}                     \\ \hline
			1                                          & US                                        & Usuario                                                 & Funciones relacionadas a las acciones a realizar por parte del usuario \\ \hline
			2                                          & AC                                        & Acceso                                                  & Funciones relacionadas al sistema de acceso a la aplicaci�n            \\ \hline
			3                                          & ES                                        & Estad�sticas                                            & Funciones relacionadas a las estad�sticas de uso de la herramienta     \\ \hline
			4                                          & AE                                        & An�lisis estad�stico                                    & Funciones relacionadas al an�lisis estad�stico de los datos de entrada \\ \hline
			5                                          & ESD                                       & Evaluaci�n set de datos                                 & Funciones relacionadas al chequeo y validaci�n de los datos de entrada \\ \hline
			6                                          & MANS                                      & Modelos de Aprendizaje No Supervisado                   & Funciones relacionadas al uso de clustering                            \\ \hline
			7                                          & MAS                                       & Modelos de Aprendizaje Supervisado                      & Funciones relacionadas al entrenamiento de modelos                     \\ \hline
			8                                          & AF                                        & An�lisis de Caracter�sticas                             & Funciones relacionadas al an�lisis de caracter�sticas                  \\ \hline
			9                                          & SC                                        & Sistema de Colas                                        & Funciones relacionadas al proceso de jobs en sistemas de colas         \\ \hline
			10                                         & SN                                        & Sistema de notificaciones                               & Funciones relacionadas a las notificaciones v�a correo electr�nico     \\ \hline
			11                                         & AP                                        & Almacenamiento Persistente                              & Funciones relacionadas al sistema de almacenamiento persistente        \\ \hline
		\end{tabular}%
	}
	\caption{Resumen categor�as de funciones.}
	\label{analisisTab1}
\end{table}

A continuaci�n, se lista y describen cada funciones, seg�n la categor�a a la que pertenecen.

\newpage
\subsection{Funciones asociadas al Usuario}

% Please add the following required packages to your document preamble:
% \usepackage{multirow}
% \usepackage{longtable}
% Note: It may be necessary to compile the document several times to get a multi-page table to line up properly
\begin{longtable}{|l|c|c|l|l|}
	\hline
	\multicolumn{5}{|c|}{\textit{\textbf{Funciones del Sistema}}}                                                                                                                                                                                                                                                                           \\ \hline
	\endfirsthead
	%
	\endhead
	%
	\multicolumn{1}{|c|}{\textit{\textbf{\#}}} & \textit{\textbf{ID Categor�a}} & \textit{\textbf{Categor�a}}        & \multicolumn{1}{c|}{\textit{\textbf{Funci�n}}}                        & \multicolumn{1}{c|}{\textit{\textbf{Descripci�n}}}                                                                                           \\ \hline
	1                                          & \multirow{10}{*}{\textbf{US}}  & \multirow{10}{*}{\textbf{Usuario}} & \begin{tabular}[c]{@{}l@{}}Registrar nuevo \\ \\ usuario\end{tabular} & \begin{tabular}[c]{@{}l@{}}El sistema deber� permitir\\ el registro de un nuevo \\ usuario en el almace-\\ namiento persistente\end{tabular} \\ \cline{1-1} \cline{4-5} 
	2                                          &                                &                                    & \begin{tabular}[c]{@{}l@{}}Editar usuario\\ existente\end{tabular}    & \begin{tabular}[c]{@{}l@{}}El sistema deber� permitir\\ la edici�n de un usuario\\ existente\end{tabular}                                    \\ \cline{1-1} \cline{4-5} 
	3                                          &                                &                                    & \begin{tabular}[c]{@{}l@{}}Eliminar usuario\\ existente\end{tabular}  & \begin{tabular}[c]{@{}l@{}}El sistema deber� permitir\\ la eliminaci�n de un\\ usuario existente\end{tabular}                                \\ \cline{1-1} \cline{4-5} 
	4                                          &                                &                                    & Visualizar usuario                                                    & \begin{tabular}[c]{@{}l@{}}El sistema deber� permitir\\ listar a todos los usuarios\\ y un detalle de los mismos\end{tabular}                \\ \cline{1-1} \cline{4-5} 
	5                                          &                                &                                    & \begin{tabular}[c]{@{}l@{}}Crear �reas de\\ trabajo\end{tabular}      & \begin{tabular}[c]{@{}l@{}}El sistema deber� permitir\\ crear �reas de trabajo\\ para un nuevo usuario\end{tabular}                          \\ \cline{1-1} \cline{4-5} 
	6                                          &                                &                                    & \begin{tabular}[c]{@{}l@{}}Eliminar �reas de\\ trabajo\end{tabular}   & \begin{tabular}[c]{@{}l@{}}El sistema deber� permitir\\ eliminar �reas de trabajo\\ para un usuario existente\end{tabular}                   \\ \cline{1-1} \cline{4-5} 
	7                                          &                                &                                    & Registrar roles                                                       & \begin{tabular}[c]{@{}l@{}}El sistema deber� permitir\\ el registro de nuevos roles\\ en el almacenamiento\\ persistente\end{tabular}        \\ \cline{1-1} \cline{4-5} 
	8                                          &                                &                                    & \begin{tabular}[c]{@{}l@{}}Editar roles\\ existente\end{tabular}      & \begin{tabular}[c]{@{}l@{}}El sistema deber� permitir\\ la edici�n de un rol existente\end{tabular}                                          \\ \cline{1-1} \cline{4-5} 
	9                                          &                                &                                    & \begin{tabular}[c]{@{}l@{}}Eliminar rol\\ existente\end{tabular}      & \begin{tabular}[c]{@{}l@{}}El sistema deber� permitir\\ la eliminaci�n de un rol\\ existente\end{tabular}                                    \\ \cline{1-1} \cline{4-5} 
	10                                         &                                &                                    & Visualizar roles                                                      & \begin{tabular}[c]{@{}l@{}}El sistema deber� permitir\\ listar los roles existentes\\ en el almacenamiento\\ persistente\end{tabular}        \\ \hline
	\caption{Funciones relacionadas a las acciones de usuario}
	\label{analisisTab2}\\
\end{longtable}

\subsection{Funciones asociadas a las acciones en el Acceso al Sistema}

% Please add the following required packages to your document preamble:
% \usepackage{multirow}
% \usepackage{longtable}
% Note: It may be necessary to compile the document several times to get a multi-page table to line up properly
\begin{longtable}{|l|c|c|l|l|}
	\hline
	\multicolumn{5}{|c|}{\textit{\textbf{Funciones del Sistema}}}                                                                                                                                                                                                                                                                          \\ \hline
	\endfirsthead
	%
	\endhead
	%
	\multicolumn{1}{|c|}{\textit{\textbf{\#}}} & \textit{\textbf{ID Categor�a}} & \textit{\textbf{Categor�a}}      & \multicolumn{1}{c|}{\textit{\textbf{Funci�n}}}                           & \multicolumn{1}{c|}{\textit{\textbf{Descripci�n}}}                                                                                         \\ \hline
	1                                          & \multirow{7}{*}{\textbf{AC}}   & \multirow{7}{*}{\textbf{Acceso}} & \begin{tabular}[c]{@{}l@{}}Iniciar sesi�n\\ de usuario\end{tabular}      & \begin{tabular}[c]{@{}l@{}}El sistema deber� permitir\\ iniciar sesi�n en el sistema\end{tabular}                                          \\ \cline{1-1} \cline{4-5} 
	2                                          &                                &                                  & \begin{tabular}[c]{@{}l@{}}Autenticar\\ usuario\end{tabular}             & \begin{tabular}[c]{@{}l@{}}El sistema deber� permitir\\ comprobar las credenciales\\ del usuario para el inicio de\\ sesi�n\end{tabular}   \\ \cline{1-1} \cline{4-5} 
	3                                          &                                &                                  & Cerrar sesi�n                                                            & \begin{tabular}[c]{@{}l@{}}El sistema deber� permitir\\ cerrar la sesi�n actual del\\ usuario\end{tabular}                                 \\ \cline{1-1} \cline{4-5} 
	4                                          &                                &                                  & \begin{tabular}[c]{@{}l@{}}Recuperar cuenta\\ de usuario\end{tabular}    & \begin{tabular}[c]{@{}l@{}}El sistema deber� permitir\\ recuperar la cuenta de usuario\\ en caso de olvidar la contrase�a\end{tabular}     \\ \cline{1-1} \cline{4-5} 
	5                                          &                                &                                  & \begin{tabular}[c]{@{}l@{}}Reestablecer\\ cuenta de usuario\end{tabular} & \begin{tabular}[c]{@{}l@{}}El sistema deber� permitir\\ reestablecer la cuenta de\\ usuario si se solicita una\\ recuperaci�n\end{tabular} \\ \cline{1-1} \cline{4-5} 
	6                                          &                                &                                  & \begin{tabular}[c]{@{}l@{}}Modificar datos\\ de acceso\end{tabular}      & \begin{tabular}[c]{@{}l@{}}El sistema deber� permitir\\ modificar los datos de\\ acceso asociados al inicio\\ de sesi�n\end{tabular}       \\ \cline{1-1} \cline{4-5} 
	7                                          &                                &                                  & \begin{tabular}[c]{@{}l@{}}Notificaci�n de\\ cambios\end{tabular}        & \begin{tabular}[c]{@{}l@{}}El sistema deber� permitir\\ notificar el cambio de datos\\ en las credenciales generado\end{tabular}           \\ \hline
	\caption{Funciones asociadas a la secci�n de Acceso al sistema}
	\label{analisisTab3}\\
\end{longtable}

\subsection{Funciones asociadas a las estad�sticas de uso}

% Please add the following required packages to your document preamble:
% \usepackage{multirow}
% \usepackage{longtable}
% Note: It may be necessary to compile the document several times to get a multi-page table to line up properly
\begin{longtable}{|l|c|l|l|l|}
	\hline
	\multicolumn{5}{|c|}{\textbf{Funciones del Sistema}}                                                                                                                                                                                                                                                                                     \\ \hline
	\endfirsthead
	%
	\endhead
	%
	\multicolumn{1}{|c|}{\textbf{\#}} & \textbf{ID Categor�a}        & \multicolumn{1}{c|}{\textbf{Categor�a}} & \multicolumn{1}{c|}{\textbf{Funci�n}}                                             & \multicolumn{1}{c|}{\textbf{Descripci�n}}                                                                                               \\ \hline
	1                                 & \multirow{5}{*}{\textbf{ES}} & \multirow{5}{*}{\textbf{Estad�sticas}}  & \begin{tabular}[c]{@{}l@{}}Visualizar estad�sticas\\ de uso\end{tabular}          & \begin{tabular}[c]{@{}l@{}}El sistema deber� permitir\\ entregar estad�sticas de uso\\ por usuario y por m�dulo\end{tabular}            \\ \cline{1-1} \cline{4-5} 
	2                                 &                              &                                         & \begin{tabular}[c]{@{}l@{}}Visualizar carga de\\ servidores\end{tabular}          & \begin{tabular}[c]{@{}l@{}}El sistema deber� permitir\\ visualizar la carga actual\\ de los servidores\end{tabular}                     \\ \cline{1-1} \cline{4-5} 
	3                                 &                              &                                         & \begin{tabular}[c]{@{}l@{}}Visualizar carga de\\ sistema de colas\end{tabular}    & \begin{tabular}[c]{@{}l@{}}El sistema deber� permitir\\ visualizar la carga del sistema\\ de colas\end{tabular}                         \\ \cline{1-1} \cline{4-5} 
	4                                 &                              &                                         & \begin{tabular}[c]{@{}l@{}}Visualizar estados\\ de trabajos\end{tabular}          & \begin{tabular}[c]{@{}l@{}}El sistema deber� permitir\\ visualizar los estados de\\ trabajo\end{tabular}                                \\ \cline{1-1} \cline{4-5} 
	5                                 &                              &                                         & \begin{tabular}[c]{@{}l@{}}Visualizar estad�sticas \\ \\ de trabajos\end{tabular} & \begin{tabular}[c]{@{}l@{}}El sistema deber� permitir\\ visualizar estad�sticas\\ relacionadas a los trabajos\\ procesados\end{tabular} \\ \hline
	\caption{Funciones asociadas a las estad�sticas de uso}
	\label{analisisTab4}\\
\end{longtable}

\subsection{Funciones asociadas al m�dulo de estad�sticas en el set de datos}

% Please add the following required packages to your document preamble:
% \usepackage{multirow}
% \usepackage{longtable}
% Note: It may be necessary to compile the document several times to get a multi-page table to line up properly
\begin{longtable}{|l|c|c|l|l|}
	\hline
	\multicolumn{5}{|c|}{\textit{\textbf{Funciones del Sistema}}}                                                                                                                                                                                                                                                                                                                                                         \\ \hline
	\endfirsthead
	%
	\endhead
	%
	\multicolumn{1}{|c|}{\textit{\textbf{\#}}} & \textit{\textbf{ID Categor�a}} & \textit{\textbf{Categor�a}}                                                               & \multicolumn{1}{c|}{\textit{\textbf{Funci�n}}}                                            & \multicolumn{1}{c|}{\textit{\textbf{Descripci�n}}}                                                                                              \\ \hline
	1                                          & \multirow{14}{*}{\textbf{AE}}  & \multirow{14}{*}{\textbf{\begin{tabular}[c]{@{}c@{}}An�lisis\\ estad�stico\end{tabular}}} & \begin{tabular}[c]{@{}l@{}}Estimar estad�sticos\\ en relaci�n a la\\ muestra\end{tabular} & \begin{tabular}[c]{@{}l@{}}El sistema deber� permitir\\ calcular estad�sticos sobre\\ la data entregada\end{tabular}                            \\ \cline{1-1} \cline{4-5} 
	2                                          &                                &                                                                                           & \begin{tabular}[c]{@{}l@{}}Visualizar\\ estad�sticos de\\ datos\end{tabular}              & \begin{tabular}[c]{@{}l@{}}El sistema deber� permitir\\ visualizar estad�sticos\\ sobre la data entregada\end{tabular}                          \\ \cline{1-1} \cline{4-5} 
	3                                          &                                &                                                                                           & Estimar box plot                                                                          & \begin{tabular}[c]{@{}l@{}}El sistema deber� permitir\\ estimar el box plot para un\\ set de datos dado\end{tabular}                            \\ \cline{1-1} \cline{4-5} 
	4                                          &                                &                                                                                           & Visualizar box plot                                                                       & \begin{tabular}[c]{@{}l@{}}El sistema deber� permitir\\ visualizar el box plot \\ \\ generado\end{tabular}                                      \\ \cline{1-1} \cline{4-5} 
	5                                          &                                &                                                                                           & Estimar histograma                                                                        & \begin{tabular}[c]{@{}l@{}}El sistema deber� permitir\\ estimar histogramas por\\ atributos en set de datos dado\end{tabular}                   \\ \cline{1-1} \cline{4-5} 
	6                                          &                                &                                                                                           & \begin{tabular}[c]{@{}l@{}}Visualizar\\ histograma\end{tabular}                           & \begin{tabular}[c]{@{}l@{}}El sistema deber� permitir\\ visualizar el histograma\\ generado\end{tabular}                                        \\ \cline{1-1} \cline{4-5} 
	7                                          &                                &                                                                                           & Estimar bar charts                                                                        & \begin{tabular}[c]{@{}l@{}}El sistema deber� permitir\\ estimar bar charts para\\ atributos con distribuci�n\\ discreta\end{tabular}            \\ \cline{1-1} \cline{4-5} 
	8                                          &                                &                                                                                           & \begin{tabular}[c]{@{}l@{}}Visualizar bar\\ chart\end{tabular}                            & \begin{tabular}[c]{@{}l@{}}El sistema deber� permitir\\ visualizar el bar chart\\ generado\end{tabular}                                         \\ \cline{1-1} \cline{4-5} 
	9                                          &                                &                                                                                           & Estimar pie charts                                                                        & \begin{tabular}[c]{@{}l@{}}El sistema deber� permitir\\ estimar pie charts para\\ atributos con distribuci�n\\ discreta\end{tabular}            \\ \cline{1-1} \cline{4-5} 
	10                                         &                                &                                                                                           & Visualizar pie charts                                                                     & \begin{tabular}[c]{@{}l@{}}El sistema deber� permitir\\ visualizar el pie charts\\ generado\end{tabular}                                        \\ \cline{1-1} \cline{4-5} 
	11                                         &                                &                                                                                           & \begin{tabular}[c]{@{}l@{}}Estimar matrices de\\ correlaci�n\end{tabular}                 & \begin{tabular}[c]{@{}l@{}}El sistema deber� permitir\\ estimar matrices de correlaci�n\\ para un set de datos de inter�s\end{tabular}          \\ \cline{1-1} \cline{4-5} 
	12                                         &                                &                                                                                           & Visualizar heat map                                                                       & \begin{tabular}[c]{@{}l@{}}El sistema deber� permitir\\ visualizar el heat map\\ generado relacionado a la\\ matriz de correlaci�n\end{tabular} \\ \cline{1-1} \cline{4-5} 
	13                                         &                                &                                                                                           & Estimar scatter plot                                                                      & \begin{tabular}[c]{@{}l@{}}El sistema deber� permitir\\ estimar scatter plot para\\ un set de datos de inter�s\end{tabular}                     \\ \cline{1-1} \cline{4-5} 
	14                                         &                                &                                                                                           & \begin{tabular}[c]{@{}l@{}}Visualizar scatter\\ plot\end{tabular}                         & \begin{tabular}[c]{@{}l@{}}El sistema deber� permitir\\ visualizar el scatter plot\\ generado\end{tabular}                                      \\ \hline
	\caption{Funciones asociadas al m�dulo estad�stico.}
	\label{analisisTab5}\\
\end{longtable}

\subsection{Funciones asociadas al set de datos}

% Please add the following required packages to your document preamble:
% \usepackage{multirow}
% \usepackage{longtable}
% Note: It may be necessary to compile the document several times to get a multi-page table to line up properly
\begin{longtable}{|l|c|l|l|l|}
	\hline
	\multicolumn{5}{|c|}{\textit{\textbf{Funciones del Sistema}}}                                                                                                                                                                                                                                                                                                                                                                                \\ \hline
	\endfirsthead
	%
	\endhead
	%
	\multicolumn{1}{|c|}{\textit{\textbf{\#}}} & \textit{\textbf{ID Categor�a}} & \multicolumn{1}{c|}{\textit{\textbf{Categor�a}}}                                            & \multicolumn{1}{c|}{\textit{\textbf{Funci�n}}}                                          & \multicolumn{1}{c|}{\textit{\textbf{Descripci�n}}}                                                                                                                     \\ \hline
	1                                          & \multirow{4}{*}{\textbf{ESD}}  & \multirow{4}{*}{\textbf{\begin{tabular}[c]{@{}l@{}}Evaluaci�n set\\ de datos\end{tabular}}} & \begin{tabular}[c]{@{}l@{}}Revisar correcto\\ estado set de datos\end{tabular}          & \begin{tabular}[c]{@{}l@{}}El sistema deber� permitir\\ revisar el correcto estado\\ del set de datos\end{tabular}                                                     \\ \cline{1-1} \cline{4-5} 
	2                                          &                                &                                                                                             & \begin{tabular}[c]{@{}l@{}}Revisar variables\\ discretas en set\\ de datos\end{tabular} & \begin{tabular}[c]{@{}l@{}}El sistema deber� permitir\\ revisar si el set de datos\\ presenta variables discretas\\ en sus atributos\end{tabular}                      \\ \cline{1-1} \cline{4-5} 
	3                                          &                                &                                                                                             & \begin{tabular}[c]{@{}l@{}}Revisar formato\\ de set de datos\end{tabular}               & \begin{tabular}[c]{@{}l@{}}El sistema deber� permitir\\ revisar el formato del set de\\ datos y chequear si viene\\ seg�n las caracter�sticas\\ impuestas\end{tabular} \\ \cline{1-1} \cline{4-5} 
	4                                          &                                &                                                                                             & \begin{tabular}[c]{@{}l@{}}Alojar set de\\ datos en �rea\\ de trabajo\end{tabular}      & \begin{tabular}[c]{@{}l@{}}El sistema deber� permitir\\ alojar el set de datos en el\\ �rea de trabajo del usuario\end{tabular}                                        \\ \hline
	\caption{Funciones asociadas al set de datos.}
	\label{analisisTab6}\\
	\end{longtable}

\subsection{Funciones asociadas al m�dulo de algoritmos de aprendizaje no supervisado}

% Please add the following required packages to your document preamble:
% \usepackage{multirow}
% \usepackage{longtable}
% Note: It may be necessary to compile the document several times to get a multi-page table to line up properly
\begin{longtable}{|l|c|c|l|l|}
	\hline
	\multicolumn{5}{|c|}{\textit{\textbf{Funciones del Sistema}}}                                                                                                                                                                                                                                                                                                                                                                                     \\ \hline
	\endfirsthead
	%
	\endhead
	%
	\multicolumn{1}{|c|}{\textit{\textbf{\#}}} & \textit{\textbf{ID Categor�a}} & \textit{\textbf{Categor�a}}                                                                        & \multicolumn{1}{c|}{\textit{\textbf{Funci�n}}}                                       & \multicolumn{1}{c|}{\textit{\textbf{Descripci�n}}}                                                                                                                      \\ \hline
	1                                          & \multirow{9}{*}{MANS}          & \multirow{9}{*}{\begin{tabular}[c]{@{}c@{}}Modelos de\\ Aprendizaje No\\ Supervisado\end{tabular}} & \begin{tabular}[c]{@{}l@{}}Implementar algoritmo\\ K-Means\end{tabular}              & \begin{tabular}[c]{@{}l@{}}El sistema deber� permitir\\ implementar el algoritmo\\ K-Means para entorno de\\ aprendizaje no supervisado\end{tabular}                    \\ \cline{1-1} \cline{4-5} 
	2                                          &                                &                                                                                                    & \begin{tabular}[c]{@{}l@{}}Implementar algoritmo\\ Mean Shift\end{tabular}           & \begin{tabular}[c]{@{}l@{}}El sistema deber� permitir\\ implementar el algoritmo\\ Mean Shift para entorno\\ de aprendizaje no supervisado\end{tabular}                 \\ \cline{1-1} \cline{4-5} 
	3                                          &                                &                                                                                                    & \begin{tabular}[c]{@{}l@{}}Implementar algoritmo\\ Affinity Propagation\end{tabular} & \begin{tabular}[c]{@{}l@{}}El sistema deber� permitir\\ implementar el algoritmo \\ \\ Affinity Propagation para\\ entorno de aprendizaje no\\ supervisado\end{tabular} \\ \cline{1-1} \cline{4-5} 
	4                                          &                                &                                                                                                    & \begin{tabular}[c]{@{}l@{}}Implementar algoritmo\\ DBScan\end{tabular}               & \begin{tabular}[c]{@{}l@{}}El sistema deber� permitir\\ implementar el algoritmo\\ DBScan para entorno de\\ aprendizaje no supervisado\end{tabular}                     \\ \cline{1-1} \cline{4-5} 
	5                                          &                                &                                                                                                    & \begin{tabular}[c]{@{}l@{}}Implementar algoritmo\\ Aglomerativos\end{tabular}        & \begin{tabular}[c]{@{}l@{}}El sistema deber� permitir\\ implementar el algoritmo\\ Aglomerativos para entorno\\ de aprendizaje no supervisado\end{tabular}              \\ \cline{1-1} \cline{4-5} 
	6                                          &                                &                                                                                                    & \begin{tabular}[c]{@{}l@{}}Implementar algoritmo\\ jerarquizado\end{tabular}         & \begin{tabular}[c]{@{}l@{}}El sistema deber� permitir\\ implementar el algoritmo\\ Jerarquizado para entorno\\ de aprendizaje no supervisado\end{tabular}               \\ \cline{1-1} \cline{4-5} 
	7                                          &                                &                                                                                                    & \begin{tabular}[c]{@{}l@{}}Implementar algoritmo\\ SOM\end{tabular}                  & \begin{tabular}[c]{@{}l@{}}El sistema deber� permitir\\ implementar el algoritmo\\ Self Organization Map para\\ entorno de aprendizaje no\\ supervisado\end{tabular}    \\ \cline{1-1} \cline{4-5} 
	8                                          &                                &                                                                                                    & \begin{tabular}[c]{@{}l@{}}Evaluar las particiones\\ generadas\end{tabular}          & \begin{tabular}[c]{@{}l@{}}El sistema deber� permitir\\ la evaluaci�n de las particiones\\ generadas mediante Coeficientes\\ determinados\end{tabular}                  \\ \cline{1-1} \cline{4-5} 
	9                                          &                                &                                                                                                    & Reportar resultados                                                                  & \begin{tabular}[c]{@{}l@{}}El sistema deber� permitir \\ \\ reportar los resultados\\ obtenidos para su posterior\\ an�lisis\end{tabular}                               \\ \hline
	\caption{Funciones asociadas a los m�dulos de clustering}
	\label{analisisTab7}\\
	\end{longtable}
	
	
\subsection{Funciones asociadas al m�dulo de clasificaciones}

% Please add the following required packages to your document preamble:
% \usepackage{multirow}
% \usepackage{longtable}
% Note: It may be necessary to compile the document several times to get a multi-page table to line up properly
\begin{longtable}{|l|c|l|l|l|}
	\hline
	\multicolumn{5}{|c|}{\textit{\textbf{Funciones del Sistema}}}                                                                                                                                                                                                                                                                                                                                                                                                \\ \hline
	\endfirsthead
	%
	\endhead
	%
	\multicolumn{1}{|c|}{\textit{\textbf{\#}}} & \textit{\textbf{ID Categor�a}} & \multicolumn{1}{c|}{\textit{\textbf{Categor�a}}}                                                                       & \multicolumn{1}{c|}{\textit{\textbf{Funci�n}}}                                      & \multicolumn{1}{c|}{\textit{\textbf{Descripci�n}}}                                                                                                              \\ \hline
	1                                          & \multirow{15}{*}{\textbf{MAS}} & \multirow{15}{*}{\textit{\textbf{\begin{tabular}[c]{@{}l@{}}Modelos de\\ Aprendizaje \\ \\ Supervisado\end{tabular}}}} & \begin{tabular}[c]{@{}l@{}}Implementar algoritmo\\ Naive Bayes\end{tabular}         & \begin{tabular}[c]{@{}l@{}}El sistema deber� permitir\\ implementar el algoritmo\\ Naive Bayes para entorno\\ de aprendizaje supervisado\end{tabular}           \\ \cline{1-1} \cline{4-5} 
	2                                          &                                &                                                                                                                        & \begin{tabular}[c]{@{}l@{}}Implementar algoritmo\\ KNN\end{tabular}                 & \begin{tabular}[c]{@{}l@{}}El sistema deber� permitir\\ implementar el algoritmo\\ KNN para entorno de \\ \\ aprendizaje supervisado\end{tabular}               \\ \cline{1-1} \cline{4-5} 
	3                                          &                                &                                                                                                                        & \begin{tabular}[c]{@{}l@{}}Implementar algoritmo\\ Random Forest\end{tabular}       & \begin{tabular}[c]{@{}l@{}}El sistema deber� permitir\\ implementar el algoritmo\\ Random Forest para\\ entorno de aprendizaje\\ supervisado\end{tabular}       \\ \cline{1-1} \cline{4-5} 
	4                                          &                                &                                                                                                                        & \begin{tabular}[c]{@{}l@{}}Implementar algoritmo\\ AdaBoost\end{tabular}            & \begin{tabular}[c]{@{}l@{}}El sistema deber� permitir\\ implementar el algoritmo\\ AdaBoost para entorno\\ de aprendizaje supervisado\end{tabular}              \\ \cline{1-1} \cline{4-5} 
	5                                          &                                &                                                                                                                        & \begin{tabular}[c]{@{}l@{}}Implementar algoritmo\\ SVM\end{tabular}                 & \begin{tabular}[c]{@{}l@{}}El sistema deber� permitir\\ implementar el algoritmo\\ SVM para entorno de\\ aprendizaje supervisado\end{tabular}                   \\ \cline{1-1} \cline{4-5} 
	6                                          &                                &                                                                                                                        & \begin{tabular}[c]{@{}l@{}}Implementar algoritmo\\ NuSVC\end{tabular}               & \begin{tabular}[c]{@{}l@{}}El sistema deber� permitir\\ implementar el algoritmo\\ NuSVC para entorno de\\ aprendizaje supervisado\end{tabular}                 \\ \cline{1-1} \cline{4-5} 
	7                                          &                                &                                                                                                                        & \begin{tabular}[c]{@{}l@{}}Implementar algoritmo\\ MLP\end{tabular}                 & \begin{tabular}[c]{@{}l@{}}El sistema deber� permitir\\ implementar el algoritmo\\ MLP para entorno de\\ aprendizaje supervisado\end{tabular}                   \\ \cline{1-1} \cline{4-5} 
	8                                          &                                &                                                                                                                        & \begin{tabular}[c]{@{}l@{}}Implementar algoritmo\\ �rboles de Decisi�n\end{tabular} & \begin{tabular}[c]{@{}l@{}}El sistema deber� permitir\\ implementar el algoritmo\\ �rboles de decisi�n para\\ entorno de aprendizaje\\ supervisado\end{tabular} \\ \cline{1-1} \cline{4-5} 
	9                                          &                                &                                                                                                                        & \begin{tabular}[c]{@{}l@{}}Implementar validaci�n\\ cruzada\end{tabular}            & \begin{tabular}[c]{@{}l@{}}El sistema deber� permitir\\ implementar los algoritmos\\ de validaci�n cruzada para\\ cada algoritmo\end{tabular}                   \\ \cline{1-1} \cline{4-5} 
	10                                         &                                &                                                                                                                        & \begin{tabular}[c]{@{}l@{}}Implementar validaci�n\\ LOU\end{tabular}                & \begin{tabular}[c]{@{}l@{}}El sistema deber� permitir\\ implementar los algoritmos\\ de validaci�n LOU para cada\\ algoritmo\end{tabular}                       \\ \cline{1-1} \cline{4-5} 
	11                                         &                                &                                                                                                                        & Estimar Curva ROC                                                                   & \begin{tabular}[c]{@{}l@{}}El sistema deber� permitir\\ estimar la curva roc para el\\ modelo resultante\end{tabular}                                           \\ \cline{1-1} \cline{4-5} 
	12                                         &                                &                                                                                                                        & \begin{tabular}[c]{@{}l@{}}Estimar curva de\\ validaci�n\end{tabular}               & \begin{tabular}[c]{@{}l@{}}El sistema deber� permitir\\ estimar la curva de validaci�n\\ para el modelo resultante\end{tabular}                                 \\ \cline{1-1} \cline{4-5} 
	13                                         &                                &                                                                                                                        & \begin{tabular}[c]{@{}l@{}}Estimar matriz de\\ confusi�n\end{tabular}               & \begin{tabular}[c]{@{}l@{}}El sistema deber� permitir\\ estimar la matriz de \\ \\ confusi�n para el modelo\\ resultante\end{tabular}                           \\ \cline{1-1} \cline{4-5} 
	14                                         &                                &                                                                                                                        & \begin{tabular}[c]{@{}l@{}}Estimar las medidas de\\ desempe�o\end{tabular}          & \begin{tabular}[c]{@{}l@{}}El sistema deber� permitir\\ estimar las medidas de\\ desempe�o obtenidas para\\ el modelo resultante\end{tabular}                   \\ \cline{1-1} \cline{4-5} 
	15                                         &                                &                                                                                                                        & Alojar resultados en job                                                            & \begin{tabular}[c]{@{}l@{}}El sistema deber� permitir\\ alojar los resultados\\ obtenidos en el job relacionado\\ al �rea de trabajo\end{tabular}               \\ \hline
	\caption{Funciones asociadas al m�dulo de clasificaci�n.}
	\label{analisisTab8}\\
\end{longtable}

\subsection{Funciones asociadas al m�dulo de an�lisis de caracter�sticas}

% Please add the following required packages to your document preamble:
% \usepackage{multirow}
% \usepackage{longtable}
% Note: It may be necessary to compile the document several times to get a multi-page table to line up properly
\begin{longtable}{|l|l|l|l|l|}
	\hline
	\multicolumn{5}{|c|}{\textit{\textbf{Funciones del Sistema}}}                                                                                                                                                                                                                                                                                                                                                                                                                  \\ \hline
	\endfirsthead
	%
	\endhead
	%
	\multicolumn{1}{|c|}{\textit{\textbf{\#}}} & \multicolumn{1}{c|}{\textit{\textbf{ID Categor�a}}} & \multicolumn{1}{c|}{\textit{\textbf{Categor�a}}}                                       & \multicolumn{1}{c|}{\textit{\textbf{Funci�n}}}                                                            & \multicolumn{1}{c|}{\textit{\textbf{Descripci�n}}}                                                                                                                     \\ \hline
	1                                          & \multirow{5}{*}{AF}                                 & \multirow{5}{*}{\begin{tabular}[c]{@{}l@{}}An�lisis de\\ Caracter�sticas\end{tabular}} & Implementar PCA                                                                                           & \begin{tabular}[c]{@{}l@{}}El sistema deber� permitir\\ la implementaci�n del algoritmo\\ PCA y exponer los resultados\\ generados\end{tabular}                        \\ \cline{1-1} \cline{4-5} 
	2                                          &                                                     &                                                                                        & \begin{tabular}[c]{@{}l@{}}Implementar\\ \\ Deformaciones\\ de Espacio\end{tabular}                       & \begin{tabular}[c]{@{}l@{}}El sistema deber� permitir\\ la implementaci�n de la\\ evaluaci�n de caracter�sticas\\ mediante Importancia en\\ Random Forest\end{tabular} \\ \cline{1-1} \cline{4-5} 
	3                                          &                                                     &                                                                                        & \begin{tabular}[c]{@{}l@{}}Implementar\\ \\ algoritmos de\\ Mutual Information\end{tabular}               & \begin{tabular}[c]{@{}l@{}}El sistema deber� permitir\\ la implementaci�n de los\\ algoritmos referentes a\\ mutual information\end{tabular}                           \\ \cline{1-1} \cline{4-5} 
	4                                          &                                                     &                                                                                        & \begin{tabular}[c]{@{}l@{}}Implementar\\ algoritmode correlaciones\end{tabular}                           & \begin{tabular}[c]{@{}l@{}}El sistema deber� permitir\\ implementar y analizar los\\ resultados de an�lisis de\\ correlaciones\end{tabular}                            \\ \cline{1-1} \cline{4-5} 
	5                                          &                                                     &                                                                                        & \begin{tabular}[c]{@{}l@{}}Implementar\\ algoritmos basados\\ en distribuciones\\ bayesianas\end{tabular} & \begin{tabular}[c]{@{}l@{}}El sistema deber� permitir\\ la implementaci�n de m�todos\\ bayesianos para an�lisis de\\ caracter�sticas\end{tabular}                      \\ \hline
	\caption{Funciones asociadas al m�dulo de an�lisis de caracter�sticas}
	\label{analisisTab9}\\
	\end{longtable}
	
	\subsection{Funciones asociada al Sistema de Colas}
	
	% Please add the following required packages to your document preamble:
	% \usepackage{multirow}
	% \usepackage{longtable}
	% Note: It may be necessary to compile the document several times to get a multi-page table to line up properly
	\begin{longtable}{|l|c|c|l|l|}
		\hline
		\multicolumn{5}{|c|}{\textit{\textbf{Funciones del Sistema}}}                                                                                                                                                                                                                                                                                                                    \\ \hline
		\endfirsthead
		%
		\endhead
		%
		\multicolumn{1}{|c|}{\textit{\textbf{\#}}} & \textit{\textbf{ID Categor�a}} & \textit{\textbf{Categor�a}}       & \multicolumn{1}{c|}{\textit{\textbf{Funci�n}}}                                   & \multicolumn{1}{c|}{\textit{\textbf{Descripci�n}}}                                                                                                                          \\ \hline
		1                                          & \multirow{7}{*}{SC}            & \multirow{7}{*}{Sistema de Colas} & \begin{tabular}[c]{@{}l@{}}Registrar procesos en\\ sistema de colas\end{tabular} & \begin{tabular}[c]{@{}l@{}}El sistema deber� permitir\\ el registro de nuevos procesos\\ en sistema de colas y establecer\\ las configuraciones sobre el mismo\end{tabular} \\ \cline{1-1} \cline{4-5} 
		2                                          &                                &                                   & \begin{tabular}[c]{@{}l@{}}Editar proceso en\\ sistema de cola\end{tabular}      & \begin{tabular}[c]{@{}l@{}}El sistema deber� permitir\\ modificar los procesos encolados,\\ cambiarles estado, modificar\\ prioridades, etc\end{tabular}                    \\ \cline{1-1} \cline{4-5} 
		3                                          &                                &                                   & \begin{tabular}[c]{@{}l@{}}Eliminar proceso de\\ sistema de cola\end{tabular}    & \begin{tabular}[c]{@{}l@{}}El sistema deber� permitir\\ remover procesos en sistemas\\ de colas e identificar las\\ consecuencias del hecho.\end{tabular}                   \\ \cline{1-1} \cline{4-5} 
		4                                          &                                &                                   & \begin{tabular}[c]{@{}l@{}}Mostrar estado\\ de cola\end{tabular}                 & \begin{tabular}[c]{@{}l@{}}El sistema deber� permitir\\ consultar los procesos en\\ sistema de colas y exponer\\ las caracter�sticas de ellos.\end{tabular}                 \\ \cline{1-1} \cline{4-5} 
		5                                          &                                &                                   & \begin{tabular}[c]{@{}l@{}}Notificar estados\\ de procesos\end{tabular}          & \begin{tabular}[c]{@{}l@{}}El sistema deber� permitir\\ notificar a los usuarios los\\ estados de los procesos\end{tabular}                                                 \\ \cline{1-1} \cline{4-5} 
		6                                          &                                &                                   & \begin{tabular}[c]{@{}l@{}}Notificar finalizaci�n\\ de trabajos\end{tabular}     & \begin{tabular}[c]{@{}l@{}}El sistema deber� permitir\\ la notificaci�n a los usuarios\\ cuando un job finalice\end{tabular}                                                \\ \cline{1-1} \cline{4-5} 
		7                                          &                                &                                   & \begin{tabular}[c]{@{}l@{}}Revisar procesos\\ en cola\end{tabular}               & \begin{tabular}[c]{@{}l@{}}El sistema deber� permitir\\ la revisi�n constante de los\\ procesos encolados\end{tabular}                                                      \\ \hline
		\caption{Funciones asociadas al Sistema de Colas}
		\label{analisisTab10}\\
		\end{longtable}
		
		
	\subsection{Funciones asociadas al Sistema de Notificaci�n}
	
	
	% Please add the following required packages to your document preamble:
	% \usepackage{multirow}
	% \usepackage{longtable}
	% Note: It may be necessary to compile the document several times to get a multi-page table to line up properly
	\begin{longtable}{|l|l|l|l|l|}
		\hline
		\multicolumn{5}{|c|}{\textit{\textbf{Funciones del Sistema}}}                                                                                                                                                                                                                                                                                                                                                                           \\ \hline
		\endfirsthead
		%
		\endhead
		%
		\multicolumn{1}{|c|}{\textit{\textbf{\#}}} & \multicolumn{1}{c|}{\textit{\textbf{ID Categor�a}}} & \multicolumn{1}{c|}{\textit{\textbf{Categor�a}}}                                     & \multicolumn{1}{c|}{\textit{\textbf{Funci�n}}}                                         & \multicolumn{1}{c|}{\textit{\textbf{Descripci�n}}}                                                                                                   \\ \hline
		1                                          & \multirow{3}{*}{SN}                                 & \multirow{3}{*}{\begin{tabular}[c]{@{}l@{}}Sistema de\\ notificaciones\end{tabular}} & \begin{tabular}[c]{@{}l@{}}Enviar mensajes de\\ notificaciones a usuarios\end{tabular} & \begin{tabular}[c]{@{}l@{}}El sistema deber� permitir\\ la notificaci�n de nuevos\\ mensajes a usuarios v�a\\ email o notificaci�n push\end{tabular} \\ \cline{1-1} \cline{4-5} 
		2                                          &                                                     &                                                                                      & \begin{tabular}[c]{@{}l@{}}Configurar mensajes\\ de notificaciones\end{tabular}        & \begin{tabular}[c]{@{}l@{}}El sistema deber� permitir\\ configurar y condicionar\\ las notificaciones que se emiten\end{tabular}                     \\ \cline{1-1} \cline{4-5} 
		3                                          &                                                     &                                                                                      & \begin{tabular}[c]{@{}l@{}}Generar notificaciones\\ globales\end{tabular}              & \begin{tabular}[c]{@{}l@{}}El sistema deber� permitir\\ generar notificaciones\\ generales a todos los usuarios\\ existentes\end{tabular}            \\ \hline
		\caption{Funciones asociadas al Sistema de Notificaciones}
		\label{analisisTab11}\\
	\end{longtable}


	\subsection{Funciones asociadas al Sistema de Almacenamiento Persistente}
	
	% Please add the following required packages to your document preamble:
	% \usepackage{multirow}
	% \usepackage{longtable}
	% Note: It may be necessary to compile the document several times to get a multi-page table to line up properly
	\begin{longtable}{|l|l|l|l|l|}
		\hline
		\multicolumn{5}{|c|}{\textit{\textbf{Funciones del Sistema}}}                                                                                                                                                                                                                                                                                                                                                                                                                \\ \hline
		\endfirsthead
		%
		\endhead
		%
		\multicolumn{1}{|c|}{\textit{\textbf{\#}}} & \multicolumn{1}{c|}{\textit{\textbf{ID Categor�a}}} & \multicolumn{1}{c|}{\textit{\textbf{Categor�a}}}                                      & \multicolumn{1}{c|}{\textit{\textbf{Funci�n}}}                                      & \multicolumn{1}{c|}{\textit{\textbf{Descripci�n}}}                                                                                                                                          \\ \hline
		1                                          & \multirow{4}{*}{AP}                                 & \multirow{4}{*}{\begin{tabular}[c]{@{}l@{}}Almacenamiento\\ Persistente\end{tabular}} & \begin{tabular}[c]{@{}l@{}}Responder solicitudes\\ de muestra de datos\end{tabular} & \begin{tabular}[c]{@{}l@{}}El sistema deber� permitir\\ responder  a las solicitudes\\ de consulta de datos desde\\ el sistema de almacenamiento\\ persistente\end{tabular}                 \\ \cline{1-1} \cline{4-5} 
		2                                          &                                                     &                                                                                       & \begin{tabular}[c]{@{}l@{}}Registrar nuevos\\ elementos\end{tabular}                & \begin{tabular}[c]{@{}l@{}}El sistema deber� permitir\\ registrar nuevos elementos\\ en el sistema de almacenamiento\\ persistente\end{tabular}                                             \\ \cline{1-1} \cline{4-5} 
		3                                          &                                                     &                                                                                       & \begin{tabular}[c]{@{}l@{}}Modificar elementos\\ en el sistema\end{tabular}         & \begin{tabular}[c]{@{}l@{}}El sistema deber� permitir\\ editar registros y mantener\\ la concordancia de las relaciones\\ asociadas al sistema de\\ almacenamiento persistente\end{tabular} \\ \cline{1-1} \cline{4-5} 
		4                                          &                                                     &                                                                                       & \begin{tabular}[c]{@{}l@{}}Eliminar elementos\\ en el sistema\end{tabular}          & \begin{tabular}[c]{@{}l@{}}El sistema deber� permitir\\ eliminar registros en el sistema\\ de almacenamiento persistente\end{tabular}                                                       \\ \hline
		\caption{Funciones asociadas al Sistema de Almacenamiento Persistente}
		\label{analisisTab12}\\
		\end{longtable}
		
\section{Atributos del Sistema}

Los atributos del sistema, representan variables cuantitativas asociadas a caracter�sticas de �ste, es decir, �ndices que permiten identificar que el software cumple con caracter�sticas deseables a modo general.

Cada uno de los posibles atributos que posee el sistema se exponen a continuaci�n en la Tabla \ref{analisisTab13}, donde se detalla el atribuo con su ID y su descripci�n o forma de medici�n

% Please add the following required packages to your document preamble:
% \usepackage{graphicx}
\begin{table}[!h]
	\resizebox{\textwidth}{!}{%
		\begin{tabular}{|l|l|l|}
			\hline
			\multicolumn{3}{|c|}{\textit{\textbf{Atributos del Sistema}}}                                                                                          \\ \hline
			\multicolumn{1}{|c|}{\textit{\textbf{ID}}} & \multicolumn{1}{c|}{\textit{\textbf{Item}}} & \multicolumn{1}{c|}{\textit{\textbf{Descripci�n medici�n}}} \\ \hline
			D                                          & Disponibilidad                              & 99\% Up Time                                                \\ \hline
			P                                          & Persistencia                                & 99\% en base a configuraci�n servidores                     \\ \hline
			TE                                         & Tiempo de Espera                            & Menor a 1 minuto por acci�n                                 \\ \hline
			U                                          & Usabilidad                                  & Menos de 5 clicks por acci�n                                \\ \hline
			TF                                         & Tolerancia a Fallos                         & 1 Fallo cada 1000 ejecuciones                               \\ \hline
			R                                          & Respaldos                                   & Data dispuesta en distintos servidores seg�n configuraci�n  \\ \hline
			CM                                         & Conexiones M�ltiples                        & Hasta 1000 conexiones m�ltiples                             \\ \hline
			SM                                         & Solicitudes M�ltiples                       & Hasta 100 ejecuciones en sistema de colas                   \\ \hline
			S                                          & Seguridad                                   & 99.9\% tolerable a ingresos corrompidos                     \\ \hline
		\end{tabular}%
	}
	\caption{Atributos presentes en el sistema}
	\label{analisisTab13}
\end{table}

Es importante asociar que cada una de las funciones expuestas cumple con a lo menos un atributo de los nombrados en la Tabla \ref{analisisTab13}, la cruza de esta informaci�n, es decir los atributos por funci�n se expone en la tabla \ref{analisisTab14}

% Please add the following required packages to your document preamble:
% \usepackage{multirow}
% \usepackage{longtable}
% Note: It may be necessary to compile the document several times to get a multi-page table to line up properly
\begin{longtable}{|l|l|l|l|}
	\hline
	\multicolumn{4}{|c|}{\textit{\textbf{Atributos por Funci�n}}}                                                                                                                                                                                     \\ \hline
	\endfirsthead
	%
	\endhead
	%
	\multicolumn{1}{|c|}{\textit{\textbf{\#}}} & \multicolumn{1}{c|}{\textit{\textbf{ID}}} & \multicolumn{1}{c|}{\textit{\textbf{Funci�n}}}                                                        & \multicolumn{1}{c|}{\textit{\textbf{Atributos}}} \\ \hline
	1                                          & \multirow{10}{*}{US}                      & Registrar nuevo usuario                                                                               & D, P, TE, U, TF                                  \\ \cline{1-1} \cline{3-4} 
	2                                          &                                           & Editar usuario existente                                                                              & D, P, TE, U, TF                                  \\ \cline{1-1} \cline{3-4} 
	3                                          &                                           & Eliminar usuario existente                                                                            & D, P, TE, U, TF                                  \\ \cline{1-1} \cline{3-4} 
	4                                          &                                           & Visualizar usuario                                                                                    & D, P, TE, U, TF                                  \\ \cline{1-1} \cline{3-4} 
	5                                          &                                           & Crear �reas de trabajo                                                                                & D, P, TE, U, TF                                  \\ \cline{1-1} \cline{3-4} 
	6                                          &                                           & Eliminar �reas de trabajo                                                                             & D, P, TE, U, TF                                  \\ \cline{1-1} \cline{3-4} 
	7                                          &                                           & Registrar roles                                                                                       & D, P, TE, U, TF                                  \\ \cline{1-1} \cline{3-4} 
	8                                          &                                           & Editar roles existente                                                                                & D, P, TE, U, TF                                  \\ \cline{1-1} \cline{3-4} 
	9                                          &                                           & Eliminar rol existente                                                                                & D, P, TE, U, TF                                  \\ \cline{1-1} \cline{3-4} 
	10                                         &                                           & Visualizar roles                                                                                      & D, P, TE, U, TF                                  \\ \hline
	1                                          & \multirow{7}{*}{AC}                       & Iniciar sesi�n de usuario                                                                             & D, S, CM, TE                                     \\ \cline{1-1} \cline{3-4} 
	2                                          &                                           & Autenticar usuario                                                                                    & D, S, CM, TE                                     \\ \cline{1-1} \cline{3-4} 
	3                                          &                                           & Cerrar sesi�n                                                                                         & D, S, CM, TE                                     \\ \cline{1-1} \cline{3-4} 
	4                                          &                                           & Recuperar cuenta de usuario                                                                           & D, S, CM, TE, TF                                 \\ \cline{1-1} \cline{3-4} 
	5                                          &                                           & Reestablecer cuenta de usuario                                                                        & D, S, CM, TE, P                                  \\ \cline{1-1} \cline{3-4} 
	6                                          &                                           & Modificar datos de acceso                                                                             & D, S, CM, TE, TF                                 \\ \cline{1-1} \cline{3-4} 
	7                                          &                                           & Notificaci�n de cambios                                                                               & D, S, CM, TE, TF                                 \\ \hline
	1                                          & \multirow{5}{*}{ES}                       & Visualizar estad�sticas de uso                                                                        & U, SM, TE                                        \\ \cline{1-1} \cline{3-4} 
	2                                          &                                           & Visualizar carga de servidores                                                                        & U, SM, TE                                        \\ \cline{1-1} \cline{3-4} 
	3                                          &                                           & Visualizar carga de sistema de colas                                                                  & U, SM, TE                                        \\ \cline{1-1} \cline{3-4} 
	4                                          &                                           & Visualizar estados de trabajos                                                                        & U, SM, TE                                        \\ \cline{1-1} \cline{3-4} 
	5                                          &                                           & Visualizar estad�sticas de trabajos                                                                   & U, SM, TE                                        \\ \hline
	1                                          & \multirow{14}{*}{AE}                      & Estimar estad�sticos en relaci�n a la muestra                                                         & U, SM, TE, CM, TF                                \\ \cline{1-1} \cline{3-4} 
	2                                          &                                           & Visualizar estad�sticos de datos                                                                      & U, SM, TE, CM, TF                                \\ \cline{1-1} \cline{3-4} 
	3                                          &                                           & Estimar box plot                                                                                      & U, SM, TE, CM, TF                                \\ \cline{1-1} \cline{3-4} 
	4                                          &                                           & Visualizar box plot                                                                                   & U, SM, TE, CM, TF                                \\ \cline{1-1} \cline{3-4} 
	5                                          &                                           & Estimar histograma                                                                                    & U, SM, TE, CM, TF                                \\ \cline{1-1} \cline{3-4} 
	6                                          &                                           & Visualizar histograma                                                                                 & U, SM, TE, CM, TF                                \\ \cline{1-1} \cline{3-4} 
	7                                          &                                           & Estimar bar charts                                                                                    & U, SM, TE, CM, TF                                \\ \cline{1-1} \cline{3-4} 
	8                                          &                                           & Visualizar bar chart                                                                                  & U, SM, TE, CM, TF                                \\ \cline{1-1} \cline{3-4} 
	9                                          &                                           & Estimar pie charts                                                                                    & U, SM, TE, CM, TF                                \\ \cline{1-1} \cline{3-4} 
	10                                         &                                           & Visualizar pie charts                                                                                 & U, SM, TE, CM, TF                                \\ \cline{1-1} \cline{3-4} 
	11                                         &                                           & Estimar matrices de correlaci�n                                                                       & U, SM, TE, CM, TF                                \\ \cline{1-1} \cline{3-4} 
	12                                         &                                           & Visualizar heat map                                                                                   & U, SM, TE, CM, TF                                \\ \cline{1-1} \cline{3-4} 
	13                                         &                                           & Estimar scatter plot                                                                                  & U, SM, TE, CM, TF                                \\ \cline{1-1} \cline{3-4} 
	14                                         &                                           & Visualizar scatter plot                                                                               & U, SM, TE, CM, TF                                \\ \hline
	1                                          & \multirow{4}{*}{ESD}                      & Revisar correcto estado set de datos                                                                  & U, SM, TE, CM, TF                                \\ \cline{1-1} \cline{3-4} 
	2                                          &                                           & Revisar variables discretas en set de datos                                                           & U, SM, TE, CM, TF                                \\ \cline{1-1} \cline{3-4} 
	3                                          &                                           & Revisar formato de set de datos                                                                       & U, SM, TE, CM, TF                                \\ \cline{1-1} \cline{3-4} 
	4                                          &                                           & Alojar set de datos en �rea de trabajo                                                                & U, SM, TE, CM, TF                                \\ \hline
	1                                          & \multirow{9}{*}{MANS}                     & Implementar algoritmo K-Means                                                                         & U, SM, TE, CM, TF                                \\ \cline{1-1} \cline{3-4} 
	2                                          &                                           & Implementar algoritmo Mean Shift                                                                      & U, SM, TE, CM, TF                                \\ \cline{1-1} \cline{3-4} 
	3                                          &                                           & Implementar algoritmo Affinity Propagation                                                            & U, SM, TE, CM, TF                                \\ \cline{1-1} \cline{3-4} 
	4                                          &                                           & Implementar algoritmo DBScan                                                                          & U, SM, TE, CM, TF                                \\ \cline{1-1} \cline{3-4} 
	5                                          &                                           & Implementar algoritmo Aglomerativos                                                                   & U, SM, TE, CM, TF                                \\ \cline{1-1} \cline{3-4} 
	6                                          &                                           & Implementar algoritmo jerarquizado                                                                    & U, SM, TE, CM, TF                                \\ \cline{1-1} \cline{3-4} 
	7                                          &                                           & Implementar algoritmo SOM                                                                             & U, SM, TE, CM, TF                                \\ \cline{1-1} \cline{3-4} 
	8                                          &                                           & Evaluar las particiones generadas                                                                     & U, SM, TE, CM, TF                                \\ \cline{1-1} \cline{3-4} 
	9                                          &                                           & Reportar resultados                                                                                   & U, SM, TE, CM, TF                                \\ \hline
	1                                          & \multirow{15}{*}{MAS}                     & Implementar algoritmo Naive Bayes                                                                     & U, SM, TE, CM, TF                                \\ \cline{1-1} \cline{3-4} 
	2                                          &                                           & Implementar algoritmo KNN                                                                             & U, SM, TE, CM, TF                                \\ \cline{1-1} \cline{3-4} 
	3                                          &                                           & Implementar algoritmo Random Forest                                                                   & U, SM, TE, CM, TF                                \\ \cline{1-1} \cline{3-4} 
	4                                          &                                           & Implementar algoritmo AdaBoost                                                                        & U, SM, TE, CM, TF                                \\ \cline{1-1} \cline{3-4} 
	5                                          &                                           & Implementar algoritmo SVM                                                                             & U, SM, TE, CM, TF                                \\ \cline{1-1} \cline{3-4} 
	6                                          &                                           & Implementar algoritmo NuSVC                                                                           & U, SM, TE, CM, TF                                \\ \cline{1-1} \cline{3-4} 
	7                                          &                                           & Implementar algoritmo MLP                                                                             & U, SM, TE, CM, TF                                \\ \cline{1-1} \cline{3-4} 
	8                                          &                                           & Implementar algoritmo �rboles de Decisi�n                                                             & U, SM, TE, CM, TF                                \\ \cline{1-1} \cline{3-4} 
	9                                          &                                           & Implementar validaci�n cruzada                                                                        & U, SM, TE, CM, TF                                \\ \cline{1-1} \cline{3-4} 
	10                                         &                                           & Implementar validaci�n LOU                                                                            & U, SM, TE, CM, TF                                \\ \cline{1-1} \cline{3-4} 
	11                                         &                                           & Estimar Curva ROC                                                                                     & U, SM, TE, CM, TF                                \\ \cline{1-1} \cline{3-4} 
	12                                         &                                           & Estimar curva de validaci�n                                                                           & U, SM, TE, CM, TF                                \\ \cline{1-1} \cline{3-4} 
	13                                         &                                           & Estimar matriz de confusi�n                                                                           & U, SM, TE, CM, TF                                \\ \cline{1-1} \cline{3-4} 
	14                                         &                                           & Estimar las medidas de desempe�o                                                                      & U, SM, TE, CM, TF                                \\ \cline{1-1} \cline{3-4} 
	15                                         &                                           & Alojar resultados en job                                                                              & U, SM, TE, CM, TF                                \\ \hline
	1                                          & \multirow{5}{*}{AF}                       & Implementar PCA                                                                                       & U, SM, TE, CM, TF                                \\ \cline{1-1} \cline{3-4} 
	2                                          &                                           & Implementar Deformaciones de Espacio                                                                  & U, SM, TE, CM, TF                                \\ \cline{1-1} \cline{3-4} 
	3                                          &                                           & Implementar algoritmos de Mutual Information                                                          & U, SM, TE, CM, TF                                \\ \cline{1-1} \cline{3-4} 
	4                                          &                                           & Implementar algoritmo de correlaciones                                                                & U, SM, TE, CM, TF                                \\ \cline{1-1} \cline{3-4} 
	5                                          &                                           & \begin{tabular}[c]{@{}l@{}}Implementar algoritmos basados en distribuciones\\ bayesianas\end{tabular} & U, SM, TE, CM, TF                                \\ \hline
	1                                          & \multirow{7}{*}{SC}                       & Registrar procesos en sistema de colas                                                                & TE, TF, SM, CM, S, R, P                          \\ \cline{1-1} \cline{3-4} 
	2                                          &                                           & Editar proceso en sistema de cola                                                                     & TE, TF, SM, CM, S, R, P                          \\ \cline{1-1} \cline{3-4} 
	3                                          &                                           & Eliminar proceso de sistema de cola                                                                   & TE, TF, SM, CM, S, R, P                          \\ \cline{1-1} \cline{3-4} 
	4                                          &                                           & Mostrar estado de cola                                                                                & TE, TF, SM, CM, S, R, P                          \\ \cline{1-1} \cline{3-4} 
	5                                          &                                           & Notificar estados de procesos                                                                         & TE, TF, SM, CM, S, R, P                          \\ \cline{1-1} \cline{3-4} 
	6                                          &                                           & Notificar finalizaci�n de trabajos                                                                    & TE, TF, SM, CM, S, R, P                          \\ \cline{1-1} \cline{3-4} 
	7                                          &                                           & Revisar procesos en cola                                                                              & TE, TF, SM, CM, S, R, P                          \\ \hline
	1                                          & \multirow{3}{*}{SN}                       & Enviar mensajes de notificaciones a usuarios                                                          & TF, CM, SM, S                                    \\ \cline{1-1} \cline{3-4} 
	2                                          &                                           & Configurar mensajes de notificaciones                                                                 & TF, CM, SM, S                                    \\ \cline{1-1} \cline{3-4} 
	3                                          &                                           & Generar notificaciones globales                                                                       & TF, CM, SM, S                                    \\ \hline
	1                                          & \multirow{4}{*}{AP}                       & Responder solicitudes de muestra de datos                                                             & P, R, SM, CM, TF, TE                             \\ \cline{1-1} \cline{3-4} 
	2                                          &                                           & Registrar nuevos elementos                                                                            & P, R, SM, CM, TF, TE                             \\ \cline{1-1} \cline{3-4} 
	3                                          &                                           & Modificar elementos en el sistema                                                                     & P, R, SM, CM, TF, TE                             \\ \cline{1-1} \cline{3-4} 
	4                                          &                                           & Eliminar elementos en el sistema                                                                      & P, R, SM, CM, TF, TE                             \\ \hline
	\caption{Tabla resumen de Atributos por Funci�n}
	\label{analisisTab14}\\
\end{longtable}

\section{Actores y Usuarios}

Los actores son aquellos entes que participan o interact�an con el software pero que no forman parte directa de �ste, es decir, son agentes externos. En este dise�o de software se han detectado los siguientes actores que se exponen en la Tabla \ref{analisisTab15}.

% Please add the following required packages to your document preamble:
% \usepackage{multirow}
% \usepackage{longtable}
% Note: It may be necessary to compile the document several times to get a multi-page table to line up properly
\begin{longtable}[c]{|l|l|l|}
	\hline
	\multicolumn{3}{|c|}{\multirow{2}{*}{\textbf{Actores del Sistema}}}                                                                                                                                                            \\
	\multicolumn{3}{|c|}{}                                                                                                                                                                                                         \\ \hline
	\endfirsthead
	%
	\endhead
	%
	\multicolumn{1}{|c|}{\textbf{Actor}} & \multicolumn{1}{c|}{\textbf{ID}} & \multicolumn{1}{c|}{\textbf{Descripci�n}}                                                                                                            \\ \hline
	Usuarios                             & U                                & \begin{tabular}[c]{@{}l@{}}Entidad externa que har�n\\ uso de los servicios y m�dulos\\ que disponga el sistema\end{tabular}                         \\ \hline
	Almacenamiento Persistente           & AP                               & \begin{tabular}[c]{@{}l@{}}Entidad que representa el\\ entorno de almacenamiento\\ persistente del sistema\end{tabular}                              \\ \hline
	Sistema de Colas                     & SC                               & \begin{tabular}[c]{@{}l@{}}Entidad externa que facilita\\ el manejo de los procesos a\\ ejecutarse de manera \\ remota en forma de jobs\end{tabular} \\ \hline
	Scikit-learn                         & SCL                              & \begin{tabular}[c]{@{}l@{}}Entidad externa que facilita\\ la implementaci�n de los \\ modelos de clasificaci�n y clustering\end{tabular}             \\ \hline
	\caption{Actores que interact�an con el sistema}
	\label{analisisTab15}\\
\end{longtable}

En la Tabla \ref{analisisTab15} se encuentran definidos los actores, adem�s se aprecia que existe el actor Usuario. Sin embargo, se destaca que los usuarios pueden tener roles o perfiles, los cuales se exponen en la Tabla \ref{analisisTab16}

% Please add the following required packages to your document preamble:
% \usepackage{graphicx}
\begin{table}[!h]
	\resizebox{\textwidth}{!}{%
		\begin{tabular}{|l|l|l|}
			\hline
			\multicolumn{3}{|c|}{\textbf{Perfiles del Sistema}}                                                                                                                                                                   \\ \hline
			\multicolumn{1}{|c|}{\textbf{Rol}} & \multicolumn{1}{c|}{\textbf{ID}} & \multicolumn{1}{c|}{\textbf{Descripci�n}}                                                                                                     \\ \hline
			Administrador                      & A                                & \begin{tabular}[c]{@{}l@{}}Encargado de administrar las cuentas\\ de usuario, los sistemas de acceso,\\ estad�sticas de uso, etc\end{tabular} \\ \hline
			Usuario Com�n                      & UC                               & \begin{tabular}[c]{@{}l@{}}Tiene acceso a los m�dulos y a la\\ utilizaci�n de recursos que el sistema\\ dispone\end{tabular}                  \\ \hline
			Root                               & R                                & Usuario con todos los privilegios de acceso                                                                                                   \\ \hline
		\end{tabular}%
	}
	\caption{Perfiles de usuario identificados}
	\label{analisisTab16}
\end{table}

\section{Casos de Uso}

Un caso de uso, representa la narrativa de un conjunto de acciones que suceden entre actores y el sistema, exponiendo los flujos de acciones o la l�gica tras una acci�n, se asocian a funciones del sistema y tienen la finalidad de cumplir con alguna de �stas, normalmente se encuentran asociados a m�s casos de uso, generando un conjunto de referencias cruzadas, adem�s se eval�a su funcionamiento mediante la determinaci�n de las pre y post condiciones. 

A continuaci�n se exponen todos los principales casos de uso para el sistema, los cuales se asocian la funci�n a la cual pertenecen, adem�s se agrupan seg�n actor o acciones que engloban.

\subsection{Casos de uso asociados a las acciones del Usuario administrador}

Estos casos de uso se asocian a las acciones que tienen estrecha relaci�n con las obligaciones que cumple un usuario con rol de administrador o root.

% Please add the following required packages to your document preamble:
% \usepackage{multirow}
% \usepackage{longtable}
% Note: It may be necessary to compile the document several times to get a multi-page table to line up properly
\begin{longtable}{ll}
	\hline
	\multicolumn{2}{|c|}{\textit{\textbf{Caso de Uso: CU01}}}                                                                                                                                                                                                                               \\ \hline
	\endfirsthead
	%
	\endhead
	%
	\multicolumn{1}{|l|}{\textbf{Nombre}}                                                                                              & \multicolumn{1}{l|}{Registrar nuevo usuario}                                                                                                       \\ \hline
	\multicolumn{1}{|l|}{\textbf{Actores}}                                                                                             & \multicolumn{1}{l|}{U, AP}                                                                                                                         \\ \hline
	\multicolumn{1}{|l|}{\textbf{Funciones Asociadas}}                                                                                 & \multicolumn{1}{l|}{US1, AP1, AP2}                                                                                                                 \\ \hline
	\multicolumn{1}{|l|}{\textbf{Objetivo}}                                                                                            & \multicolumn{1}{l|}{\begin{tabular}[c]{@{}l@{}}Insertar en el sistema de\\ almacenamiento persistente\\ un nuevo usuario\end{tabular}}             \\ \hline
	\multicolumn{1}{|l|}{\textbf{Pre Condiciones}}                                                                                     & \multicolumn{1}{l|}{\begin{tabular}[c]{@{}l@{}}Sistema de almacenamiento\\ persistente sin registro de usuario\end{tabular}}                       \\ \hline
	\multicolumn{1}{|l|}{\textbf{Post Condiciones}}                                                                                    & \multicolumn{1}{l|}{\begin{tabular}[c]{@{}l@{}}Usuario registrado en sistema,\\ �rea de trabajo creada correctamente\end{tabular}}                 \\ \hline
	\multicolumn{1}{|c|}{\textbf{Acciones del Actor}}                                                                                  & \multicolumn{1}{c|}{\textbf{Acciones del Sistema}}                                                                                                 \\ \hline
	\multicolumn{1}{|l|}{\begin{tabular}[c]{@{}l@{}}Administrador ingresa a\\ formulario de nuevo registro\\ de usuario\end{tabular}}  & \multicolumn{1}{l|}{}                                                                                                                              \\ \hline
	\multicolumn{1}{|l|}{}                                                                                                             & \multicolumn{1}{l|}{\begin{tabular}[c]{@{}l@{}}Sistema solicita los datos\\ de usuario\end{tabular}}                                               \\ \hline
	\multicolumn{1}{|l|}{\begin{tabular}[c]{@{}l@{}}Administrador completa\\ y env�a formulario\end{tabular}}                          & \multicolumn{1}{l|}{}                                                                                                                              \\ \hline
	\multicolumn{1}{|l|}{}                                                                                                             & \multicolumn{1}{l|}{\begin{tabular}[c]{@{}l@{}}Sistema procesa y\\ valida los datos\end{tabular}}                                                  \\ \hline
	\multicolumn{1}{|l|}{}                                                                                                             & \multicolumn{1}{l|}{\begin{tabular}[c]{@{}l@{}}Sistema registra nuevo\\ dato en almacenamiento\\ persistente (CU02)\end{tabular}}                  \\ \hline
	\multicolumn{1}{|l|}{}                                                                                                             & \multicolumn{1}{l|}{\begin{tabular}[c]{@{}l@{}}Sistema crea �rea de trabajo\\ para nuevo usuario (CU03)\end{tabular}}                              \\ \hline
	\multicolumn{1}{|l|}{}                                                                                                             & \multicolumn{1}{l|}{\begin{tabular}[c]{@{}l@{}}Sistema notifica el correcto\\ t�rmino del proceso\end{tabular}}                                    \\ \hline
	\multicolumn{1}{|l|}{\textbf{Referencias Cruzadas}}                                                                                & \multicolumn{1}{l|}{CU02, CU03}                                                                                                                    \\ \hline
	\multicolumn{1}{|l|}{\multirow{2}{*}{\textbf{Cursos alternativos}}}                                                                & \multicolumn{1}{l|}{\begin{tabular}[c]{@{}l@{}}Usuario ya existe y el\\ sistema lo notifica\end{tabular}}                                          \\ \cline{2-2} 
	\multicolumn{1}{|l|}{}                                                                                                             & \multicolumn{1}{l|}{\begin{tabular}[c]{@{}l@{}}No se puede registrar el\\ usuario por problemas en AP,\\ se notifica\end{tabular}}                 \\ \hline
	&                                                                                                                                                    \\ \hline
	\multicolumn{2}{|c|}{\textit{\textbf{Caso de Uso: CU06}}}                                                                                                                                                                                                                               \\ \hline
	\multicolumn{1}{|l|}{\textbf{Nombre}}                                                                                              & \multicolumn{1}{l|}{Editar usuario existente}                                                                                                      \\ \hline
	\multicolumn{1}{|l|}{\textbf{Actores}}                                                                                             & \multicolumn{1}{l|}{U, AP}                                                                                                                         \\ \hline
	\multicolumn{1}{|l|}{\textbf{Funciones Asociadas}}                                                                                 & \multicolumn{1}{l|}{}                                                                                                                              \\ \hline
	\multicolumn{1}{|l|}{\textbf{Objetivo}}                                                                                            & \multicolumn{1}{l|}{\begin{tabular}[c]{@{}l@{}}Editar usuario existente en el\\ sistema de almacenamiento\\ persistente\end{tabular}}              \\ \hline
	\multicolumn{1}{|l|}{\textbf{Pre Condiciones}}                                                                                     & \multicolumn{1}{l|}{\begin{tabular}[c]{@{}l@{}}Sistema de almacenamiento\\ persistente sin usuario modificado\end{tabular}}                        \\ \hline
	\multicolumn{1}{|l|}{\textbf{Post Condiciones}}                                                                                    & \multicolumn{1}{l|}{\begin{tabular}[c]{@{}l@{}}Usuario registrado modificado de\\ manera correcta\end{tabular}}                                    \\ \hline
	\multicolumn{1}{|c|}{\textbf{Acciones del Actor}}                                                                                  & \multicolumn{1}{c|}{\textbf{Acciones del Sistema}}                                                                                                 \\ \hline
	\multicolumn{1}{|l|}{\begin{tabular}[c]{@{}l@{}}Administrador ingresa a\\ formulario de editar registro\\ de usuario\end{tabular}} & \multicolumn{1}{l|}{}                                                                                                                              \\ \hline
	\multicolumn{1}{|l|}{}                                                                                                             & \multicolumn{1}{l|}{\begin{tabular}[c]{@{}l@{}}Sistema solicita los datos\\ de usuario\end{tabular}}                                               \\ \hline
	\multicolumn{1}{|l|}{\begin{tabular}[c]{@{}l@{}}Administrador completa y\\ env�a formulario\end{tabular}}                          & \multicolumn{1}{l|}{}                                                                                                                              \\ \hline
	\multicolumn{1}{|l|}{}                                                                                                             & \multicolumn{1}{l|}{\begin{tabular}[c]{@{}l@{}}Sistema procesa y valida\\ los datos\end{tabular}}                                                  \\ \hline
	\multicolumn{1}{|l|}{}                                                                                                             & \multicolumn{1}{l|}{\begin{tabular}[c]{@{}l@{}}Sistema edita registro en\\ almacenamiento persistente (CU07)\end{tabular}}                         \\ \hline
	\multicolumn{1}{|l|}{}                                                                                                             & \multicolumn{1}{l|}{\begin{tabular}[c]{@{}l@{}}Sistema notifica el correcto\\ t�rmino del proceso\end{tabular}}                                    \\ \hline
	\multicolumn{1}{|l|}{\textbf{Referencias Cruzadas}}                                                                                & \multicolumn{1}{l|}{CU07}                                                                                                                          \\ \hline
	\multicolumn{1}{|l|}{\textbf{Cursos alternativos}}                                                                                 & \multicolumn{1}{l|}{\begin{tabular}[c]{@{}l@{}}No se puede editar el usuario por\\ problemas en AP, se notifica\end{tabular}}                      \\ \hline
	&                                                                                                                                                    \\ \hline
	\multicolumn{2}{|c|}{\textit{\textbf{Caso de Uso: CU08}}}                                                                                                                                                                                                                               \\ \hline
	\multicolumn{1}{|l|}{\textbf{Nombre}}                                                                                              & \multicolumn{1}{l|}{Eliminar usuario existente}                                                                                                    \\ \hline
	\multicolumn{1}{|l|}{\textbf{Actores}}                                                                                             & \multicolumn{1}{l|}{U, AP}                                                                                                                         \\ \hline
	\multicolumn{1}{|l|}{\textbf{Funciones Asociadas}}                                                                                 & \multicolumn{1}{l|}{}                                                                                                                              \\ \hline
	\multicolumn{1}{|l|}{\textbf{Objetivo}}                                                                                            & \multicolumn{1}{l|}{\begin{tabular}[c]{@{}l@{}}Eliminar usuario existente\\ en el sistema de almacenamiento\\ persistente\end{tabular}}            \\ \hline
	\multicolumn{1}{|l|}{\textbf{Pre Condiciones}}                                                                                     & \multicolumn{1}{l|}{\begin{tabular}[c]{@{}l@{}}Sistema de almacenamiento\\ persistente con usuario registrado\end{tabular}}                        \\ \hline
	\multicolumn{1}{|l|}{\textbf{Post Condiciones}}                                                                                    & \multicolumn{1}{l|}{\begin{tabular}[c]{@{}l@{}}Usuario registrado eliminado de\\ manera correcta\end{tabular}}                                     \\ \hline
	\multicolumn{1}{|c|}{\textbf{Acciones del Actor}}                                                                                  & \multicolumn{1}{c|}{\textbf{Acciones del Sistema}}                                                                                                 \\ \hline
	\multicolumn{1}{|l|}{\begin{tabular}[c]{@{}l@{}}Administrador ingresa a formulario\\ de eliminar registro de usuario\end{tabular}} & \multicolumn{1}{l|}{}                                                                                                                              \\ \hline
	\multicolumn{1}{|l|}{}                                                                                                             & \multicolumn{1}{l|}{\begin{tabular}[c]{@{}l@{}}Sistema solicita los datos\\ de usuario\end{tabular}}                                               \\ \hline
	\multicolumn{1}{|l|}{\begin{tabular}[c]{@{}l@{}}Administrador completa y\\ env�a formulario\end{tabular}}                          & \multicolumn{1}{l|}{}                                                                                                                              \\ \hline
	\multicolumn{1}{|l|}{}                                                                                                             & \multicolumn{1}{l|}{\begin{tabular}[c]{@{}l@{}}Sistema procesa y valida\\ los datos\end{tabular}}                                                  \\ \hline
	\multicolumn{1}{|l|}{}                                                                                                             & \multicolumn{1}{l|}{\begin{tabular}[c]{@{}l@{}}Sistema elimina registro en\\ almacenamiento persistente (CU09)\end{tabular}}                       \\ \hline
	\multicolumn{1}{|l|}{}                                                                                                             & \multicolumn{1}{l|}{\begin{tabular}[c]{@{}l@{}}Sistema elimina �rea de trabajo\\ del usuario (CU10)\end{tabular}}                                  \\ \hline
	\multicolumn{1}{|l|}{}                                                                                                             & \multicolumn{1}{l|}{\begin{tabular}[c]{@{}l@{}}Sistema notifica el correcto\\ t�rmino del proceso\end{tabular}}                                    \\ \hline
	\multicolumn{1}{|l|}{\textbf{Referencias Cruzadas}}                                                                                & \multicolumn{1}{l|}{CU09, CU10}                                                                                                                    \\ \hline
	\multicolumn{1}{|l|}{\textbf{Cursos alternativos}}                                                                                 & \multicolumn{1}{l|}{\begin{tabular}[c]{@{}l@{}}No se puede eliminar el usuario\\ por problemas en AP, se notifica\end{tabular}}                    \\ \hline
	&                                                                                                                                                    \\ \hline
	\multicolumn{2}{|c|}{\textit{\textbf{Caso de Uso: CU11}}}                                                                                                                                                                                                                               \\ \hline
	\multicolumn{1}{|l|}{\textbf{Nombre}}                                                                                              & \multicolumn{1}{l|}{Visualizar usuarios}                                                                                                           \\ \hline
	\multicolumn{1}{|l|}{\textbf{Actores}}                                                                                             & \multicolumn{1}{l|}{U, AP}                                                                                                                         \\ \hline
	\multicolumn{1}{|l|}{\textbf{Funciones Asociadas}}                                                                                 & \multicolumn{1}{l|}{}                                                                                                                              \\ \hline
	\multicolumn{1}{|l|}{\textbf{Objetivo}}                                                                                            & \multicolumn{1}{l|}{\begin{tabular}[c]{@{}l@{}}Visualizar lista de usuarios o\\ datos de usuario seleccionado\end{tabular}}                        \\ \hline
	\multicolumn{1}{|l|}{\textbf{Pre Condiciones}}                                                                                     & \multicolumn{1}{l|}{\begin{tabular}[c]{@{}l@{}}Solicitud de visualizaci�n de\\ usuarios\end{tabular}}                                              \\ \hline
	\multicolumn{1}{|l|}{\textbf{Post Condiciones}}                                                                                    & \multicolumn{1}{l|}{\begin{tabular}[c]{@{}l@{}}Solicitud respondida de visualizaci�n\\ de usuarios\end{tabular}}                                   \\ \hline
	\multicolumn{1}{|c|}{\textbf{Acciones del Actor}}                                                                                  & \multicolumn{1}{c|}{\textbf{Acciones del Sistema}}                                                                                                 \\ \hline
	\multicolumn{1}{|l|}{\begin{tabular}[c]{@{}l@{}}Administrador ingresa a secci�n\\ visualizar usuarios\end{tabular}}                & \multicolumn{1}{l|}{}                                                                                                                              \\ \hline
	\multicolumn{1}{|l|}{}                                                                                                             & \multicolumn{1}{l|}{Sistema recibe la solicitud}                                                                                                   \\ \hline
	\multicolumn{1}{|l|}{}                                                                                                             & \multicolumn{1}{l|}{\begin{tabular}[c]{@{}l@{}}Sistema consulta la data al\\ Sistema de Almacenamiento\\ persistente\end{tabular}}                 \\ \hline
	\multicolumn{1}{|l|}{}                                                                                                             & \multicolumn{1}{l|}{\begin{tabular}[c]{@{}l@{}}Sistema recupera la respuesta\\ en formato JSON\end{tabular}}                                       \\ \hline
	\multicolumn{1}{|l|}{}                                                                                                             & \multicolumn{1}{l|}{Sistema decodifica la informaci�n}                                                                                             \\ \hline
	\multicolumn{1}{|l|}{}                                                                                                             & \multicolumn{1}{l|}{\begin{tabular}[c]{@{}l@{}}Sistema despliega la respuesta en\\ la interfaz de usuario\end{tabular}}                            \\ \hline
	\multicolumn{1}{|l|}{\textbf{Referencias Cruzadas}}                                                                                & \multicolumn{1}{l|}{CU12}                                                                                                                          \\ \hline
	\multicolumn{1}{|l|}{\textbf{Cursos alternativos}}                                                                                 & \multicolumn{1}{l|}{\begin{tabular}[c]{@{}l@{}}No se puede mostrar la data,\\ se notifica la acci�n\end{tabular}}                                  \\ \hline
	&                                                                                                                                                    \\ \hline
	\multicolumn{2}{|c|}{\textit{\textbf{Caso de Uso: CU13}}}                                                                                                                                                                                                                               \\ \hline
	\multicolumn{1}{|l|}{\textbf{Nombre}}                                                                                              & \multicolumn{1}{l|}{Registrar nuevo rol}                                                                                                           \\ \hline
	\multicolumn{1}{|l|}{\textbf{Actores}}                                                                                             & \multicolumn{1}{l|}{U, AP}                                                                                                                         \\ \hline
	\multicolumn{1}{|l|}{\textbf{Funciones Asociadas}}                                                                                 & \multicolumn{1}{l|}{}                                                                                                                              \\ \hline
	\multicolumn{1}{|l|}{\textbf{Objetivo}}                                                                                            & \multicolumn{1}{l|}{\begin{tabular}[c]{@{}l@{}}Insertar en el sistema de\\ almacenamiento persistente\\ un nuevo rol\end{tabular}}                 \\ \hline
	\multicolumn{1}{|l|}{\textbf{Pre Condiciones}}                                                                                     & \multicolumn{1}{l|}{\begin{tabular}[c]{@{}l@{}}Sistema de almacenamiento\\ persistente sin registro de rol\end{tabular}}                           \\ \hline
	\multicolumn{1}{|l|}{\textbf{Post Condiciones}}                                                                                    & \multicolumn{1}{l|}{Rol registrado en sistema}                                                                                                     \\ \hline
	\multicolumn{1}{|c|}{\textbf{Acciones del Actor}}                                                                                  & \multicolumn{1}{c|}{\textbf{Acciones del Sistema}}                                                                                                 \\ \hline
	\multicolumn{1}{|l|}{\begin{tabular}[c]{@{}l@{}}Administrador ingresa a\\ formulario de nuevo\\ registro de rol\end{tabular}}      & \multicolumn{1}{l|}{}                                                                                                                              \\ \hline
	\multicolumn{1}{|l|}{}                                                                                                             & \multicolumn{1}{l|}{\begin{tabular}[c]{@{}l@{}}Sistema solicita los\\ datos de rol\end{tabular}}                                                   \\ \hline
	\multicolumn{1}{|l|}{\begin{tabular}[c]{@{}l@{}}Administrador completa\\ y env�a formulario\end{tabular}}                          & \multicolumn{1}{l|}{}                                                                                                                              \\ \hline
	\multicolumn{1}{|l|}{}                                                                                                             & \multicolumn{1}{l|}{\begin{tabular}[c]{@{}l@{}}Sistema procesa y valida\\ los datos\end{tabular}}                                                  \\ \hline
	\multicolumn{1}{|l|}{}                                                                                                             & \multicolumn{1}{l|}{\begin{tabular}[c]{@{}l@{}}Sistema registra nuevo dato en\\ almacenamiento persistente (CU02)\end{tabular}}                    \\ \hline
	\multicolumn{1}{|l|}{}                                                                                                             & \multicolumn{1}{l|}{\begin{tabular}[c]{@{}l@{}}Sistema notifica el correcto\\ t�rmino del proceso\end{tabular}}                                    \\ \hline
	\multicolumn{1}{|l|}{\textbf{Referencias Cruzadas}}                                                                                & \multicolumn{1}{l|}{CU02}                                                                                                                          \\ \hline
	\multicolumn{1}{|l|}{\multirow{2}{*}{\textbf{Cursos alternativos}}}                                                                & \multicolumn{1}{l|}{Rol ya existe y el sistema lo notifica}                                                                                        \\ \cline{2-2} 
	\multicolumn{1}{|l|}{}                                                                                                             & \multicolumn{1}{l|}{\begin{tabular}[c]{@{}l@{}}No se puede registrar el rol por\\ problemas en AP, se notifica\end{tabular}}                       \\ \hline
	&                                                                                                                                                    \\ \hline
	\multicolumn{2}{|c|}{\textit{\textbf{Caso de Uso: CU14}}}                                                                                                                                                                                                                               \\ \hline
	\multicolumn{1}{|l|}{\textbf{Nombre}}                                                                                              & \multicolumn{1}{l|}{Editar rol existente}                                                                                                          \\ \hline
	\multicolumn{1}{|l|}{\textbf{Actores}}                                                                                             & \multicolumn{1}{l|}{U, AP}                                                                                                                         \\ \hline
	\multicolumn{1}{|l|}{\textbf{Funciones Asociadas}}                                                                                 & \multicolumn{1}{l|}{}                                                                                                                              \\ \hline
	\multicolumn{1}{|l|}{\textbf{Objetivo}}                                                                                            & \multicolumn{1}{l|}{\begin{tabular}[c]{@{}l@{}}Editar rol existente en el sistema\\ de almacenamiento persistente\end{tabular}}                    \\ \hline
	\multicolumn{1}{|l|}{\textbf{Pre Condiciones}}                                                                                     & \multicolumn{1}{l|}{\begin{tabular}[c]{@{}l@{}}Sistema de almacenamiento\\ persistente sin rol modificado\end{tabular}}                            \\ \hline
	\multicolumn{1}{|l|}{\textbf{Post Condiciones}}                                                                                    & \multicolumn{1}{l|}{\begin{tabular}[c]{@{}l@{}}Rol editado modificado de\\ manera correcta\end{tabular}}                                           \\ \hline
	\multicolumn{1}{|c|}{\textbf{Acciones del Actor}}                                                                                  & \multicolumn{1}{c|}{\textbf{Acciones del Sistema}}                                                                                                 \\ \hline
	\multicolumn{1}{|l|}{\begin{tabular}[c]{@{}l@{}}Administrador ingresa a\\ formulario de editar\\ registro de rol\end{tabular}}     & \multicolumn{1}{l|}{}                                                                                                                              \\ \hline
	\multicolumn{1}{|l|}{}                                                                                                             & \multicolumn{1}{l|}{\begin{tabular}[c]{@{}l@{}}Sistema solicita los\\ datos de rol\end{tabular}}                                                   \\ \hline
	\multicolumn{1}{|l|}{\begin{tabular}[c]{@{}l@{}}Administrador completa\\ y env�a formulario\end{tabular}}                          & \multicolumn{1}{l|}{}                                                                                                                              \\ \hline
	\multicolumn{1}{|l|}{}                                                                                                             & \multicolumn{1}{l|}{\begin{tabular}[c]{@{}l@{}}Sistema procesa y valida\\ los datos\end{tabular}}                                                  \\ \hline
	\multicolumn{1}{|l|}{}                                                                                                             & \multicolumn{1}{l|}{\begin{tabular}[c]{@{}l@{}}Sistema edita registro en\\ almacenamiento persistente (CU07)\end{tabular}}                         \\ \hline
	\multicolumn{1}{|l|}{}                                                                                                             & \multicolumn{1}{l|}{\begin{tabular}[c]{@{}l@{}}Sistema notifica el correcto\\ t�rmino del proceso\end{tabular}}                                    \\ \hline
	\multicolumn{1}{|l|}{\textbf{Referencias Cruzadas}}                                                                                & \multicolumn{1}{l|}{CU07}                                                                                                                          \\ \hline
	\multicolumn{1}{|l|}{\textbf{Cursos alternativos}}                                                                                 & \multicolumn{1}{l|}{\begin{tabular}[c]{@{}l@{}}No se puede editar el rol por\\ problemas en AP, se notifica\end{tabular}}                          \\ \hline
	&                                                                                                                                                    \\ \hline
	\multicolumn{2}{|c|}{\textit{\textbf{Caso de Uso: CU15}}}                                                                                                                                                                                                                               \\ \hline
	\multicolumn{1}{|l|}{\textbf{Nombre}}                                                                                              & \multicolumn{1}{l|}{Eliminar rol existente}                                                                                                        \\ \hline
	\multicolumn{1}{|l|}{\textbf{Actores}}                                                                                             & \multicolumn{1}{l|}{U, AP}                                                                                                                         \\ \hline
	\multicolumn{1}{|l|}{\textbf{Funciones Asociadas}}                                                                                 & \multicolumn{1}{l|}{}                                                                                                                              \\ \hline
	\multicolumn{1}{|l|}{\textbf{Objetivo}}                                                                                            & \multicolumn{1}{l|}{\begin{tabular}[c]{@{}l@{}}Eliminar rol existente en el\\ sistema de almacenamiento\\ persistente\end{tabular}}                \\ \hline
	\multicolumn{1}{|l|}{\textbf{Pre Condiciones}}                                                                                     & \multicolumn{1}{l|}{\begin{tabular}[c]{@{}l@{}}Sistema de almacenamiento\\ persistente con rol registrado\end{tabular}}                            \\ \hline
	\multicolumn{1}{|l|}{\textbf{Post Condiciones}}                                                                                    & \multicolumn{1}{l|}{\begin{tabular}[c]{@{}l@{}}Rol registrado eliminado de\\ manera correcta\end{tabular}}                                         \\ \hline
	\multicolumn{1}{|c|}{\textit{\textbf{Acciones del Actor}}}                                                                         & \multicolumn{1}{c|}{\textit{\textbf{Acciones del Sistema}}}                                                                                        \\ \hline
	\multicolumn{1}{|l|}{\begin{tabular}[c]{@{}l@{}}Administrador ingresa a formulario\\ de eliminar registro de rol\end{tabular}}     & \multicolumn{1}{l|}{}                                                                                                                              \\ \hline
	\multicolumn{1}{|l|}{}                                                                                                             & \multicolumn{1}{l|}{\begin{tabular}[c]{@{}l@{}}Sistema solicita los\\ datos de rol\end{tabular}}                                                   \\ \hline
	\multicolumn{1}{|l|}{\begin{tabular}[c]{@{}l@{}}Administrador completa y\\ env�a formulario\end{tabular}}                          & \multicolumn{1}{l|}{}                                                                                                                              \\ \hline
	\multicolumn{1}{|l|}{}                                                                                                             & \multicolumn{1}{l|}{\begin{tabular}[c]{@{}l@{}}Sistema procesa y\\ valida los datos\end{tabular}}                                                  \\ \hline
	\multicolumn{1}{|l|}{}                                                                                                             & \multicolumn{1}{l|}{\begin{tabular}[c]{@{}l@{}}Sistema elimina registro\\ en almacenamiento persistente (CU09)\end{tabular}}                       \\ \hline
	\multicolumn{1}{|l|}{}                                                                                                             & \multicolumn{1}{l|}{\begin{tabular}[c]{@{}l@{}}Sistema notifica el correcto\\ t�rmino del proceso\end{tabular}}                                    \\ \hline
	\multicolumn{1}{|l|}{\textbf{Referencias Cruzadas}}                                                                                & \multicolumn{1}{l|}{CU09}                                                                                                                          \\ \hline
	\multicolumn{1}{|l|}{\textbf{Cursos alternativos}}                                                                                 & \multicolumn{1}{l|}{\begin{tabular}[c]{@{}l@{}}No se puede eliminar el rol\\ por problemas en AP, se notifica\end{tabular}}                        \\ \hline
	&                                                                                                                                                    \\ \hline
	\multicolumn{2}{|c|}{\textit{\textbf{Caso de Uso: CU16}}}                                                                                                                                                                                                                               \\ \hline
	\multicolumn{1}{|l|}{\textbf{Nombre}}                                                                                              & \multicolumn{1}{l|}{Visualizar roles}                                                                                                              \\ \hline
	\multicolumn{1}{|l|}{\textbf{Actores}}                                                                                             & \multicolumn{1}{l|}{U, AP}                                                                                                                         \\ \hline
	\multicolumn{1}{|l|}{\textbf{Funciones Asociadas}}                                                                                 & \multicolumn{1}{l|}{}                                                                                                                              \\ \hline
	\multicolumn{1}{|l|}{\textbf{Objetivo}}                                                                                            & \multicolumn{1}{l|}{\begin{tabular}[c]{@{}l@{}}Visualizar lista de roles registrados\\ en el sistema de almacenamiento\\ persistente\end{tabular}} \\ \hline
	\multicolumn{1}{|l|}{\textbf{Pre Condiciones}}                                                                                     & \multicolumn{1}{l|}{\begin{tabular}[c]{@{}l@{}}Solicitud de visualizaci�n\\ de roles\end{tabular}}                                                 \\ \hline
	\multicolumn{1}{|l|}{\textbf{Post Condiciones}}                                                                                    & \multicolumn{1}{l|}{\begin{tabular}[c]{@{}l@{}}Solicitud respondida de visualizaci�n\\ de roles\end{tabular}}                                      \\ \hline
	\multicolumn{1}{|c|}{\textit{\textbf{Acciones del Actor}}}                                                                         & \multicolumn{1}{c|}{\textit{\textbf{Acciones del Sistema}}}                                                                                        \\ \hline
	\multicolumn{1}{|l|}{\begin{tabular}[c]{@{}l@{}}Administrador ingresa a\\ secci�n visualizar roles\end{tabular}}                   & \multicolumn{1}{l|}{}                                                                                                                              \\ \hline
	\multicolumn{1}{|l|}{}                                                                                                             & \multicolumn{1}{l|}{\begin{tabular}[c]{@{}l@{}}Sistema recibe la\\ solicitud\end{tabular}}                                                         \\ \hline
	\multicolumn{1}{|l|}{}                                                                                                             & \multicolumn{1}{l|}{\begin{tabular}[c]{@{}l@{}}Sistema consulta la data al\\ Sistema de Almacenamiento\\ persistente\end{tabular}}                 \\ \hline
	\multicolumn{1}{|l|}{}                                                                                                             & \multicolumn{1}{l|}{\begin{tabular}[c]{@{}l@{}}Sistema recupera la respuesta\\ en formato JSON\end{tabular}}                                       \\ \hline
	\multicolumn{1}{|l|}{}                                                                                                             & \multicolumn{1}{l|}{Sistema decodifica la informaci�n}                                                                                             \\ \hline
	\multicolumn{1}{|l|}{}                                                                                                             & \multicolumn{1}{l|}{\begin{tabular}[c]{@{}l@{}}Sistema despliega la respuesta\\ en la interfaz de usuario\end{tabular}}                            \\ \hline
	\multicolumn{1}{|l|}{\textbf{Referencias Cruzadas}}                                                                                & \multicolumn{1}{l|}{CU12}                                                                                                                          \\ \hline
	\multicolumn{1}{|l|}{\textbf{Cursos alternativos}}                                                                                 & \multicolumn{1}{l|}{\begin{tabular}[c]{@{}l@{}}No se puede mostrar la data, se\\ notifica la acci�n\end{tabular}}                                  \\ \hline
	\caption{Casos de uso asociados a los usuarios administrativo y root}
	\label{cu01}\\
	\end{longtable}
	
	\subsection{Casos de uso asociados al uso de recursos}
	
	Estos casos de uso se asocian al uso de recursos en el servidor, espacios f�sicos de almacenamiento, �reas de trabajo para los usuarios, etc.
	
	% Please add the following required packages to your document preamble:
	% \usepackage{longtable}
	% Note: It may be necessary to compile the document several times to get a multi-page table to line up properly
	\begin{longtable}{ll}
		\hline
		\multicolumn{2}{|c|}{\textit{\textbf{Caso de Uso: CU03}}}                                                                                                                                                                                               \\ \hline
		\endfirsthead
		%
		\endhead
		%
		\multicolumn{1}{|l|}{\textbf{Nombre}}                                                                                 & \multicolumn{1}{l|}{\begin{tabular}[c]{@{}l@{}}Crear �rea de trabajo de\\ usuario registrado\end{tabular}}                      \\ \hline
		\multicolumn{1}{|l|}{\textbf{Actores}}                                                                                & \multicolumn{1}{l|}{U}                                                                                                          \\ \hline
		\multicolumn{1}{|l|}{\textbf{Funciones Asociadas}}                                                                    & \multicolumn{1}{l|}{}                                                                                                           \\ \hline
		\multicolumn{1}{|l|}{\textbf{Objetivo}}                                                                               & \multicolumn{1}{l|}{\begin{tabular}[c]{@{}l@{}}Crear �rea de trabajo a usuario\\ registrado\end{tabular}}                       \\ \hline
		\multicolumn{1}{|l|}{\textbf{Pre Condiciones}}                                                                        & \multicolumn{1}{l|}{\begin{tabular}[c]{@{}l@{}}No existe �rea de trabajo ni\\ habilitaci�n de recursos\end{tabular}}            \\ \hline
		\multicolumn{1}{|l|}{\textbf{Post Condiciones}}                                                                       & \multicolumn{1}{l|}{\begin{tabular}[c]{@{}l@{}}Se habilita recursos para usuario\\ y se crea �rea de trabajo\end{tabular}}      \\ \hline
		\multicolumn{1}{|c|}{\textit{\textbf{Acciones del Actor}}}                                                            & \multicolumn{1}{c|}{\textit{\textbf{Acciones del Sistema}}}                                                                     \\ \hline
		\multicolumn{1}{|l|}{\begin{tabular}[c]{@{}l@{}}Usuario solicita la creaci�n\\ de nueva �rea de trabajo\end{tabular}} & \multicolumn{1}{l|}{}                                                                                                           \\ \hline
		\multicolumn{1}{|l|}{}                                                                                                & \multicolumn{1}{l|}{\begin{tabular}[c]{@{}l@{}}Sistema habilita el usuario\\ y crea espacio de trabajo\end{tabular}}            \\ \hline
		\multicolumn{1}{|l|}{}                                                                                                & \multicolumn{1}{l|}{\begin{tabular}[c]{@{}l@{}}Sistema habilita los recursos\\ para el nuevo usuario\end{tabular}}              \\ \hline
		\multicolumn{1}{|l|}{}                                                                                                & \multicolumn{1}{l|}{\begin{tabular}[c]{@{}l@{}}Sistema habilita al usuario en\\ el proceso de colas\end{tabular}}               \\ \hline
		\multicolumn{1}{|l|}{}                                                                                                & \multicolumn{1}{l|}{\begin{tabular}[c]{@{}l@{}}Sistema habilita al usuario en\\ el registro de correos\end{tabular}}            \\ \hline
		\multicolumn{1}{|l|}{}                                                                                                & \multicolumn{1}{l|}{\begin{tabular}[c]{@{}l@{}}Sistema notifica el correcto\\ t�rmino del proceso\end{tabular}}                 \\ \hline
		\multicolumn{1}{|l|}{\textbf{Referencias Cruzadas}}                                                                   & \multicolumn{1}{l|}{--}                                                                                                         \\ \hline
		\multicolumn{1}{|l|}{\textbf{Cursos alternativos}}                                                                    & \multicolumn{1}{l|}{--}                                                                                                         \\ \hline
		&                                                                                                                                 \\ \hline
		\multicolumn{2}{|c|}{\textit{\textbf{Caso de Uso: CU10}}}                                                                                                                                                                                               \\ \hline
		\multicolumn{1}{|l|}{\textbf{Nombre}}                                                                                 & \multicolumn{1}{l|}{\begin{tabular}[c]{@{}l@{}}Eliminar �rea de trabajo de\\ usuario registrado\end{tabular}}                   \\ \hline
		\multicolumn{1}{|l|}{\textbf{Actores}}                                                                                & \multicolumn{1}{l|}{U}                                                                                                          \\ \hline
		\multicolumn{1}{|l|}{\textbf{Funciones Asociadas}}                                                                    & \multicolumn{1}{l|}{}                                                                                                           \\ \hline
		\multicolumn{1}{|l|}{\textbf{Objetivo}}                                                                               & \multicolumn{1}{l|}{\begin{tabular}[c]{@{}l@{}}Eliminar �rea de trabajo a\\ usuario registrado\end{tabular}}                    \\ \hline
		\multicolumn{1}{|l|}{\textbf{Pre Condiciones}}                                                                        & \multicolumn{1}{l|}{\begin{tabular}[c]{@{}l@{}}Si existe �rea de trabajo ni\\ habilitaci�n de recursos\end{tabular}}            \\ \hline
		\multicolumn{1}{|l|}{\textbf{Post Condiciones}}                                                                       & \multicolumn{1}{l|}{\begin{tabular}[c]{@{}l@{}}Se elimina �rea de trabajo y\\ deshabilitan los recursos asociados\end{tabular}} \\ \hline
		\multicolumn{1}{|c|}{\textit{\textbf{Acciones del Actor}}}                                                            & \multicolumn{1}{c|}{\textit{\textbf{Acciones del Sistema}}}                                                                     \\ \hline
		\multicolumn{1}{|l|}{\begin{tabular}[c]{@{}l@{}}Usuario solicita la eliminaci�n\\ de �rea de trabajo\end{tabular}}    & \multicolumn{1}{l|}{}                                                                                                           \\ \hline
		\multicolumn{1}{|l|}{}                                                                                                & \multicolumn{1}{l|}{\begin{tabular}[c]{@{}l@{}}Sistema deshabilita el usuario\\ y elimina espacio de trabajo\end{tabular}}      \\ \hline
		\multicolumn{1}{|l|}{}                                                                                                & \multicolumn{1}{l|}{\begin{tabular}[c]{@{}l@{}}Sistema elimina los recursos\\ para el usuario\end{tabular}}                     \\ \hline
		\multicolumn{1}{|l|}{}                                                                                                & \multicolumn{1}{l|}{\begin{tabular}[c]{@{}l@{}}Sistema deshabilita al usuario\\ en el proceso de colas\end{tabular}}            \\ \hline
		\multicolumn{1}{|l|}{}                                                                                                & \multicolumn{1}{l|}{\begin{tabular}[c]{@{}l@{}}Sistema deshabilita al usuario\\ en el registro de correos\end{tabular}}         \\ \hline
		\multicolumn{1}{|l|}{}                                                                                                & \multicolumn{1}{l|}{\begin{tabular}[c]{@{}l@{}}Sistema notifica el correcto\\ t�rmino del proceso\end{tabular}}                 \\ \hline
		\multicolumn{1}{|l|}{\textbf{Referencias Cruzadas}}                                                                   & \multicolumn{1}{l|}{--}                                                                                                         \\ \hline
		\multicolumn{1}{|l|}{\textbf{Cursos alternativos}}                                                                    & \multicolumn{1}{l|}{--}                                                                                                         \\ \hline
		&                                                                                                                                 \\ \hline
		\multicolumn{2}{|c|}{\textit{\textbf{Caso de Uso: CU40}}}                                                                                                                                                                                               \\ \hline
		\multicolumn{1}{|l|}{\textbf{Nombre}}                                                                                 & \multicolumn{1}{l|}{\begin{tabular}[c]{@{}l@{}}Eliminar archivo en �rea\\ de trabajo\end{tabular}}                              \\ \hline
		\multicolumn{1}{|l|}{\textbf{Actores}}                                                                                & \multicolumn{1}{l|}{U}                                                                                                          \\ \hline
		\multicolumn{1}{|l|}{\textbf{Funciones Asociadas}}                                                                    & \multicolumn{1}{l|}{}                                                                                                           \\ \hline
		\multicolumn{1}{|l|}{\textbf{Objetivo}}                                                                               & \multicolumn{1}{l|}{\begin{tabular}[c]{@{}l@{}}Eliminar archivo en �rea\\ de trabajo\end{tabular}}                              \\ \hline
		\multicolumn{1}{|l|}{\textbf{Pre Condiciones}}                                                                        & \multicolumn{1}{l|}{\begin{tabular}[c]{@{}l@{}}Archivo existe en el �rea\\ de trabajo\end{tabular}}                             \\ \hline
		\multicolumn{1}{|l|}{\textbf{Post Condiciones}}                                                                       & \multicolumn{1}{l|}{\begin{tabular}[c]{@{}l@{}}Archivo eliminado del �rea\\ de trabajo\end{tabular}}                            \\ \hline
		\multicolumn{1}{|c|}{\textit{\textbf{Acciones del Actor}}}                                                            & \multicolumn{1}{c|}{\textit{\textbf{Acciones del Sistema}}}                                                                     \\ \hline
		\multicolumn{1}{|l|}{\begin{tabular}[c]{@{}l@{}}Usuario sube un archivo\\ al �rea de trabajo\end{tabular}}            & \multicolumn{1}{l|}{}                                                                                                           \\ \hline
		\multicolumn{1}{|l|}{}                                                                                                & \multicolumn{1}{l|}{\begin{tabular}[c]{@{}l@{}}Sistema detecta fallas en\\ el archivo\end{tabular}}                             \\ \hline
		\multicolumn{1}{|l|}{}                                                                                                & \multicolumn{1}{l|}{\begin{tabular}[c]{@{}l@{}}Sistema elimina el archivo\\ ingresado\end{tabular}}                             \\ \hline
		\multicolumn{1}{|l|}{}                                                                                                & \multicolumn{1}{l|}{\begin{tabular}[c]{@{}l@{}}Sistema notifica el correcto\\ t�rmino del proceso\end{tabular}}                 \\ \hline
		\multicolumn{1}{|l|}{\textbf{Referencias Cruzadas}}                                                                   & \multicolumn{1}{l|}{--}                                                                                                         \\ \hline
		\multicolumn{1}{|l|}{\textbf{Cursos alternativos}}                                                                    & \multicolumn{1}{l|}{--}                                                                                                         \\ \hline
		&                                                                                                                                 \\ \hline
		\multicolumn{2}{|c|}{\textit{\textbf{Caso de Uso: CU41}}}                                                                                                                                                                                               \\ \hline
		\multicolumn{1}{|l|}{\textbf{Nombre}}                                                                                 & \multicolumn{1}{l|}{\begin{tabular}[c]{@{}l@{}}Agregar archivo en\\ �rea de trabajo\end{tabular}}                               \\ \hline
		\multicolumn{1}{|l|}{\textbf{Actores}}                                                                                & \multicolumn{1}{l|}{U}                                                                                                          \\ \hline
		\multicolumn{1}{|l|}{\textbf{Funciones Asociadas}}                                                                    & \multicolumn{1}{l|}{}                                                                                                           \\ \hline
		\multicolumn{1}{|l|}{\textbf{Objetivo}}                                                                               & \multicolumn{1}{l|}{\begin{tabular}[c]{@{}l@{}}Agregar archivo en\\ �rea de trabajo\end{tabular}}                               \\ \hline
		\multicolumn{1}{|l|}{\textbf{Pre Condiciones}}                                                                        & \multicolumn{1}{l|}{\begin{tabular}[c]{@{}l@{}}Archivo existe en el �rea\\ de trabajo parcial\end{tabular}}                     \\ \hline
		\multicolumn{1}{|l|}{\textbf{Post Condiciones}}                                                                       & \multicolumn{1}{l|}{\begin{tabular}[c]{@{}l@{}}Archivo agregado al �rea\\ de trabajo\end{tabular}}                              \\ \hline
		\multicolumn{1}{|c|}{\textit{\textbf{Acciones del Actor}}}                                                            & \multicolumn{1}{c|}{\textit{\textbf{Acciones del Sistema}}}                                                                     \\ \hline
		\multicolumn{1}{|l|}{\begin{tabular}[c]{@{}l@{}}Usuario sube un archivo\\ al �rea de trabajo\end{tabular}}            & \multicolumn{1}{l|}{}                                                                                                           \\ \hline
		\multicolumn{1}{|l|}{}                                                                                                & \multicolumn{1}{l|}{Sistema revisa el archivo}                                                                                  \\ \hline
		\multicolumn{1}{|l|}{}                                                                                                & \multicolumn{1}{l|}{\begin{tabular}[c]{@{}l@{}}Sistema determina correcto\\ archivo ingresado\end{tabular}}                     \\ \hline
		\multicolumn{1}{|l|}{}                                                                                                & \multicolumn{1}{l|}{\begin{tabular}[c]{@{}l@{}}Sistema aloja en el �rea de\\ trabajo el set de datos revisado\end{tabular}}     \\ \hline
		\multicolumn{1}{|l|}{\textbf{Referencias Cruzadas}}                                                                   & \multicolumn{1}{l|}{--}                                                                                                         \\ \hline
		\multicolumn{1}{|l|}{\textbf{Cursos alternativos}}                                                                    & \multicolumn{1}{l|}{--}                                                                                                         \\ \hline
		\caption{Casos de uso asociados al uso de recursos.}
		\label{CU02}\\
		\end{longtable}
	
	\subsection{Casos de uso asociados al sistema de almacenamiento persistente}
	
	El sistema de almacenamiento persistente tiene la responsabilidad de controlar todas las acciones relacionadas con el respaldo de data y mantenimiento de informaci�n de configuraci�n del sistema.
	
	% Please add the following required packages to your document preamble:
	% \usepackage{multirow}
	% \usepackage{longtable}
	% Note: It may be necessary to compile the document several times to get a multi-page table to line up properly
	\begin{longtable}{ll}
		\hline
		\multicolumn{2}{|c|}{\textit{\textbf{Caso de Uso: CU02}}}                                                                                                                                                                                                                                \\ \hline
		\endfirsthead
		%
		\endhead
		%
		\multicolumn{1}{|l|}{\textbf{Nombre}}                                                                                           & \multicolumn{1}{l|}{\begin{tabular}[c]{@{}l@{}}Insertar registro en Sistema de\\ Almacenamiento persistente\end{tabular}}                              \\ \hline
		\multicolumn{1}{|l|}{\textbf{Actores}}                                                                                          & \multicolumn{1}{l|}{AP}                                                                                                                                \\ \hline
		\multicolumn{1}{|l|}{\textbf{Funciones Asociadas}}                                                                              & \multicolumn{1}{l|}{}                                                                                                                                  \\ \hline
		\multicolumn{1}{|l|}{\textbf{Objetivo}}                                                                                         & \multicolumn{1}{l|}{\begin{tabular}[c]{@{}l@{}}Crear un nuevo registro en el sistema\\ de almacenamiento persistente\end{tabular}}                     \\ \hline
		\multicolumn{1}{|l|}{\textbf{Pre Condiciones}}                                                                                  & \multicolumn{1}{l|}{\begin{tabular}[c]{@{}l@{}}Sistema de almacenamiento persistente\\ sin registro\end{tabular}}                                      \\ \hline
		\multicolumn{1}{|l|}{\textbf{Post Condiciones}}                                                                                 & \multicolumn{1}{l|}{\begin{tabular}[c]{@{}l@{}}Sistema de almacenamiento persistente\\ con registro agregado\end{tabular}}                             \\ \hline
		\multicolumn{1}{|c|}{\textit{\textbf{Acciones del Actor}}}                                                                      & \multicolumn{1}{c|}{\textit{\textbf{Acciones del Sistema}}}                                                                                            \\ \hline
		\multicolumn{1}{|l|}{\begin{tabular}[c]{@{}l@{}}AP recibe una solicitud de registro\\ de datos en formato JSON\end{tabular}}    & \multicolumn{1}{l|}{}                                                                                                                                  \\ \hline
		\multicolumn{1}{|l|}{\begin{tabular}[c]{@{}l@{}}AP procesa conexi�n a la base de\\ datos (CU04)\end{tabular}}                   & \multicolumn{1}{l|}{}                                                                                                                                  \\ \hline
		\multicolumn{1}{|l|}{}                                                                                                          & \multicolumn{1}{l|}{\begin{tabular}[c]{@{}l@{}}Sistema recibe los datos mediante\\ proceso JSON\end{tabular}}                                          \\ \hline
		\multicolumn{1}{|l|}{}                                                                                                          & \multicolumn{1}{l|}{Sistema decodifica los datos}                                                                                                      \\ \hline
		\multicolumn{1}{|l|}{\begin{tabular}[c]{@{}l@{}}AP inserta los datos en el entorno\\ correspondiente\end{tabular}}              & \multicolumn{1}{l|}{}                                                                                                                                  \\ \hline
		\multicolumn{1}{|l|}{}                                                                                                          & \multicolumn{1}{l|}{\begin{tabular}[c]{@{}l@{}}Sistema notifica el correcto t�rmino\\ del proceso\end{tabular}}                                        \\ \hline
		\multicolumn{1}{|l|}{\begin{tabular}[c]{@{}l@{}}AP establece desconexi�n del\\ sistema BD (CU05)\end{tabular}}                  & \multicolumn{1}{l|}{}                                                                                                                                  \\ \hline
		\multicolumn{1}{|l|}{\textbf{Referencias Cruzadas}}                                                                             & \multicolumn{1}{l|}{CU04, CU05}                                                                                                                        \\ \hline
		\multicolumn{1}{|l|}{\textbf{Cursos alternativos}}                                                                              & \multicolumn{1}{l|}{\begin{tabular}[c]{@{}l@{}}No se puede iniciar sesi�n en BD, se\\ notifica y el proceso se encola\end{tabular}}                    \\ \hline
		&                                                                                                                                                        \\ \hline
		\multicolumn{2}{|c|}{\textit{\textbf{Caso de Uso: CU04}}}                                                                                                                                                                                                                                \\ \hline
		\multicolumn{1}{|l|}{\textbf{Nombre}}                                                                                           & \multicolumn{1}{l|}{Iniciar sesi�n en base de datos}                                                                                                   \\ \hline
		\multicolumn{1}{|l|}{\textbf{Actores}}                                                                                          & \multicolumn{1}{l|}{AP}                                                                                                                                \\ \hline
		\multicolumn{1}{|l|}{\textbf{Funciones Asociadas}}                                                                              & \multicolumn{1}{l|}{}                                                                                                                                  \\ \hline
		\multicolumn{1}{|l|}{\textbf{Objetivo}}                                                                                         & \multicolumn{1}{l|}{\begin{tabular}[c]{@{}l@{}}Iniciar nueva sesi�n en base\\ de datos\end{tabular}}                                                   \\ \hline
		\multicolumn{1}{|l|}{\textbf{Pre Condiciones}}                                                                                  & \multicolumn{1}{l|}{Sesi�n no iniciada en BD}                                                                                                          \\ \hline
		\multicolumn{1}{|l|}{\textbf{Post Condiciones}}                                                                                 & \multicolumn{1}{l|}{Sesi�n creada e iniciada en BD}                                                                                                    \\ \hline
		\multicolumn{1}{|c|}{\textit{\textbf{Acciones del Actor}}}                                                                      & \multicolumn{1}{c|}{\textit{\textbf{Acciones del Sistema}}}                                                                                            \\ \hline
		\multicolumn{1}{|l|}{AP solicita iniciar sesi�n en BD}                                                                          & \multicolumn{1}{l|}{}                                                                                                                                  \\ \hline
		\multicolumn{1}{|l|}{}                                                                                                          & \multicolumn{1}{l|}{Sistema recibe solicitud}                                                                                                          \\ \hline
		\multicolumn{1}{|l|}{}                                                                                                          & \multicolumn{1}{l|}{Sistema procesa las credenciales}                                                                                                  \\ \hline
		\multicolumn{1}{|l|}{}                                                                                                          & \multicolumn{1}{l|}{Sistema establece conexi�n}                                                                                                        \\ \hline
		\multicolumn{1}{|l|}{}                                                                                                          & \multicolumn{1}{l|}{Sistema notifica inicio de sesi�n correcto}                                                                                        \\ \hline
		\multicolumn{1}{|l|}{\textbf{Referencias Cruzadas}}                                                                             & \multicolumn{1}{l|}{--}                                                                                                                                \\ \hline
		\multicolumn{1}{|l|}{\multirow{2}{*}{\textbf{Cursos alternativos}}}                                                             & \multicolumn{1}{l|}{\begin{tabular}[c]{@{}l@{}}Credenciales de acceso incorrectas,\\ se notifica mediante el sistema\end{tabular}}                     \\ \cline{2-2} 
		\multicolumn{1}{|l|}{}                                                                                                          & \multicolumn{1}{l|}{\begin{tabular}[c]{@{}l@{}}Sistema BD no disponible por m�ltiples\\ conexiones activas, sistema notifica el\\ evento\end{tabular}} \\ \hline
		&                                                                                                                                                        \\ \hline
		\multicolumn{2}{|c|}{\textit{\textbf{Caso de Uso: CU05}}}                                                                                                                                                                                                                                \\ \hline
		\multicolumn{1}{|l|}{\textbf{Nombre}}                                                                                           & \multicolumn{1}{l|}{Cerrar sesi�n en BD}                                                                                                               \\ \hline
		\multicolumn{1}{|l|}{\textbf{Actores}}                                                                                          & \multicolumn{1}{l|}{AP}                                                                                                                                \\ \hline
		\multicolumn{1}{|l|}{\textbf{Funciones Asociadas}}                                                                              & \multicolumn{1}{l|}{}                                                                                                                                  \\ \hline
		\multicolumn{1}{|l|}{\textbf{Objetivo}}                                                                                         & \multicolumn{1}{l|}{\begin{tabular}[c]{@{}l@{}}Cerrar sesi�n iniciada en\\ base de datos\end{tabular}}                                                 \\ \hline
		\multicolumn{1}{|l|}{\textbf{Pre Condiciones}}                                                                                  & \multicolumn{1}{l|}{Sesi�n en estado iniciada}                                                                                                         \\ \hline
		\multicolumn{1}{|l|}{\textbf{Post Condiciones}}                                                                                 & \multicolumn{1}{l|}{Sesi�n en estado cerrada}                                                                                                          \\ \hline
		\multicolumn{1}{|c|}{\textit{\textbf{Acciones del Actor}}}                                                                      & \multicolumn{1}{c|}{\textit{\textbf{Acciones del Sistema}}}                                                                                            \\ \hline
		\multicolumn{1}{|l|}{AP solicita cerrar sesi�n}                                                                                 & \multicolumn{1}{l|}{}                                                                                                                                  \\ \hline
		\multicolumn{1}{|l|}{}                                                                                                          & \multicolumn{1}{l|}{Sistema BD recibe solicitud}                                                                                                       \\ \hline
		\multicolumn{1}{|l|}{}                                                                                                          & \multicolumn{1}{l|}{Sistema cierra la sesi�n}                                                                                                          \\ \hline
		\multicolumn{1}{|l|}{}                                                                                                          & \multicolumn{1}{l|}{\begin{tabular}[c]{@{}l@{}}Sistema elimina los datos de\\ memoria cach� y registros activos\end{tabular}}                          \\ \hline
		\multicolumn{1}{|l|}{}                                                                                                          & \multicolumn{1}{l|}{\begin{tabular}[c]{@{}l@{}}Sistema notifica el correcto\\ t�rmino del proceso\end{tabular}}                                        \\ \hline
		\multicolumn{1}{|l|}{\textbf{Referencias Cruzadas}}                                                                             & \multicolumn{1}{l|}{--}                                                                                                                                \\ \hline
		\multicolumn{1}{|l|}{\textbf{Cursos alternativos}}                                                                              & \multicolumn{1}{l|}{--}                                                                                                                                \\ \hline
		&                                                                                                                                                        \\ \hline
		\multicolumn{2}{|c|}{\textit{\textbf{Caso de Uso: CU07}}}                                                                                                                                                                                                                                \\ \hline
		\multicolumn{1}{|l|}{\textbf{Nombre}}                                                                                           & \multicolumn{1}{l|}{\begin{tabular}[c]{@{}l@{}}Editar registro en Sistema de\\ Almacenamiento persistente\end{tabular}}                                \\ \hline
		\multicolumn{1}{|l|}{\textbf{Actores}}                                                                                          & \multicolumn{1}{l|}{AP}                                                                                                                                \\ \hline
		\multicolumn{1}{|l|}{\textbf{Funciones Asociadas}}                                                                              & \multicolumn{1}{l|}{}                                                                                                                                  \\ \hline
		\multicolumn{1}{|l|}{\textbf{Objetivo}}                                                                                         & \multicolumn{1}{l|}{\begin{tabular}[c]{@{}l@{}}Editar un registro en el sistema de\\ almacenamiento persistente\end{tabular}}                          \\ \hline
		\multicolumn{1}{|l|}{\textbf{Pre Condiciones}}                                                                                  & \multicolumn{1}{l|}{\begin{tabular}[c]{@{}l@{}}Sistema de almacenamiento persistente\\ sin registro editado\end{tabular}}                              \\ \hline
		\multicolumn{1}{|l|}{\textbf{Post Condiciones}}                                                                                 & \multicolumn{1}{l|}{\begin{tabular}[c]{@{}l@{}}Sistema de almacenamiento persistente\\ con registro editado\end{tabular}}                              \\ \hline
		\multicolumn{1}{|c|}{\textit{\textbf{Acciones del Actor}}}                                                                      & \multicolumn{1}{c|}{\textit{\textbf{Acciones del Sistema}}}                                                                                            \\ \hline
		\multicolumn{1}{|l|}{\begin{tabular}[c]{@{}l@{}}AP recibe una solicitud de edici�n\\ de datos en formato JSON\end{tabular}}     & \multicolumn{1}{l|}{}                                                                                                                                  \\ \hline
		\multicolumn{1}{|l|}{\begin{tabular}[c]{@{}l@{}}AP procesa conexi�n a la base de\\ datos (CU04)\end{tabular}}                   & \multicolumn{1}{l|}{}                                                                                                                                  \\ \hline
		\multicolumn{1}{|l|}{}                                                                                                          & \multicolumn{1}{l|}{\begin{tabular}[c]{@{}l@{}}Sistema recibe los datos mediante\\ proceso JSON\end{tabular}}                                          \\ \hline
		\multicolumn{1}{|l|}{}                                                                                                          & \multicolumn{1}{l|}{Sistema decodifica los datos}                                                                                                      \\ \hline
		\multicolumn{1}{|l|}{\begin{tabular}[c]{@{}l@{}}AP edita los datos en el entorno\\ correspondiente\end{tabular}}                & \multicolumn{1}{l|}{}                                                                                                                                  \\ \hline
		\multicolumn{1}{|l|}{}                                                                                                          & \multicolumn{1}{l|}{\begin{tabular}[c]{@{}l@{}}Sistema notifica el correcto t�rmino\\ del proceso\end{tabular}}                                        \\ \hline
		\multicolumn{1}{|l|}{\begin{tabular}[c]{@{}l@{}}AP establece desconexi�n del\\ sistema BD (CU05)\end{tabular}}                  & \multicolumn{1}{l|}{}                                                                                                                                  \\ \hline
		\multicolumn{1}{|l|}{\textbf{Referencias Cruzadas}}                                                                             & \multicolumn{1}{l|}{CU04, CU05}                                                                                                                        \\ \hline
		\multicolumn{1}{|l|}{\textbf{Cursos alternativos}}                                                                              & \multicolumn{1}{l|}{\begin{tabular}[c]{@{}l@{}}No se puede iniciar sesi�n en BD,\\ se notifica y el proceso se encola\end{tabular}}                    \\ \hline
		&                                                                                                                                                        \\ \hline
		\multicolumn{2}{|c|}{\textit{\textbf{Caso de Uso: CU09}}}                                                                                                                                                                                                                                \\ \hline
		\multicolumn{1}{|l|}{\textbf{Nombre}}                                                                                           & \multicolumn{1}{l|}{\begin{tabular}[c]{@{}l@{}}Eliminar registro en Sistema de\\ Almacenamiento persistente\end{tabular}}                              \\ \hline
		\multicolumn{1}{|l|}{\textbf{Actores}}                                                                                          & \multicolumn{1}{l|}{AP}                                                                                                                                \\ \hline
		\multicolumn{1}{|l|}{\textbf{Funciones Asociadas}}                                                                              & \multicolumn{1}{l|}{}                                                                                                                                  \\ \hline
		\multicolumn{1}{|l|}{\textbf{Objetivo}}                                                                                         & \multicolumn{1}{l|}{\begin{tabular}[c]{@{}l@{}}Eliminar un registro en el sistema\\ de almacenamiento persistente\end{tabular}}                        \\ \hline
		\multicolumn{1}{|l|}{\textbf{Pre Condiciones}}                                                                                  & \multicolumn{1}{l|}{\begin{tabular}[c]{@{}l@{}}Sistema de almacenamiento persistente\\ sin registro eliminado\end{tabular}}                            \\ \hline
		\multicolumn{1}{|l|}{\textbf{Post Condiciones}}                                                                                 & \multicolumn{1}{l|}{\begin{tabular}[c]{@{}l@{}}Sistema de almacenamiento persistente\\ con registro eliminado\end{tabular}}                            \\ \hline
		\multicolumn{1}{|c|}{\textit{\textbf{Acciones del Actor}}}                                                                      & \multicolumn{1}{c|}{\textit{\textbf{Acciones del Sistema}}}                                                                                            \\ \hline
		\multicolumn{1}{|l|}{\begin{tabular}[c]{@{}l@{}}AP recibe una solicitud de eliminaci�n\\ de datos en formato JSON\end{tabular}} & \multicolumn{1}{l|}{}                                                                                                                                  \\ \hline
		\multicolumn{1}{|l|}{\begin{tabular}[c]{@{}l@{}}AP procesa conexi�n a la base de\\ datos (CU04)\end{tabular}}                   & \multicolumn{1}{l|}{}                                                                                                                                  \\ \hline
		\multicolumn{1}{|l|}{}                                                                                                          & \multicolumn{1}{l|}{\begin{tabular}[c]{@{}l@{}}Sistema recibe los datos mediante\\ proceso JSON\end{tabular}}                                          \\ \hline
		\multicolumn{1}{|l|}{}                                                                                                          & \multicolumn{1}{l|}{Sistema decodifica los datos}                                                                                                      \\ \hline
		\multicolumn{1}{|l|}{\begin{tabular}[c]{@{}l@{}}AP elimina los datos en el entorno\\ correspondiente\end{tabular}}              & \multicolumn{1}{l|}{}                                                                                                                                  \\ \hline
		\multicolumn{1}{|l|}{}                                                                                                          & \multicolumn{1}{l|}{\begin{tabular}[c]{@{}l@{}}Sistema notifica el correcto t�rmino\\ del proceso\end{tabular}}                                        \\ \hline
		\multicolumn{1}{|l|}{\begin{tabular}[c]{@{}l@{}}AP establece desconexi�n del sistema\\ BD (CU05)\end{tabular}}                  & \multicolumn{1}{l|}{}                                                                                                                                  \\ \hline
		\multicolumn{1}{|l|}{\textbf{Referencias Cruzadas}}                                                                             & \multicolumn{1}{l|}{CU04, CU05}                                                                                                                        \\ \hline
		\multicolumn{1}{|l|}{\textbf{Cursos alternativos}}                                                                              & \multicolumn{1}{l|}{\begin{tabular}[c]{@{}l@{}}No se puede iniciar sesi�n en BD,\\ se notifica y el proceso se encola\end{tabular}}                    \\ \hline
		&                                                                                                                                                        \\ \hline
		\multicolumn{2}{|c|}{\textit{\textbf{Caso de Uso: CU12}}}                                                                                                                                                                                                                                \\ \hline
		\multicolumn{1}{|l|}{\textbf{Nombre}}                                                                                           & \multicolumn{1}{l|}{\begin{tabular}[c]{@{}l@{}}Mostrar registro en Sistema de\\ Almacenamiento persistente\end{tabular}}                               \\ \hline
		\multicolumn{1}{|l|}{\textbf{Actores}}                                                                                          & \multicolumn{1}{l|}{AP}                                                                                                                                \\ \hline
		\multicolumn{1}{|l|}{\textbf{Funciones Asociadas}}                                                                              & \multicolumn{1}{l|}{}                                                                                                                                  \\ \hline
		\multicolumn{1}{|l|}{\textbf{Objetivo}}                                                                                         & \multicolumn{1}{l|}{\begin{tabular}[c]{@{}l@{}}Mostrar un conjunto de registros en\\ el sistema de almacenamiento\\ persistente\end{tabular}}          \\ \hline
		\multicolumn{1}{|l|}{\textbf{Pre Condiciones}}                                                                                  & \multicolumn{1}{l|}{\begin{tabular}[c]{@{}l@{}}Solicitud de b�squeda de informaci�n\\ en estado espera\end{tabular}}                                   \\ \hline
		\multicolumn{1}{|l|}{\textbf{Post Condiciones}}                                                                                 & \multicolumn{1}{l|}{\begin{tabular}[c]{@{}l@{}}Solicitud de b�squeda de informaci�n\\ en estado finalizada\end{tabular}}                               \\ \hline
		\multicolumn{1}{|l|}{\textit{\textbf{Acciones del Actor}}}                                                                      & \multicolumn{1}{l|}{\textit{\textbf{Acciones del Sistema}}}                                                                                            \\ \hline
		\multicolumn{1}{|l|}{\begin{tabular}[c]{@{}l@{}}AP recibe la solicitud de b�squeda\\ de datos\end{tabular}}                     & \multicolumn{1}{l|}{}                                                                                                                                  \\ \hline
		\multicolumn{1}{|l|}{\begin{tabular}[c]{@{}l@{}}AP procesa conexi�n a la base de\\ datos (CU04)\end{tabular}}                   & \multicolumn{1}{l|}{}                                                                                                                                  \\ \hline
		\multicolumn{1}{|l|}{}                                                                                                          & \multicolumn{1}{l|}{\begin{tabular}[c]{@{}l@{}}Sistema recibe los datos mediante\\ proceso JSON\end{tabular}}                                          \\ \hline
		\multicolumn{1}{|l|}{}                                                                                                          & \multicolumn{1}{l|}{\begin{tabular}[c]{@{}l@{}}Sistema responde la solicitud con el\\ correspondiente resultado obtenido\end{tabular}}                 \\ \hline
		\multicolumn{1}{|l|}{\begin{tabular}[c]{@{}l@{}}AP establece desconexi�n del\\ sistema BD (CU05)\end{tabular}}                  & \multicolumn{1}{l|}{}                                                                                                                                  \\ \hline
		\multicolumn{1}{|l|}{\textbf{Referencias Cruzadas}}                                                                             & \multicolumn{1}{l|}{CU04, CU05}                                                                                                                        \\ \hline
		\multicolumn{1}{|l|}{\textbf{Cursos alternativos}}                                                                              & \multicolumn{1}{l|}{\begin{tabular}[c]{@{}l@{}}No se puede iniciar sesi�n en BD,\\ se notifica y el proceso se encola\end{tabular}}                    \\ \hline
		\caption{Casos de uso relacionados al sistema de almacenamiento persistente.}
		\label{CU03}\\
		\end{longtable}
		
	
	\subsection{Casos de uso asociados al sistema de notificaci�n}
	
	El sistema de notificaci�n tiene la responsabilidad de establecer los protocolos de comunicaci�n hacia los usuarios en caso de que se genere un evento asociado.
	
	% Please add the following required packages to your document preamble:
	% \usepackage{multirow}
	% \usepackage{longtable}
	% Note: It may be necessary to compile the document several times to get a multi-page table to line up properly
	\begin{longtable}{|l|l|}
		\hline
		\multicolumn{2}{|c|}{\textit{\textbf{Caso de Uso: CU21}}}                                                                                                            \\ \hline
		\endfirsthead
		%
		\endhead
		%
		\textbf{Nombre}                                            & Notificar mensaje v�a correo electr�nico                                                                \\ \hline
		\textbf{Actores}                                           & AP, U                                                                                                   \\ \hline
		\textbf{Funciones Asociadas}                               &                                                                                                         \\ \hline
		\textbf{Objetivo}                                          & \begin{tabular}[c]{@{}l@{}}Enviar correo electr�nico a usuario\\ con mensaje asociado\end{tabular}      \\ \hline
		\textbf{Pre Condiciones}                                   & Sistema de mensajer�a disponible                                                                        \\ \hline
		\textbf{Post Condiciones}                                  & \begin{tabular}[c]{@{}l@{}}Sistema de mensajer�a con solicitud\\ realizada\end{tabular}                 \\ \hline
		\multicolumn{1}{|c|}{\textit{\textbf{Acciones del Actor}}} & \multicolumn{1}{c|}{\textit{\textbf{Acciones del Sistema}}}                                             \\ \hline
		AP genera solicitud de env�o de data                       &                                                                                                         \\ \hline
		& Sistema recibe la solicitud                                                                             \\ \hline
		& Sistema revisa la data                                                                                  \\ \hline
		& Sistema crea el cuerpo del correo                                                                       \\ \hline
		& \begin{tabular}[c]{@{}l@{}}Sistema env�a v�a SMTP el correo al\\ destino asociado\end{tabular}          \\ \hline
		& \begin{tabular}[c]{@{}l@{}}Sistema registra en su log el proceso\\ generado\end{tabular}                \\ \hline
		& \begin{tabular}[c]{@{}l@{}}Sistema notifica el proceso finalizado de\\ manera correcta\end{tabular}     \\ \hline
		\textbf{Referencias Cruzadas}                              & --                                                                                                      \\ \hline
		\multirow{2}{*}{\textbf{Cursos alternativos}}              & \begin{tabular}[c]{@{}l@{}}Sistema de mensajer�a no disponible,\\ se notifica la acci�n\end{tabular}    \\ \cline{2-2} 
		& \begin{tabular}[c]{@{}l@{}}Servidor de correos SMTP no habilitado,\\ se notifica la acci�n\end{tabular} \\ \hline
		\caption{Casos de uso relacionados al sistema de notificaciones}
		\label{CU04}\\
		\end{longtable}
		
	\subsection{Casos de uso asociados al manejo de sesiones}
	
	El manejo de sesiones en cualquier sistema de informaci�n es una de las tareas m�s relevantes, debido a que facilita la autenticaci�n de usuario y permite una restricci�n de las acciones que los usuarios puedan realizar en el software, adem�s permite mantener un control de los usuarios y desarrollar estad�sticas de uso.
	
	% Please add the following required packages to your document preamble:
	% \usepackage{multirow}
	% \usepackage{longtable}
	% Note: It may be necessary to compile the document several times to get a multi-page table to line up properly
	\begin{longtable}{ll}
		\hline
		\multicolumn{2}{|c|}{\textit{\textbf{Caso de Uso: CU17}}}                                                                                                                                                                                                                        \\ \hline
		\endfirsthead
		%
		\endhead
		%
		\multicolumn{1}{|l|}{\textbf{Nombre}}                                                                                  & \multicolumn{1}{l|}{Iniciar sesi�n}                                                                                                                     \\ \hline
		\multicolumn{1}{|l|}{\textbf{Actores}}                                                                                 & \multicolumn{1}{l|}{U, AP}                                                                                                                              \\ \hline
		\multicolumn{1}{|l|}{\textbf{Funciones Asociadas}}                                                                     & \multicolumn{1}{l|}{}                                                                                                                                   \\ \hline
		\multicolumn{1}{|l|}{\textbf{Objetivo}}                                                                                & \multicolumn{1}{l|}{\begin{tabular}[c]{@{}l@{}}Ingresar al sistema de informaci�n mediante\\ la validaci�n de las credenciales de usuario\end{tabular}} \\ \hline
		\multicolumn{1}{|l|}{\textbf{Pre Condiciones}}                                                                         & \multicolumn{1}{l|}{Solicitud de inicio de sesi�n}                                                                                                      \\ \hline
		\multicolumn{1}{|l|}{\textbf{Post Condiciones}}                                                                        & \multicolumn{1}{l|}{Inicio de sesi�n creado}                                                                                                            \\ \hline
		\multicolumn{1}{|c|}{\textit{\textbf{Acciones del Actor}}}                                                             & \multicolumn{1}{c|}{\textit{\textbf{Acciones del Sistema}}}                                                                                             \\ \hline
		\multicolumn{1}{|l|}{Usuario accede al formulario login}                                                               & \multicolumn{1}{l|}{}                                                                                                                                   \\ \hline
		\multicolumn{1}{|l|}{Usuario completa formulario}                                                                      & \multicolumn{1}{l|}{}                                                                                                                                   \\ \hline
		\multicolumn{1}{|l|}{}                                                                                                 & \multicolumn{1}{l|}{Sistema recupera la data del formulario}                                                                                            \\ \hline
		\multicolumn{1}{|l|}{}                                                                                                 & \multicolumn{1}{l|}{\begin{tabular}[c]{@{}l@{}}Sistema consulta la data al sistema de\\ almacenamiento persistente\end{tabular}}                        \\ \hline
		\multicolumn{1}{|l|}{}                                                                                                 & \multicolumn{1}{l|}{Sistema verifica la respuesta}                                                                                                      \\ \hline
		\multicolumn{1}{|l|}{}                                                                                                 & \multicolumn{1}{l|}{\begin{tabular}[c]{@{}l@{}}Sistema inicia la sesi�n y entrega\\ token de acceso\end{tabular}}                                       \\ \hline
		\multicolumn{1}{|l|}{\begin{tabular}[c]{@{}l@{}}Usuario validado correctamente\\ ingresa al sistema\end{tabular}}      & \multicolumn{1}{l|}{}                                                                                                                                   \\ \hline
		\multicolumn{1}{|l|}{\textbf{Referencias Cruzadas}}                                                                    & \multicolumn{1}{l|}{CU12}                                                                                                                               \\ \hline
		\multicolumn{1}{|l|}{\multirow{2}{*}{\textbf{Cursos alternativos}}}                                                    & \multicolumn{1}{l|}{\begin{tabular}[c]{@{}l@{}}No se puede acceder al AP,\\ se notifica al usuario\end{tabular}}                                        \\ \cline{2-2} 
		\multicolumn{1}{|l|}{}                                                                                                 & \multicolumn{1}{l|}{\begin{tabular}[c]{@{}l@{}}Credenciales incorrectas,\\ se notifica al usuario\end{tabular}}                                         \\ \hline
		&                                                                                                                                                         \\ \hline
		\multicolumn{2}{|c|}{\textit{\textbf{Caso de Uso: CU18}}}                                                                                                                                                                                                                        \\ \hline
		\multicolumn{1}{|l|}{\textbf{Nombre}}                                                                                  & \multicolumn{1}{l|}{Autenticar usuario}                                                                                                                 \\ \hline
		\multicolumn{1}{|l|}{\textbf{Actores}}                                                                                 & \multicolumn{1}{l|}{U, AP}                                                                                                                              \\ \hline
		\multicolumn{1}{|l|}{\textbf{Funciones Asociadas}}                                                                     & \multicolumn{1}{l|}{}                                                                                                                                   \\ \hline
		\multicolumn{1}{|l|}{\textbf{Objetivo}}                                                                                & \multicolumn{1}{l|}{Autenticar usuario en sistema}                                                                                                      \\ \hline
		\multicolumn{1}{|l|}{\textbf{Pre Condiciones}}                                                                         & \multicolumn{1}{l|}{\begin{tabular}[c]{@{}l@{}}Usuario registrado en sistema\\ pero sin acceso\end{tabular}}                                            \\ \hline
		\multicolumn{1}{|l|}{\textbf{Post Condiciones}}                                                                        & \multicolumn{1}{l|}{\begin{tabular}[c]{@{}l@{}}Usuario registrado en sistema y\\ habilitado para acceder\end{tabular}}                                  \\ \hline
		\multicolumn{1}{|c|}{\textit{\textbf{Acciones del Actor}}}                                                             & \multicolumn{1}{c|}{\textit{\textbf{Acciones del Sistema}}}                                                                                             \\ \hline
		\multicolumn{1}{|l|}{\begin{tabular}[c]{@{}l@{}}Administrador registra un nuevo\\ usuario\end{tabular}}                & \multicolumn{1}{l|}{}                                                                                                                                   \\ \hline
		\multicolumn{1}{|l|}{}                                                                                                 & \multicolumn{1}{l|}{Sistema responde solicitud}                                                                                                         \\ \hline
		\multicolumn{1}{|l|}{}                                                                                                 & \multicolumn{1}{l|}{Sistema inicia proceso de autenticaci�n}                                                                                            \\ \hline
		\multicolumn{1}{|l|}{}                                                                                                 & \multicolumn{1}{l|}{Sistema genera token de registro}                                                                                                   \\ \hline
		\multicolumn{1}{|l|}{}                                                                                                 & \multicolumn{1}{l|}{Sistema almacena registro en BD}                                                                                                    \\ \hline
		\multicolumn{1}{|l|}{}                                                                                                 & \multicolumn{1}{l|}{Sistema codifica credenciales de usuario}                                                                                           \\ \hline
		\multicolumn{1}{|l|}{}                                                                                                 & \multicolumn{1}{l|}{\begin{tabular}[c]{@{}l@{}}Sistema habilita las credenciales\\ para usuario registrado\end{tabular}}                                \\ \hline
		\multicolumn{1}{|l|}{\textbf{Referencias Cruzadas}}                                                                    & \multicolumn{1}{l|}{CU12}                                                                                                                               \\ \hline
		\multicolumn{1}{|l|}{\multirow{2}{*}{\textbf{Cursos alternativos}}}                                                    & \multicolumn{1}{l|}{\begin{tabular}[c]{@{}l@{}}No se puede acceder al AP,\\ se notifica al usuario\end{tabular}}                                        \\ \cline{2-2} 
		\multicolumn{1}{|l|}{}                                                                                                 & \multicolumn{1}{l|}{\begin{tabular}[c]{@{}l@{}}Credenciales incorrectas,\\ se notifica al usuario\end{tabular}}                                         \\ \hline
		&                                                                                                                                                         \\ \hline
		\multicolumn{2}{|c|}{\textit{\textbf{Caso de Uso: CU19}}}                                                                                                                                                                                                                        \\ \hline
		\multicolumn{1}{|l|}{\textbf{Nombre}}                                                                                  & \multicolumn{1}{l|}{Cerrar sesi�n}                                                                                                                      \\ \hline
		\multicolumn{1}{|l|}{\textbf{Actores}}                                                                                 & \multicolumn{1}{l|}{U, AP}                                                                                                                              \\ \hline
		\multicolumn{1}{|l|}{\textbf{Funciones Asociadas}}                                                                     & \multicolumn{1}{l|}{}                                                                                                                                   \\ \hline
		\multicolumn{1}{|l|}{\textbf{Objetivo}}                                                                                & \multicolumn{1}{l|}{\begin{tabular}[c]{@{}l@{}}Salir del sistema de informaci�n\\ mediante la validaci�n de las\\ credenciales de usuario\end{tabular}} \\ \hline
		\multicolumn{1}{|l|}{\textbf{Pre Condiciones}}                                                                         & \multicolumn{1}{l|}{Solicitud de cierre de sesi�n}                                                                                                      \\ \hline
		\multicolumn{1}{|l|}{\textbf{Post Condiciones}}                                                                        & \multicolumn{1}{l|}{\begin{tabular}[c]{@{}l@{}}Cierre de sesi�n procesado de\\ manera correcta\end{tabular}}                                            \\ \hline
		\multicolumn{1}{|c|}{\textit{\textbf{Acciones del Actor}}}                                                             & \multicolumn{1}{c|}{\textit{\textbf{Acciones del Sistema}}}                                                                                             \\ \hline
		\multicolumn{1}{|l|}{Usuario solicita cerrar sesi�n}                                                                   & \multicolumn{1}{l|}{}                                                                                                                                   \\ \hline
		\multicolumn{1}{|l|}{}                                                                                                 & \multicolumn{1}{l|}{Sistema recibe la solicitud}                                                                                                        \\ \hline
		\multicolumn{1}{|l|}{}                                                                                                 & \multicolumn{1}{l|}{Sistema registra la hora de acceso}                                                                                                 \\ \hline
		\multicolumn{1}{|l|}{}                                                                                                 & \multicolumn{1}{l|}{\begin{tabular}[c]{@{}l@{}}Sistema almacena en sus log el registro\\ de finalizar sesi�n\end{tabular}}                              \\ \hline
		\multicolumn{1}{|l|}{}                                                                                                 & \multicolumn{1}{l|}{Sistema borra la data cach�}                                                                                                        \\ \hline
		\multicolumn{1}{|l|}{}                                                                                                 & \multicolumn{1}{l|}{Sistema elimina token generado}                                                                                                     \\ \hline
		\multicolumn{1}{|l|}{}                                                                                                 & \multicolumn{1}{l|}{\begin{tabular}[c]{@{}l@{}}Sistema cierra sesi�n redireccionando\\ a la interfaz principal\end{tabular}}                            \\ \hline
		\multicolumn{1}{|l|}{\textbf{Referencias Cruzadas}}                                                                    & \multicolumn{1}{l|}{CU12}                                                                                                                               \\ \hline
		\multicolumn{1}{|l|}{\multirow{2}{*}{\textbf{Cursos alternativos}}}                                                    & \multicolumn{1}{l|}{\begin{tabular}[c]{@{}l@{}}No se puede acceder al AP,\\ se notifica al usuario\end{tabular}}                                        \\ \cline{2-2} 
		\multicolumn{1}{|l|}{}                                                                                                 & \multicolumn{1}{l|}{\begin{tabular}[c]{@{}l@{}}Credenciales incorrectas,\\ se notifica al usuario\end{tabular}}                                         \\ \hline
		&                                                                                                                                                         \\ \hline
		\multicolumn{2}{|c|}{\textit{\textbf{Caso de Uso: CU20}}}                                                                                                                                                                                                                        \\ \hline
		\multicolumn{1}{|l|}{\textbf{Nombre}}                                                                                  & \multicolumn{1}{l|}{Recuperar cuenta de usuario}                                                                                                        \\ \hline
		\multicolumn{1}{|l|}{\textbf{Actores}}                                                                                 & \multicolumn{1}{l|}{U, AP}                                                                                                                              \\ \hline
		\multicolumn{1}{|l|}{\textbf{Funciones Asociadas}}                                                                     & \multicolumn{1}{l|}{}                                                                                                                                   \\ \hline
		\multicolumn{1}{|l|}{\textbf{Objetivo}}                                                                                & \multicolumn{1}{l|}{Recuperar acceso al sistema}                                                                                                        \\ \hline
		\multicolumn{1}{|l|}{\textbf{Pre Condiciones}}                                                                         & \multicolumn{1}{l|}{\begin{tabular}[c]{@{}l@{}}Solicitud realizada de recuperaci�n\\ de cuenta\end{tabular}}                                            \\ \hline
		\multicolumn{1}{|l|}{\textbf{Post Condiciones}}                                                                        & \multicolumn{1}{l|}{\begin{tabular}[c]{@{}l@{}}Solicitud de recuperaci�n de cuentas\\ finalizada\end{tabular}}                                          \\ \hline
		\multicolumn{1}{|c|}{\textit{\textbf{Acciones del Actor}}}                                                             & \multicolumn{1}{c|}{\textit{\textbf{Acciones del Sistema}}}                                                                                             \\ \hline
		\multicolumn{1}{|l|}{\begin{tabular}[c]{@{}l@{}}Usuario accede al formulario\\ de recuperaci�n de cuenta\end{tabular}} & \multicolumn{1}{l|}{}                                                                                                                                   \\ \hline
		\multicolumn{1}{|l|}{\begin{tabular}[c]{@{}l@{}}Usuario completa formulario\\ de recuperaci�n\end{tabular}}            & \multicolumn{1}{l|}{}                                                                                                                                   \\ \hline
		\multicolumn{1}{|l|}{}                                                                                                 & \multicolumn{1}{l|}{Sistema recibe la solicitud}                                                                                                        \\ \hline
		\multicolumn{1}{|l|}{}                                                                                                 & \multicolumn{1}{l|}{\begin{tabular}[c]{@{}l@{}}Sistema valida que la data\\ ingresada sea correcta\end{tabular}}                                        \\ \hline
		\multicolumn{1}{|l|}{}                                                                                                 & \multicolumn{1}{l|}{\begin{tabular}[c]{@{}l@{}}Sistema notifica v�a correo electr�nico\\ link de recuperaci�n\end{tabular}}                             \\ \hline
		\multicolumn{1}{|l|}{}                                                                                                 & \multicolumn{1}{l|}{\begin{tabular}[c]{@{}l@{}}Sistema habilita link de recuperaci�n\\ por un tiempo m�ximo de 30 minutos\end{tabular}}                 \\ \hline
		\multicolumn{1}{|l|}{\begin{tabular}[c]{@{}l@{}}Usuario recibe correo electr�nico\\ y accede a link\end{tabular}}      & \multicolumn{1}{l|}{}                                                                                                                                   \\ \hline
		\multicolumn{1}{|l|}{Usuario completa formulario}                                                                      & \multicolumn{1}{l|}{}                                                                                                                                   \\ \hline
		\multicolumn{1}{|l|}{}                                                                                                 & \multicolumn{1}{l|}{Sistema recibe la solicitud}                                                                                                        \\ \hline
		\multicolumn{1}{|l|}{}                                                                                                 & \multicolumn{1}{l|}{Sistema actualiza la data}                                                                                                          \\ \hline
		\multicolumn{1}{|l|}{}                                                                                                 & \multicolumn{1}{l|}{Sistema autentica al usuario}                                                                                                       \\ \hline
		\multicolumn{1}{|l|}{}                                                                                                 & \multicolumn{1}{l|}{\begin{tabular}[c]{@{}l@{}}Sistema notifica v�a correo electr�nico\\ la recuperaci�n de cuenta\end{tabular}}                        \\ \hline
		\multicolumn{1}{|l|}{\textbf{Referencias Cruzadas}}                                                                    & \multicolumn{1}{l|}{CU12, CU21}                                                                                                                         \\ \hline
		\multicolumn{1}{|l|}{\multirow{3}{*}{\textbf{Cursos alternativos}}}                                                    & \multicolumn{1}{l|}{\begin{tabular}[c]{@{}l@{}}No se puede acceder al AP,\\ se notifica al usuario\end{tabular}}                                        \\ \cline{2-2} 
		\multicolumn{1}{|l|}{}                                                                                                 & \multicolumn{1}{l|}{\begin{tabular}[c]{@{}l@{}}Credenciales incorrectas,\\ se notifica al usuario\end{tabular}}                                         \\ \cline{2-2} 
		\multicolumn{1}{|l|}{}                                                                                                 & \multicolumn{1}{l|}{\begin{tabular}[c]{@{}l@{}}Sistema de correos no se encuentra\\ disponible, se notifica al usuario la acci�n\end{tabular}}          \\ \hline
		&                                                                                                                                                         \\ \hline
		\multicolumn{2}{|c|}{\textit{\textbf{Caso de Uso: CU22}}}                                                                                                                                                                                                                        \\ \hline
		\multicolumn{1}{|l|}{\textbf{Nombre}}                                                                                  & \multicolumn{1}{l|}{Reestablecer cuenta de usuario}                                                                                                     \\ \hline
		\multicolumn{1}{|l|}{\textbf{Actores}}                                                                                 & \multicolumn{1}{l|}{U, AP}                                                                                                                              \\ \hline
		\multicolumn{1}{|l|}{\textbf{Funciones Asociadas}}                                                                     & \multicolumn{1}{l|}{}                                                                                                                                   \\ \hline
		\multicolumn{1}{|l|}{\textbf{Objetivo}}                                                                                & \multicolumn{1}{l|}{Reestablecer cuenta de usuario}                                                                                                     \\ \hline
		\multicolumn{1}{|l|}{\textbf{Pre Condiciones}}                                                                         & \multicolumn{1}{l|}{\begin{tabular}[c]{@{}l@{}}Solicitud realizada de reestablecer\\ cuenta de usuario\end{tabular}}                                    \\ \hline
		\multicolumn{1}{|l|}{\textbf{Post Condiciones}}                                                                        & \multicolumn{1}{l|}{\begin{tabular}[c]{@{}l@{}}Solicitud de reestablecer cuenta\\ finalizada\end{tabular}}                                              \\ \hline
		\multicolumn{1}{|c|}{\textit{\textbf{Acciones del Actor}}}                                                             & \multicolumn{1}{c|}{\textit{\textbf{Acciones del Sistema}}}                                                                                             \\ \hline
		\multicolumn{1}{|l|}{\begin{tabular}[c]{@{}l@{}}Usuario accede al formulario de\\ recuperaci�n de cuenta\end{tabular}} & \multicolumn{1}{l|}{}                                                                                                                                   \\ \hline
		\multicolumn{1}{|l|}{Usuario completa formulario de recuperaci�n}                                                      & \multicolumn{1}{l|}{}                                                                                                                                   \\ \hline
		\multicolumn{1}{|l|}{}                                                                                                 & \multicolumn{1}{l|}{Sistema recibe la solicitud}                                                                                                        \\ \hline
		\multicolumn{1}{|l|}{}                                                                                                 & \multicolumn{1}{l|}{\begin{tabular}[c]{@{}l@{}}Sistema valida que la data ingresada\\ sea correcta\end{tabular}}                                        \\ \hline
		\multicolumn{1}{|l|}{}                                                                                                 & \multicolumn{1}{l|}{\begin{tabular}[c]{@{}l@{}}Sistema notifica v�a correo electr�nico\\ link de recuperaci�n\end{tabular}}                             \\ \hline
		\multicolumn{1}{|l|}{}                                                                                                 & \multicolumn{1}{l|}{\begin{tabular}[c]{@{}l@{}}Sistema habilita link de recuperaci�n\\ por un tiempo m�ximo de 30 minutos\end{tabular}}                 \\ \hline
		\multicolumn{1}{|l|}{\begin{tabular}[c]{@{}l@{}}Usuario recibe correo electr�nico\\ y accede a link\end{tabular}}      & \multicolumn{1}{l|}{}                                                                                                                                   \\ \hline
		\multicolumn{1}{|l|}{Usuario completa formulario}                                                                      & \multicolumn{1}{l|}{}                                                                                                                                   \\ \hline
		\multicolumn{1}{|l|}{}                                                                                                 & \multicolumn{1}{l|}{Sistema recibe la solicitud}                                                                                                        \\ \hline
		\multicolumn{1}{|l|}{}                                                                                                 & \multicolumn{1}{l|}{Sistema actualiza la data}                                                                                                          \\ \hline
		\multicolumn{1}{|l|}{}                                                                                                 & \multicolumn{1}{l|}{Sistema autentica al usuario}                                                                                                       \\ \hline
		\multicolumn{1}{|l|}{}                                                                                                 & \multicolumn{1}{l|}{\begin{tabular}[c]{@{}l@{}}Sistema notifica v�a correo electr�nico\\ la recuperaci�n de cuenta\end{tabular}}                        \\ \hline
		\multicolumn{1}{|l|}{\textbf{Referencias Cruzadas}}                                                                    & \multicolumn{1}{l|}{CU12, CU21}                                                                                                                         \\ \hline
		\multicolumn{1}{|l|}{\multirow{3}{*}{\textbf{Cursos alternativos}}}                                                    & \multicolumn{1}{l|}{No se puede acceder al AP, se notifica al usuario}                                                                                  \\ \cline{2-2} 
		\multicolumn{1}{|l|}{}                                                                                                 & \multicolumn{1}{l|}{Credenciales incorrectas, se notifica al usuario}                                                                                   \\ \cline{2-2} 
		\multicolumn{1}{|l|}{}                                                                                                 & \multicolumn{1}{l|}{\begin{tabular}[c]{@{}l@{}}Sistema de correos no se encuentra\\ disponible, se notifica al usuario la acci�n\end{tabular}}          \\ \hline
		&                                                                                                                                                         \\ \hline
		\multicolumn{2}{|c|}{\textit{\textbf{Caso de Uso: CU23}}}                                                                                                                                                                                                                        \\ \hline
		\multicolumn{1}{|l|}{\textbf{Nombre}}                                                                                  & \multicolumn{1}{l|}{Modificar datos de acceso}                                                                                                          \\ \hline
		\multicolumn{1}{|l|}{\textbf{Actores}}                                                                                 & \multicolumn{1}{l|}{U, AP}                                                                                                                              \\ \hline
		\multicolumn{1}{|l|}{\textbf{Funciones Asociadas}}                                                                     & \multicolumn{1}{l|}{}                                                                                                                                   \\ \hline
		\multicolumn{1}{|l|}{\textbf{Objetivo}}                                                                                & \multicolumn{1}{l|}{\begin{tabular}[c]{@{}l@{}}Modificar los valores de las credenciales\\ para un usuario\end{tabular}}                                \\ \hline
		\multicolumn{1}{|l|}{\textbf{Pre Condiciones}}                                                                         & \multicolumn{1}{l|}{Credenciales de usuario no modificadas}                                                                                             \\ \hline
		\multicolumn{1}{|l|}{\textbf{Post Condiciones}}                                                                        & \multicolumn{1}{l|}{Credenciales de usuario modificadas}                                                                                                \\ \hline
		\multicolumn{1}{|c|}{\textit{\textbf{Acciones del Actor}}}                                                             & \multicolumn{1}{c|}{\textit{\textbf{Acciones del Sistema}}}                                                                                             \\ \hline
		\multicolumn{1}{|l|}{\begin{tabular}[c]{@{}l@{}}Usuario accede al formulario de\\ cambio de contrase�a\end{tabular}}   & \multicolumn{1}{l|}{}                                                                                                                                   \\ \hline
		\multicolumn{1}{|l|}{Usuario completa formulario de cambio}                                                            & \multicolumn{1}{l|}{}                                                                                                                                   \\ \hline
		\multicolumn{1}{|l|}{}                                                                                                 & \multicolumn{1}{l|}{Sistema recibe la solicitud}                                                                                                        \\ \hline
		\multicolumn{1}{|l|}{}                                                                                                 & \multicolumn{1}{l|}{\begin{tabular}[c]{@{}l@{}}Sistema valida que la data ingresada\\ sea correcta\end{tabular}}                                        \\ \hline
		\multicolumn{1}{|l|}{}                                                                                                 & \multicolumn{1}{l|}{\begin{tabular}[c]{@{}l@{}}Sistema notifica v�a correo electr�nico\\ el cambio generado\end{tabular}}                               \\ \hline
		\multicolumn{1}{|l|}{}                                                                                                 & \multicolumn{1}{l|}{Sistema actualiza la data}                                                                                                          \\ \hline
		\multicolumn{1}{|l|}{}                                                                                                 & \multicolumn{1}{l|}{Sistema autentica al usuario}                                                                                                       \\ \hline
		\multicolumn{1}{|l|}{}                                                                                                 & \multicolumn{1}{l|}{\begin{tabular}[c]{@{}l@{}}Sistema notifica v�a correo electr�nico\\ la recuperaci�n de cuenta\end{tabular}}                        \\ \hline
		\multicolumn{1}{|l|}{\textbf{Referencias Cruzadas}}                                                                    & \multicolumn{1}{l|}{CU12, CU21}                                                                                                                         \\ \hline
		\multicolumn{1}{|l|}{\multirow{3}{*}{\textbf{Cursos alternativos}}}                                                    & \multicolumn{1}{l|}{No se puede acceder al AP, se notifica al usuario}                                                                                  \\ \cline{2-2} 
		\multicolumn{1}{|l|}{}                                                                                                 & \multicolumn{1}{l|}{Credenciales incorrectas, se notifica al usuario}                                                                                   \\ \cline{2-2} 
		\multicolumn{1}{|l|}{}                                                                                                 & \multicolumn{1}{l|}{\begin{tabular}[c]{@{}l@{}}Sistema de correos no se encuentra disponible,\\ se notifica al usuario la acci�n\end{tabular}}          \\ \hline
		\caption{Casos de uso asociados al manejo de sesiones en el sistema}
		\label{CU05}\\
		\end{longtable}
	
	\subsection{Casos de uso asociados al m�dulo estad�sticas de uso}
	
	Estos casos de uso est�n relacionados a las estad�sticas que implican el uso del software por parte de los usuarios y la carga del servidor.
	
	% Please add the following required packages to your document preamble:
	% \usepackage{longtable}
	% Note: It may be necessary to compile the document several times to get a multi-page table to line up properly
	\begin{longtable}{ll}
		\hline
		\multicolumn{2}{|c|}{\textit{\textbf{Caso de Uso: CU24}}}                                                                                                                                                                                           \\ \hline
		\endfirsthead
		%
		\endhead
		%
		\multicolumn{1}{|l|}{\textbf{Nombre}}                                                                                 & \multicolumn{1}{l|}{Visualizar estad�sticas de uso}                                                                         \\ \hline
		\multicolumn{1}{|l|}{\textbf{Actores}}                                                                                & \multicolumn{1}{l|}{AP, U}                                                                                                  \\ \hline
		\multicolumn{1}{|l|}{\textbf{Funciones Asociadas}}                                                                    & \multicolumn{1}{l|}{}                                                                                                       \\ \hline
		\multicolumn{1}{|l|}{\textbf{Objetivo}}                                                                               & \multicolumn{1}{l|}{\begin{tabular}[c]{@{}l@{}}Visualizar estad�sticas de uso del\\ servicio por usuario\end{tabular}}      \\ \hline
		\multicolumn{1}{|l|}{\textbf{Pre Condiciones}}                                                                        & \multicolumn{1}{l|}{\begin{tabular}[c]{@{}l@{}}Solicitud de estad�sticas de uso\\ realizada\end{tabular}}                   \\ \hline
		\multicolumn{1}{|l|}{\textbf{Post Condiciones}}                                                                       & \multicolumn{1}{l|}{\begin{tabular}[c]{@{}l@{}}Solicitud de estad�sticas de uso\\ respondida\end{tabular}}                  \\ \hline
		\multicolumn{1}{|c|}{\textit{\textbf{Acciones del Actor}}}                                                            & \multicolumn{1}{c|}{\textit{\textbf{Acciones del Sistema}}}                                                                 \\ \hline
		\multicolumn{1}{|l|}{\begin{tabular}[c]{@{}l@{}}Usuario accede a secci�n de\\ estad�sticas de uso\end{tabular}}       & \multicolumn{1}{l|}{}                                                                                                       \\ \hline
		\multicolumn{1}{|l|}{}                                                                                                & \multicolumn{1}{l|}{Sistema recibe la solicitud}                                                                            \\ \hline
		\multicolumn{1}{|l|}{}                                                                                                & \multicolumn{1}{l|}{\begin{tabular}[c]{@{}l@{}}Sistema consulta a AP para solicitud\\ de informaci�n\end{tabular}}          \\ \hline
		\multicolumn{1}{|l|}{AP recibe la solicitud y la procesa}                                                             & \multicolumn{1}{l|}{}                                                                                                       \\ \hline
		\multicolumn{1}{|l|}{}                                                                                                & \multicolumn{1}{l|}{Sistema recibe respuesta AP}                                                                            \\ \hline
		\multicolumn{1}{|l|}{}                                                                                                & \multicolumn{1}{l|}{Sistema procesa respuesta}                                                                              \\ \hline
		\multicolumn{1}{|l|}{}                                                                                                & \multicolumn{1}{l|}{\begin{tabular}[c]{@{}l@{}}Sistema decodifica respuesta en\\ formato JSON\end{tabular}}                 \\ \hline
		\multicolumn{1}{|l|}{}                                                                                                & \multicolumn{1}{l|}{\begin{tabular}[c]{@{}l@{}}Sistema genera tabla resumen de\\ estad�sticas recibidas\end{tabular}}       \\ \hline
		\multicolumn{1}{|l|}{}                                                                                                & \multicolumn{1}{l|}{\begin{tabular}[c]{@{}l@{}}Sistema genera gr�fico de\\ reportes con el uso\end{tabular}}                \\ \hline
		\multicolumn{1}{|l|}{\textbf{Referencias Cruzadas}}                                                                   & \multicolumn{1}{l|}{CU12, CU25, CU26}                                                                                       \\ \hline
		\multicolumn{1}{|l|}{\textbf{Cursos alternativos}}                                                                    & \multicolumn{1}{l|}{--}                                                                                                     \\ \hline
		&                                                                                                                             \\ \hline
		\multicolumn{2}{|c|}{\textit{\textbf{Caso de Uso: CU27}}}                                                                                                                                                                                           \\ \hline
		\multicolumn{1}{|l|}{\textbf{Nombre}}                                                                                 & \multicolumn{1}{l|}{Visualizar carga de servidores}                                                                         \\ \hline
		\multicolumn{1}{|l|}{\textbf{Actores}}                                                                                & \multicolumn{1}{l|}{AP, U}                                                                                                  \\ \hline
		\multicolumn{1}{|l|}{\textbf{Funciones Asociadas}}                                                                    & \multicolumn{1}{l|}{}                                                                                                       \\ \hline
		\multicolumn{1}{|l|}{\textbf{Objetivo}}                                                                               & \multicolumn{1}{l|}{\begin{tabular}[c]{@{}l@{}}Visualizar carga de servidores\\ en tiempo real\end{tabular}}                \\ \hline
		\multicolumn{1}{|l|}{\textbf{Pre Condiciones}}                                                                        & \multicolumn{1}{l|}{Solicitud de estado de servidores}                                                                      \\ \hline
		\multicolumn{1}{|l|}{\textbf{Post Condiciones}}                                                                       & \multicolumn{1}{l|}{\begin{tabular}[c]{@{}l@{}}Solicitud de estado de servidores\\ respondida\end{tabular}}                 \\ \hline
		\multicolumn{1}{|c|}{\textit{\textbf{Acciones del Actor}}}                                                            & \multicolumn{1}{c|}{\textit{\textbf{Acciones del Sistema}}}                                                                 \\ \hline
		\multicolumn{1}{|l|}{\begin{tabular}[c]{@{}l@{}}Usuario accede a secci�n de carga\\ de servidores\end{tabular}}       & \multicolumn{1}{l|}{}                                                                                                       \\ \hline
		\multicolumn{1}{|l|}{}                                                                                                & \multicolumn{1}{l|}{Sistema recibe la solicitud}                                                                            \\ \hline
		\multicolumn{1}{|l|}{}                                                                                                & \multicolumn{1}{l|}{\begin{tabular}[c]{@{}l@{}}Sistema consulta a AP para solicitud\\ de informaci�n\end{tabular}}          \\ \hline
		\multicolumn{1}{|l|}{AP recibe la solicitud y la procesa}                                                             & \multicolumn{1}{l|}{}                                                                                                       \\ \hline
		\multicolumn{1}{|l|}{}                                                                                                & \multicolumn{1}{l|}{Sistema recibe respuesta AP}                                                                            \\ \hline
		\multicolumn{1}{|l|}{}                                                                                                & \multicolumn{1}{l|}{Sistema procesa respuesta}                                                                              \\ \hline
		\multicolumn{1}{|l|}{}                                                                                                & \multicolumn{1}{l|}{\begin{tabular}[c]{@{}l@{}}Sistema decodifica respuesta en\\ formato JSON\end{tabular}}                 \\ \hline
		\multicolumn{1}{|l|}{}                                                                                                & \multicolumn{1}{l|}{\begin{tabular}[c]{@{}l@{}}Sistema genera tabla resumen de\\ estad�sticas recibidas\end{tabular}}       \\ \hline
		\multicolumn{1}{|l|}{}                                                                                                & \multicolumn{1}{l|}{\begin{tabular}[c]{@{}l@{}}Sistema genera gr�fico de\\ reportes con el uso\end{tabular}}                \\ \hline
		\multicolumn{1}{|l|}{\textbf{Referencias Cruzadas}}                                                                   & \multicolumn{1}{l|}{CU12, CU25, CU26}                                                                                       \\ \hline
		\multicolumn{1}{|l|}{\textbf{Cursos alternativos}}                                                                    & \multicolumn{1}{l|}{--}                                                                                                     \\ \hline
		&                                                                                                                             \\ \hline
		\multicolumn{2}{|c|}{\textit{\textbf{Caso de Uso: CU28}}}                                                                                                                                                                                           \\ \hline
		\multicolumn{1}{|l|}{\textbf{Nombre}}                                                                                 & \multicolumn{1}{l|}{\begin{tabular}[c]{@{}l@{}}Visualizar carga de sistema\\ de colas\end{tabular}}                         \\ \hline
		\multicolumn{1}{|l|}{\textbf{Actores}}                                                                                & \multicolumn{1}{l|}{AP, U}                                                                                                  \\ \hline
		\multicolumn{1}{|l|}{\textbf{Funciones Asociadas}}                                                                    & \multicolumn{1}{l|}{}                                                                                                       \\ \hline
		\multicolumn{1}{|l|}{\textbf{Objetivo}}                                                                               & \multicolumn{1}{l|}{\begin{tabular}[c]{@{}l@{}}Visualizar carga del sistema de\\ colas de procesos\end{tabular}}            \\ \hline
		\multicolumn{1}{|l|}{\textbf{Pre Condiciones}}                                                                        & \multicolumn{1}{l|}{\begin{tabular}[c]{@{}l@{}}Solicitud de carga del sistema de\\ colas generada\end{tabular}}             \\ \hline
		\multicolumn{1}{|l|}{\textbf{Post Condiciones}}                                                                       & \multicolumn{1}{l|}{\begin{tabular}[c]{@{}l@{}}Solicitud de carga del sistema de\\ colas respondida\end{tabular}}           \\ \hline
		\multicolumn{1}{|c|}{\textit{\textbf{Acciones del Actor}}}                                                            & \multicolumn{1}{c|}{\textit{\textbf{Acciones del Sistema}}}                                                                 \\ \hline
		\multicolumn{1}{|l|}{\begin{tabular}[c]{@{}l@{}}Usuario accede a secci�n de\\ carga de sistema de colas\end{tabular}} & \multicolumn{1}{l|}{}                                                                                                       \\ \hline
		\multicolumn{1}{|l|}{}                                                                                                & \multicolumn{1}{l|}{Sistema recibe la solicitud}                                                                            \\ \hline
		\multicolumn{1}{|l|}{}                                                                                                & \multicolumn{1}{l|}{\begin{tabular}[c]{@{}l@{}}Sistema consulta a AP para\\ solicitud de informaci�n\end{tabular}}          \\ \hline
		\multicolumn{1}{|l|}{AP recibe la solicitud y la procesa}                                                             & \multicolumn{1}{l|}{}                                                                                                       \\ \hline
		\multicolumn{1}{|l|}{}                                                                                                & \multicolumn{1}{l|}{Sistema recibe respuesta AP}                                                                            \\ \hline
		\multicolumn{1}{|l|}{}                                                                                                & \multicolumn{1}{l|}{Sistema procesa respuesta}                                                                              \\ \hline
		\multicolumn{1}{|l|}{}                                                                                                & \multicolumn{1}{l|}{\begin{tabular}[c]{@{}l@{}}Sistema decodifica respuesta en\\ formato JSON\end{tabular}}                 \\ \hline
		\multicolumn{1}{|l|}{}                                                                                                & \multicolumn{1}{l|}{\begin{tabular}[c]{@{}l@{}}Sistema genera tabla resumen de\\ estad�sticas recibidas\end{tabular}}       \\ \hline
		\multicolumn{1}{|l|}{}                                                                                                & \multicolumn{1}{l|}{\begin{tabular}[c]{@{}l@{}}Sistema genera gr�fico de reportes\\ con el uso\end{tabular}}                \\ \hline
		\multicolumn{1}{|l|}{\textbf{Referencias Cruzadas}}                                                                   & \multicolumn{1}{l|}{CU12, CU25, CU26}                                                                                       \\ \hline
		\multicolumn{1}{|l|}{\textbf{Cursos alternativos}}                                                                    & \multicolumn{1}{l|}{--}                                                                                                     \\ \hline
		&                                                                                                                             \\ \hline
		\multicolumn{2}{|c|}{\textit{\textbf{Caso de Uso: CU29}}}                                                                                                                                                                                           \\ \hline
		\multicolumn{1}{|l|}{\textbf{Nombre}}                                                                                 & \multicolumn{1}{l|}{Visualizar estados de trabajos}                                                                         \\ \hline
		\multicolumn{1}{|l|}{\textbf{Actores}}                                                                                & \multicolumn{1}{l|}{AP, U}                                                                                                  \\ \hline
		\multicolumn{1}{|l|}{\textbf{Funciones Asociadas}}                                                                    & \multicolumn{1}{l|}{}                                                                                                       \\ \hline
		\multicolumn{1}{|l|}{\textbf{Objetivo}}                                                                               & \multicolumn{1}{l|}{Visualizar estados de trabajos}                                                                         \\ \hline
		\multicolumn{1}{|l|}{\textbf{Pre Condiciones}}                                                                        & \multicolumn{1}{l|}{\begin{tabular}[c]{@{}l@{}}Solicitud de estados de trabajos en\\ servidor generada\end{tabular}}        \\ \hline
		\multicolumn{1}{|l|}{\textbf{Post Condiciones}}                                                                       & \multicolumn{1}{l|}{\begin{tabular}[c]{@{}l@{}}Solicitud de estados de trabajos en\\ servidor respondida\end{tabular}}      \\ \hline
		\multicolumn{1}{|c|}{\textit{\textbf{Acciones del Actor}}}                                                            & \multicolumn{1}{c|}{\textit{\textbf{Acciones del Sistema}}}                                                                 \\ \hline
		\multicolumn{1}{|l|}{\begin{tabular}[c]{@{}l@{}}Usuario accede a secci�n de\\ estados de trabajos\end{tabular}}       & \multicolumn{1}{l|}{}                                                                                                       \\ \hline
		\multicolumn{1}{|l|}{}                                                                                                & \multicolumn{1}{l|}{Sistema recibe la solicitud}                                                                            \\ \hline
		\multicolumn{1}{|l|}{}                                                                                                & \multicolumn{1}{l|}{\begin{tabular}[c]{@{}l@{}}Sistema consulta a AP para solicitud\\ de informaci�n\end{tabular}}          \\ \hline
		\multicolumn{1}{|l|}{AP recibe la solicitud y la procesa}                                                             & \multicolumn{1}{l|}{}                                                                                                       \\ \hline
		\multicolumn{1}{|l|}{}                                                                                                & \multicolumn{1}{l|}{Sistema recibe respuesta AP}                                                                            \\ \hline
		\multicolumn{1}{|l|}{}                                                                                                & \multicolumn{1}{l|}{Sistema procesa respuesta}                                                                              \\ \hline
		\multicolumn{1}{|l|}{}                                                                                                & \multicolumn{1}{l|}{\begin{tabular}[c]{@{}l@{}}Sistema decodifica respuesta en\\ formato JSON\end{tabular}}                 \\ \hline
		\multicolumn{1}{|l|}{}                                                                                                & \multicolumn{1}{l|}{\begin{tabular}[c]{@{}l@{}}Sistema genera tabla resumen de\\ estad�sticas recibidas\end{tabular}}       \\ \hline
		\multicolumn{1}{|l|}{}                                                                                                & \multicolumn{1}{l|}{\begin{tabular}[c]{@{}l@{}}Sistema genera gr�fico de reportes\\ con el uso\end{tabular}}                \\ \hline
		\multicolumn{1}{|l|}{\textbf{Referencias Cruzadas}}                                                                   & \multicolumn{1}{l|}{CU12, CU25, CU26}                                                                                       \\ \hline
		\multicolumn{1}{|l|}{\textbf{Cursos alternativos}}                                                                    & \multicolumn{1}{l|}{--}                                                                                                     \\ \hline
		&                                                                                                                             \\ \hline
		\multicolumn{2}{|c|}{\textit{\textbf{Caso de Uso: CU30}}}                                                                                                                                                                                           \\ \hline
		\multicolumn{1}{|l|}{\textbf{Nombre}}                                                                                 & \multicolumn{1}{l|}{Visualizar estad�sticas de trabajos}                                                                    \\ \hline
		\multicolumn{1}{|l|}{\textbf{Actores}}                                                                                & \multicolumn{1}{l|}{AP, U}                                                                                                  \\ \hline
		\multicolumn{1}{|l|}{\textbf{Funciones Asociadas}}                                                                    & \multicolumn{1}{l|}{}                                                                                                       \\ \hline
		\multicolumn{1}{|l|}{\textbf{Objetivo}}                                                                               & \multicolumn{1}{l|}{Visualizar estad�sticas de trabajos}                                                                    \\ \hline
		\multicolumn{1}{|l|}{\textbf{Pre Condiciones}}                                                                        & \multicolumn{1}{l|}{\begin{tabular}[c]{@{}l@{}}Solicitud de estad�sticas de trabajos\\ en servidor generada\end{tabular}}   \\ \hline
		\multicolumn{1}{|l|}{\textbf{Post Condiciones}}                                                                       & \multicolumn{1}{l|}{\begin{tabular}[c]{@{}l@{}}Solicitud de estad�sticas de trabajos\\ en servidor respondida\end{tabular}} \\ \hline
		\multicolumn{1}{|c|}{\textit{\textbf{Acciones del Actor}}}                                                            & \multicolumn{1}{c|}{\textit{\textbf{Acciones del Sistema}}}                                                                 \\ \hline
		\multicolumn{1}{|l|}{\begin{tabular}[c]{@{}l@{}}Usuario accede a secci�n de estad�sticas\\ de trabajos\end{tabular}}  & \multicolumn{1}{l|}{}                                                                                                       \\ \hline
		\multicolumn{1}{|l|}{}                                                                                                & \multicolumn{1}{l|}{Sistema recibe la solicitud}                                                                            \\ \hline
		\multicolumn{1}{|l|}{}                                                                                                & \multicolumn{1}{l|}{\begin{tabular}[c]{@{}l@{}}Sistema consulta a AP para\\ solicitud de informaci�n\end{tabular}}          \\ \hline
		\multicolumn{1}{|l|}{AP recibe la solicitud y la procesa}                                                             & \multicolumn{1}{l|}{}                                                                                                       \\ \hline
		\multicolumn{1}{|l|}{}                                                                                                & \multicolumn{1}{l|}{Sistema recibe respuesta AP}                                                                            \\ \hline
		\multicolumn{1}{|l|}{}                                                                                                & \multicolumn{1}{l|}{Sistema procesa respuesta}                                                                              \\ \hline
		\multicolumn{1}{|l|}{}                                                                                                & \multicolumn{1}{l|}{\begin{tabular}[c]{@{}l@{}}Sistema decodifica respuesta en\\ formato JSON\end{tabular}}                 \\ \hline
		\multicolumn{1}{|l|}{}                                                                                                & \multicolumn{1}{l|}{\begin{tabular}[c]{@{}l@{}}Sistema genera tabla resumen de\\ estad�sticas recibidas\end{tabular}}       \\ \hline
		\multicolumn{1}{|l|}{}                                                                                                & \multicolumn{1}{l|}{\begin{tabular}[c]{@{}l@{}}Sistema genera gr�fico de reportes\\ con el uso\end{tabular}}                \\ \hline
		\multicolumn{1}{|l|}{\textbf{Referencias Cruzadas}}                                                                   & \multicolumn{1}{l|}{CU12, CU25, CU26}                                                                                       \\ \hline
		\multicolumn{1}{|l|}{\textbf{Cursos alternativos}}                                                                    & \multicolumn{1}{l|}{--}                                                                                                     \\ \hline
		\caption{Casos de uso asociados al m�dulo estad�stico}
		\label{CU06}\\
		\end{longtable}
		
	\subsection{Casos de uso asociados a la generaci�n de reportes}
	
	Estos casos de uso se encuentran relacionados al manejo de reportes y la generaci�n de tablas res�menes y gr�ficos de inter�s.
	
	% Please add the following required packages to your document preamble:
	% \usepackage{longtable}
	% Note: It may be necessary to compile the document several times to get a multi-page table to line up properly
	\begin{longtable}{ll}
		\hline
		\multicolumn{2}{|c|}{\textit{\textbf{Caso de Uso: CU25}}}                                                                                                                                                                                       \\ \hline
		\endfirsthead
		%
		\endhead
		%
		\multicolumn{1}{|l|}{\textbf{Nombre}}                                                                            & \multicolumn{1}{l|}{Generar tabla resumen de estad�sticos}                                                                   \\ \hline
		\multicolumn{1}{|l|}{\textbf{Actores}}                                                                           & \multicolumn{1}{l|}{AP}                                                                                                      \\ \hline
		\multicolumn{1}{|l|}{\textbf{Funciones Asociadas}}                                                               & \multicolumn{1}{l|}{}                                                                                                        \\ \hline
		\multicolumn{1}{|l|}{\textbf{Objetivo}}                                                                          & \multicolumn{1}{l|}{\begin{tabular}[c]{@{}l@{}}Generar tabla resumen responsive\\ y con opciones de descarga\end{tabular}}   \\ \hline
		\multicolumn{1}{|l|}{\textbf{Pre Condiciones}}                                                                   & \multicolumn{1}{l|}{Tabla resumen sin datos disponibles}                                                                     \\ \hline
		\multicolumn{1}{|l|}{\textbf{Post Condiciones}}                                                                  & \multicolumn{1}{l|}{Tabla resumen generada correcta}                                                                         \\ \hline
		\multicolumn{1}{|c|}{\textit{\textbf{Acciones del Actor}}}                                                       & \multicolumn{1}{c|}{\textit{\textbf{Acciones del Sistema}}}                                                                  \\ \hline
		\multicolumn{1}{|l|}{\begin{tabular}[c]{@{}l@{}}AP responde la solicitud de\\ estad�sticas resumen\end{tabular}} & \multicolumn{1}{l|}{}                                                                                                        \\ \hline
		\multicolumn{1}{|l|}{}                                                                                           & \multicolumn{1}{l|}{\begin{tabular}[c]{@{}l@{}}Sistema recibe la respuesta y procesa\\ el formato JSON\end{tabular}}         \\ \hline
		\multicolumn{1}{|l|}{}                                                                                           & \multicolumn{1}{l|}{\begin{tabular}[c]{@{}l@{}}Sistema carga la data en estructura\\ dataTable\end{tabular}}                 \\ \hline
		\multicolumn{1}{|l|}{}                                                                                           & \multicolumn{1}{l|}{\begin{tabular}[c]{@{}l@{}}Sistema habilita los botones de acci�n\\ en el datatable\end{tabular}}        \\ \hline
		\multicolumn{1}{|l|}{}                                                                                           & \multicolumn{1}{l|}{\begin{tabular}[c]{@{}l@{}}Sistema carga la tabla en el div\\ correspondiente\end{tabular}}              \\ \hline
		\multicolumn{1}{|l|}{\textbf{Referencias Cruzadas}}                                                              & \multicolumn{1}{l|}{--}                                                                                                      \\ \hline
		\multicolumn{1}{|l|}{\textbf{Cursos alternativos}}                                                               & \multicolumn{1}{l|}{--}                                                                                                      \\ \hline
		&                                                                                                                              \\ \hline
		\multicolumn{2}{|c|}{\textit{\textbf{Caso de Uso: CU26}}}                                                                                                                                                                                       \\ \hline
		\multicolumn{1}{|l|}{\textbf{Nombre}}                                                                            & \multicolumn{1}{l|}{Generar gr�fico resumen de estad�sticos}                                                                 \\ \hline
		\multicolumn{1}{|l|}{\textbf{Actores}}                                                                           & \multicolumn{1}{l|}{AP}                                                                                                      \\ \hline
		\multicolumn{1}{|l|}{\textbf{Funciones Asociadas}}                                                               & \multicolumn{1}{l|}{}                                                                                                        \\ \hline
		\multicolumn{1}{|l|}{\textbf{Objetivo}}                                                                          & \multicolumn{1}{l|}{\begin{tabular}[c]{@{}l@{}}Generar gr�fico resumen responsive y\\ con opciones de descarga\end{tabular}} \\ \hline
		\multicolumn{1}{|l|}{\textbf{Pre Condiciones}}                                                                   & \multicolumn{1}{l|}{Gr�fico resumen sin datos disponibles}                                                                   \\ \hline
		\multicolumn{1}{|l|}{\textbf{Post Condiciones}}                                                                  & \multicolumn{1}{l|}{Gr�fico resumen generado correctamente}                                                                  \\ \hline
		\multicolumn{1}{|c|}{\textit{\textbf{Acciones del Actor}}}                                                       & \multicolumn{1}{c|}{\textit{\textbf{Acciones del Sistema}}}                                                                  \\ \hline
		\multicolumn{1}{|l|}{\begin{tabular}[c]{@{}l@{}}AP responde la solicitud de\\ estad�sticas resumen\end{tabular}} & \multicolumn{1}{l|}{}                                                                                                        \\ \hline
		\multicolumn{1}{|l|}{}                                                                                           & \multicolumn{1}{l|}{\begin{tabular}[c]{@{}l@{}}Sistema recibe la respuesta y procesa el\\ formato JSON\end{tabular}}         \\ \hline
		\multicolumn{1}{|l|}{}                                                                                           & \multicolumn{1}{l|}{Sistema carga la data en estructura JSON}                                                                \\ \hline
		\multicolumn{1}{|l|}{}                                                                                           & \multicolumn{1}{l|}{\begin{tabular}[c]{@{}l@{}}Sistema habilita los botones de acci�n en el\\ Higcharts plugin\end{tabular}} \\ \hline
		\multicolumn{1}{|l|}{}                                                                                           & \multicolumn{1}{l|}{Sistema carga el gr�fico en el div correspondiente}                                                      \\ \hline
		\multicolumn{1}{|l|}{\textbf{Referencias Cruzadas}}                                                              & \multicolumn{1}{l|}{--}                                                                                                      \\ \hline
		\multicolumn{1}{|l|}{\textbf{Cursos alternativos}}                                                               & \multicolumn{1}{l|}{--}                                                                                                      \\ \hline
		\caption{Casos de uso asociados a reportes}
		\label{CU07}\\
		\end{longtable}
		
		\subsection{Casos de uso relacionados al m�dulo de an�lisis estad�stico}
		
		Estos casos de uso cumplen con los requerimientos asociados al an�lisis estad�stico de los set de datos, estos contemplan desarrollo de histogramas, matrices de heat map, estad�sticos res�menes, box plot, entre los principales.
		
		% Please add the following required packages to your document preamble:
		% \usepackage{longtable}
		% Note: It may be necessary to compile the document several times to get a multi-page table to line up properly
		\begin{longtable}{ll}
			\hline
			\multicolumn{2}{|c|}{\textit{\textbf{Caso de Uso: CU31}}}                                                                                                                                                                                                                             \\ \hline
			\endfirsthead
			%
			\endhead
			%
			\multicolumn{1}{|l|}{\textbf{Nombre}}                                                                                           & \multicolumn{1}{l|}{\begin{tabular}[c]{@{}l@{}}Estimar estad�sticos en relaci�n\\ a la muestra\end{tabular}}                                        \\ \hline
			\multicolumn{1}{|l|}{\textbf{Actores}}                                                                                          & \multicolumn{1}{l|}{AP, U}                                                                                                                          \\ \hline
			\multicolumn{1}{|l|}{\textbf{Funciones Asociadas}}                                                                              & \multicolumn{1}{l|}{}                                                                                                                               \\ \hline
			\multicolumn{1}{|l|}{\textbf{Objetivo}}                                                                                         & \multicolumn{1}{l|}{\begin{tabular}[c]{@{}l@{}}Estimar estad�sticos en relaci�n\\ a la muestra\end{tabular}}                                        \\ \hline
			\multicolumn{1}{|l|}{\textbf{Pre Condiciones}}                                                                                  & \multicolumn{1}{l|}{\begin{tabular}[c]{@{}l@{}}Solicitud de estad�sticos para set de datos de\\ inter�s generada\end{tabular}}                      \\ \hline
			\multicolumn{1}{|l|}{\textbf{Post Condiciones}}                                                                                 & \multicolumn{1}{l|}{\begin{tabular}[c]{@{}l@{}}Solicitud de estad�sticos para set de datos de\\ inter�s respondida\end{tabular}}                    \\ \hline
			\multicolumn{1}{|c|}{\textit{\textbf{Acciones del Actor}}}                                                                      & \multicolumn{1}{c|}{\textit{\textbf{Acciones del Sistema}}}                                                                                         \\ \hline
			\multicolumn{1}{|l|}{\begin{tabular}[c]{@{}l@{}}Usuario accede a la secci�n estad�sticas\\ en set de datos\end{tabular}}        & \multicolumn{1}{l|}{}                                                                                                                               \\ \hline
			\multicolumn{1}{|l|}{\begin{tabular}[c]{@{}l@{}}Usuario solicita las estad�sticas\\ res�menes\end{tabular}}                     & \multicolumn{1}{l|}{}                                                                                                                               \\ \hline
			\multicolumn{1}{|l|}{}                                                                                                          & \multicolumn{1}{l|}{Sistema recibe solicitud}                                                                                                       \\ \hline
			\multicolumn{1}{|l|}{}                                                                                                          & \multicolumn{1}{l|}{Sistema procesa la matriz de informaci�n}                                                                                       \\ \hline
			\multicolumn{1}{|l|}{}                                                                                                          & \multicolumn{1}{l|}{\begin{tabular}[c]{@{}l@{}}Sistema estima promedio, varianza, desviaci�n,\\ m�ximos y m�nimos en la muestra\end{tabular}}       \\ \hline
			\multicolumn{1}{|l|}{}                                                                                                          & \multicolumn{1}{l|}{Sistema genera respuesta en formato JSON}                                                                                       \\ \hline
			\multicolumn{1}{|l|}{}                                                                                                          & \multicolumn{1}{l|}{\begin{tabular}[c]{@{}l@{}}Sistema recibe respuesta y genera tabla\\ resumen\end{tabular}}                                      \\ \hline
			\multicolumn{1}{|l|}{}                                                                                                          & \multicolumn{1}{l|}{\begin{tabular}[c]{@{}l@{}}Sistema notifica proceso generado de manera\\ correcta\end{tabular}}                                 \\ \hline
			\multicolumn{1}{|l|}{}                                                                                                          & \multicolumn{1}{l|}{\begin{tabular}[c]{@{}l@{}}Sistema despliega cuadro resumen con los\\ estad�sticos generados\end{tabular}}                      \\ \hline
			\multicolumn{1}{|l|}{\textbf{Referencias Cruzadas}}                                                                             & \multicolumn{1}{l|}{CU12, CU25}                                                                                                                     \\ \hline
			\multicolumn{1}{|l|}{\textbf{Cursos alternativos}}                                                                              & \multicolumn{1}{l|}{--}                                                                                                                             \\ \hline
			&                                                                                                                                                     \\ \hline
			\multicolumn{2}{|c|}{\textit{\textbf{Caso de Uso: CU32}}}                                                                                                                                                                                                                             \\ \hline
			\multicolumn{1}{|l|}{\textbf{Nombre}}                                                                                           & \multicolumn{1}{l|}{Estimar box plot}                                                                                                               \\ \hline
			\multicolumn{1}{|l|}{\textbf{Actores}}                                                                                          & \multicolumn{1}{l|}{AP, U}                                                                                                                          \\ \hline
			\multicolumn{1}{|l|}{\textbf{Funciones Asociadas}}                                                                              & \multicolumn{1}{l|}{}                                                                                                                               \\ \hline
			\multicolumn{1}{|l|}{\textbf{Objetivo}}                                                                                         & \multicolumn{1}{l|}{Estimar box plot en relaci�n a la muestra}                                                                                      \\ \hline
			\multicolumn{1}{|l|}{\textbf{Pre Condiciones}}                                                                                  & \multicolumn{1}{l|}{\begin{tabular}[c]{@{}l@{}}Solicitud de box plot para set de datos de\\ inter�s generada\end{tabular}}                          \\ \hline
			\multicolumn{1}{|l|}{\textbf{Post Condiciones}}                                                                                 & \multicolumn{1}{l|}{\begin{tabular}[c]{@{}l@{}}Solicitud de box plot para set de datos de\\ inter�s respondida\end{tabular}}                        \\ \hline
			\multicolumn{1}{|c|}{\textit{\textbf{Acciones del Actor}}}                                                                      & \multicolumn{1}{c|}{\textit{\textbf{Acciones del Sistema}}}                                                                                         \\ \hline
			\multicolumn{1}{|l|}{\begin{tabular}[c]{@{}l@{}}Usuario accede a la secci�n estad�sticas\\ en set de datos\end{tabular}}        & \multicolumn{1}{l|}{}                                                                                                                               \\ \hline
			\multicolumn{1}{|l|}{Usuario solicita el box plot de la muestra}                                                                & \multicolumn{1}{l|}{}                                                                                                                               \\ \hline
			\multicolumn{1}{|l|}{}                                                                                                          & \multicolumn{1}{l|}{Sistema recibe solicitud}                                                                                                       \\ \hline
			\multicolumn{1}{|l|}{}                                                                                                          & \multicolumn{1}{l|}{Sistema procesa la matriz de informaci�n}                                                                                       \\ \hline
			\multicolumn{1}{|l|}{}                                                                                                          & \multicolumn{1}{l|}{\begin{tabular}[c]{@{}l@{}}Sistema estima el box plot para todos los\\ atributos en el set de datos\end{tabular}}               \\ \hline
			\multicolumn{1}{|l|}{}                                                                                                          & \multicolumn{1}{l|}{Sistema genera respuesta en formato JSON}                                                                                       \\ \hline
			\multicolumn{1}{|l|}{}                                                                                                          & \multicolumn{1}{l|}{\begin{tabular}[c]{@{}l@{}}Sistema recibe respuesta y genera gr�fico\\ resumen\end{tabular}}                                    \\ \hline
			\multicolumn{1}{|l|}{}                                                                                                          & \multicolumn{1}{l|}{\begin{tabular}[c]{@{}l@{}}Sistema notifica proceso generado de manera\\ correcta\end{tabular}}                                 \\ \hline
			\multicolumn{1}{|l|}{}                                                                                                          & \multicolumn{1}{l|}{\begin{tabular}[c]{@{}l@{}}Sistema despliega gr�fico resumen con el\\ box plot generado\end{tabular}}                           \\ \hline
			\multicolumn{1}{|l|}{\textbf{Referencias Cruzadas}}                                                                             & \multicolumn{1}{l|}{CU12, CU26}                                                                                                                     \\ \hline
			\multicolumn{1}{|l|}{\textbf{Cursos alternativos}}                                                                              & \multicolumn{1}{l|}{--}                                                                                                                             \\ \hline
			&                                                                                                                                                     \\ \hline
			\multicolumn{2}{|c|}{\textit{\textbf{Caso de Uso: CU33}}}                                                                                                                                                                                                                             \\ \hline
			\multicolumn{1}{|l|}{\textbf{Nombre}}                                                                                           & \multicolumn{1}{l|}{Estimar Histograma}                                                                                                             \\ \hline
			\multicolumn{1}{|l|}{\textbf{Actores}}                                                                                          & \multicolumn{1}{l|}{AP, U}                                                                                                                          \\ \hline
			\multicolumn{1}{|l|}{\textbf{Funciones Asociadas}}                                                                              & \multicolumn{1}{l|}{}                                                                                                                               \\ \hline
			\multicolumn{1}{|l|}{\textbf{Objetivo}}                                                                                         & \multicolumn{1}{l|}{\begin{tabular}[c]{@{}l@{}}Estimar histogramas en relaci�n\\ a la muestra\end{tabular}}                                         \\ \hline
			\multicolumn{1}{|l|}{\textbf{Pre Condiciones}}                                                                                  & \multicolumn{1}{l|}{\begin{tabular}[c]{@{}l@{}}Solicitud de histogramas para set de\\ datos de inter�s generada\end{tabular}}                       \\ \hline
			\multicolumn{1}{|l|}{\textbf{Post Condiciones}}                                                                                 & \multicolumn{1}{l|}{\begin{tabular}[c]{@{}l@{}}Solicitud de histogramas para set de\\ datos de inter�s respondida\end{tabular}}                     \\ \hline
			\multicolumn{1}{|c|}{\textit{\textbf{Acciones del Actor}}}                                                                      & \multicolumn{1}{c|}{\textit{\textbf{Acciones del Sistema}}}                                                                                         \\ \hline
			\multicolumn{1}{|l|}{\begin{tabular}[c]{@{}l@{}}Usuario accede a la secci�n estad�sticas\\ en set de datos\end{tabular}}        & \multicolumn{1}{l|}{}                                                                                                                               \\ \hline
			\multicolumn{1}{|l|}{\begin{tabular}[c]{@{}l@{}}Usuario solicita histogramas de la\\ muestra\end{tabular}}                      & \multicolumn{1}{l|}{}                                                                                                                               \\ \hline
			\multicolumn{1}{|l|}{}                                                                                                          & \multicolumn{1}{l|}{Sistema recibe solicitud}                                                                                                       \\ \hline
			\multicolumn{1}{|l|}{}                                                                                                          & \multicolumn{1}{l|}{Sistema procesa la matriz de informaci�n}                                                                                       \\ \hline
			\multicolumn{1}{|l|}{}                                                                                                          & \multicolumn{1}{l|}{\begin{tabular}[c]{@{}l@{}}Sistema estima los histogramas para todos los\\ atributos en el set de datos\end{tabular}}           \\ \hline
			\multicolumn{1}{|l|}{}                                                                                                          & \multicolumn{1}{l|}{Sistema genera respuesta en formato JSON}                                                                                       \\ \hline
			\multicolumn{1}{|l|}{}                                                                                                          & \multicolumn{1}{l|}{\begin{tabular}[c]{@{}l@{}}Sistema recibe respuesta y genera gr�fico\\ resumen\end{tabular}}                                    \\ \hline
			\multicolumn{1}{|l|}{}                                                                                                          & \multicolumn{1}{l|}{\begin{tabular}[c]{@{}l@{}}Sistema notifica proceso generado de manera\\ correcta\end{tabular}}                                 \\ \hline
			\multicolumn{1}{|l|}{}                                                                                                          & \multicolumn{1}{l|}{\begin{tabular}[c]{@{}l@{}}Sistema despliega gr�fico resumen con el\\ histograma generado\end{tabular}}                         \\ \hline
			\multicolumn{1}{|l|}{\textbf{Referencias Cruzadas}}                                                                             & \multicolumn{1}{l|}{CU12, CU26}                                                                                                                     \\ \hline
			\multicolumn{1}{|l|}{\textbf{Cursos alternativos}}                                                                              & \multicolumn{1}{l|}{--}                                                                                                                             \\ \hline
			&                                                                                                                                                     \\ \hline
			\multicolumn{2}{|c|}{\textit{\textbf{Caso de Uso: CU34}}}                                                                                                                                                                                                                             \\ \hline
			\multicolumn{1}{|l|}{\textbf{Nombre}}                                                                                           & \multicolumn{1}{l|}{Estimar Barcharts}                                                                                                              \\ \hline
			\multicolumn{1}{|l|}{\textbf{Actores}}                                                                                          & \multicolumn{1}{l|}{AP, U}                                                                                                                          \\ \hline
			\multicolumn{1}{|l|}{\textbf{Funciones Asociadas}}                                                                              & \multicolumn{1}{l|}{}                                                                                                                               \\ \hline
			\multicolumn{1}{|l|}{\textbf{Objetivo}}                                                                                         & \multicolumn{1}{l|}{Estimar barcharts en relaci�n a la muestra}                                                                                     \\ \hline
			\multicolumn{1}{|l|}{\textbf{Pre Condiciones}}                                                                                  & \multicolumn{1}{l|}{\begin{tabular}[c]{@{}l@{}}Solicitud de barcharts para set de datos de\\ inter�s generada\end{tabular}}                         \\ \hline
			\multicolumn{1}{|l|}{\textbf{Post Condiciones}}                                                                                 & \multicolumn{1}{l|}{\begin{tabular}[c]{@{}l@{}}Solicitud de barcharts para set de datos de\\ inter�s respondida\end{tabular}}                       \\ \hline
			\multicolumn{1}{|c|}{\textit{\textbf{Acciones del Actor}}}                                                                      & \multicolumn{1}{c|}{\textit{\textbf{Acciones del Sistema}}}                                                                                         \\ \hline
			\multicolumn{1}{|l|}{\begin{tabular}[c]{@{}l@{}}Usuario accede a la secci�n estad�sticas\\ en set de datos\end{tabular}}        & \multicolumn{1}{l|}{}                                                                                                                               \\ \hline
			\multicolumn{1}{|l|}{\begin{tabular}[c]{@{}l@{}}Usuario solicita barcharts de la\\ muestra\end{tabular}}                        & \multicolumn{1}{l|}{}                                                                                                                               \\ \hline
			\multicolumn{1}{|l|}{}                                                                                                          & \multicolumn{1}{l|}{Sistema recibe solicitud}                                                                                                       \\ \hline
			\multicolumn{1}{|l|}{}                                                                                                          & \multicolumn{1}{l|}{Sistema procesa la matriz de informaci�n}                                                                                       \\ \hline
			\multicolumn{1}{|l|}{}                                                                                                          & \multicolumn{1}{l|}{\begin{tabular}[c]{@{}l@{}}Sistema estima los barcharts para todos los\\ atributos en el set de datos\end{tabular}}             \\ \hline
			\multicolumn{1}{|l|}{}                                                                                                          & \multicolumn{1}{l|}{Sistema genera respuesta en formato JSON}                                                                                       \\ \hline
			\multicolumn{1}{|l|}{}                                                                                                          & \multicolumn{1}{l|}{\begin{tabular}[c]{@{}l@{}}Sistema recibe respuesta y genera gr�fico\\ resumen\end{tabular}}                                    \\ \hline
			\multicolumn{1}{|l|}{}                                                                                                          & \multicolumn{1}{l|}{\begin{tabular}[c]{@{}l@{}}Sistema notifica proceso generado de\\ manera correcta\end{tabular}}                                 \\ \hline
			\multicolumn{1}{|l|}{}                                                                                                          & \multicolumn{1}{l|}{\begin{tabular}[c]{@{}l@{}}Sistema despliega gr�fico resumen con el\\ barcharts generado\end{tabular}}                          \\ \hline
			\multicolumn{1}{|l|}{\textbf{Referencias Cruzadas}}                                                                             & \multicolumn{1}{l|}{CU12, CU26}                                                                                                                     \\ \hline
			\multicolumn{1}{|l|}{\textbf{Cursos alternativos}}                                                                              & \multicolumn{1}{l|}{--}                                                                                                                             \\ \hline
			&                                                                                                                                                     \\ \hline
			\multicolumn{2}{|c|}{\textit{\textbf{Caso de Uso: CU35}}}                                                                                                                                                                                                                             \\ \hline
			\multicolumn{1}{|l|}{\textbf{Nombre}}                                                                                           & \multicolumn{1}{l|}{Estimar Piecharts}                                                                                                              \\ \hline
			\multicolumn{1}{|l|}{\textbf{Actores}}                                                                                          & \multicolumn{1}{l|}{AP, U}                                                                                                                          \\ \hline
			\multicolumn{1}{|l|}{\textbf{Funciones Asociadas}}                                                                              & \multicolumn{1}{l|}{}                                                                                                                               \\ \hline
			\multicolumn{1}{|l|}{\textbf{Objetivo}}                                                                                         & \multicolumn{1}{l|}{Estimar piecharts en relaci�n a la muestra}                                                                                     \\ \hline
			\multicolumn{1}{|l|}{\textbf{Pre Condiciones}}                                                                                  & \multicolumn{1}{l|}{\begin{tabular}[c]{@{}l@{}}Solicitud de piecharts para set de datos de\\ inter�s generada\end{tabular}}                         \\ \hline
			\multicolumn{1}{|l|}{\textbf{Post Condiciones}}                                                                                 & \multicolumn{1}{l|}{\begin{tabular}[c]{@{}l@{}}Solicitud de piecharts para set de datos de\\ inter�s respondida\end{tabular}}                       \\ \hline
			\multicolumn{1}{|c|}{\textit{\textbf{Acciones del Actor}}}                                                                      & \multicolumn{1}{c|}{\textit{\textbf{Acciones del Sistema}}}                                                                                         \\ \hline
			\multicolumn{1}{|l|}{\begin{tabular}[c]{@{}l@{}}Usuario accede a la secci�n estad�sticas\\ en set de datos\end{tabular}}        & \multicolumn{1}{l|}{}                                                                                                                               \\ \hline
			\multicolumn{1}{|l|}{\begin{tabular}[c]{@{}l@{}}Usuario solicita piecharts de la\\ muestra\end{tabular}}                        & \multicolumn{1}{l|}{}                                                                                                                               \\ \hline
			\multicolumn{1}{|l|}{}                                                                                                          & \multicolumn{1}{l|}{Sistema recibe solicitud}                                                                                                       \\ \hline
			\multicolumn{1}{|l|}{}                                                                                                          & \multicolumn{1}{l|}{Sistema procesa la matriz de informaci�n}                                                                                       \\ \hline
			\multicolumn{1}{|l|}{}                                                                                                          & \multicolumn{1}{l|}{\begin{tabular}[c]{@{}l@{}}Sistema estima los piecharts para todos los\\ atributos en el set de datos\end{tabular}}             \\ \hline
			\multicolumn{1}{|l|}{}                                                                                                          & \multicolumn{1}{l|}{Sistema genera respuesta en formato JSON}                                                                                       \\ \hline
			\multicolumn{1}{|l|}{}                                                                                                          & \multicolumn{1}{l|}{\begin{tabular}[c]{@{}l@{}}Sistema recibe respuesta y genera gr�fico\\ resumen\end{tabular}}                                    \\ \hline
			\multicolumn{1}{|l|}{}                                                                                                          & \multicolumn{1}{l|}{\begin{tabular}[c]{@{}l@{}}Sistema notifica proceso generado de\\ manera correcta\end{tabular}}                                 \\ \hline
			\multicolumn{1}{|l|}{}                                                                                                          & \multicolumn{1}{l|}{\begin{tabular}[c]{@{}l@{}}Sistema despliega gr�fico resumen con\\ el piecharts generado\end{tabular}}                          \\ \hline
			\multicolumn{1}{|l|}{\textbf{Referencias Cruzadas}}                                                                             & \multicolumn{1}{l|}{CU12, CU26}                                                                                                                     \\ \hline
			\multicolumn{1}{|l|}{\textbf{Cursos alternativos}}                                                                              & \multicolumn{1}{l|}{--}                                                                                                                             \\ \hline
			&                                                                                                                                                     \\ \hline
			\multicolumn{2}{|c|}{\textit{\textbf{Caso de Uso: CU36}}}                                                                                                                                                                                                                             \\ \hline
			\multicolumn{1}{|l|}{\textbf{Nombre}}                                                                                           & \multicolumn{1}{l|}{Estimar Matriz de correlaci�n de los atributos}                                                                                 \\ \hline
			\multicolumn{1}{|l|}{\textbf{Actores}}                                                                                          & \multicolumn{1}{l|}{AP, U}                                                                                                                          \\ \hline
			\multicolumn{1}{|l|}{\textbf{Funciones Asociadas}}                                                                              & \multicolumn{1}{l|}{}                                                                                                                               \\ \hline
			\multicolumn{1}{|l|}{\textbf{Objetivo}}                                                                                         & \multicolumn{1}{l|}{\begin{tabular}[c]{@{}l@{}}Estimar matriz de correlaci�n en relaci�n\\ a la muestra\end{tabular}}                               \\ \hline
			\multicolumn{1}{|l|}{\textbf{Pre Condiciones}}                                                                                  & \multicolumn{1}{l|}{\begin{tabular}[c]{@{}l@{}}Solicitud de matriz de correlaci�n para set de\\ datos de inter�s generada\end{tabular}}             \\ \hline
			\multicolumn{1}{|l|}{\textbf{Post Condiciones}}                                                                                 & \multicolumn{1}{l|}{\begin{tabular}[c]{@{}l@{}}Solicitud de matriz de correlaci�n para set de\\ datos de inter�s respondida\end{tabular}}           \\ \hline
			\multicolumn{1}{|c|}{\textit{\textbf{Acciones del Actor}}}                                                                      & \multicolumn{1}{c|}{\textit{\textbf{Acciones del Sistema}}}                                                                                         \\ \hline
			\multicolumn{1}{|l|}{\begin{tabular}[c]{@{}l@{}}Usuario accede a la secci�n estad�sticas\\ en set de datos\end{tabular}}        & \multicolumn{1}{l|}{}                                                                                                                               \\ \hline
			\multicolumn{1}{|l|}{\begin{tabular}[c]{@{}l@{}}Usuario solicita matriz de correlaci�n\\ de la muestra\end{tabular}}            & \multicolumn{1}{l|}{}                                                                                                                               \\ \hline
			\multicolumn{1}{|l|}{}                                                                                                          & \multicolumn{1}{l|}{Sistema recibe solicitud}                                                                                                       \\ \hline
			\multicolumn{1}{|l|}{}                                                                                                          & \multicolumn{1}{l|}{Sistema procesa la matriz de informaci�n}                                                                                       \\ \hline
			\multicolumn{1}{|l|}{}                                                                                                          & \multicolumn{1}{l|}{\begin{tabular}[c]{@{}l@{}}Sistema estima los matriz de correlaci�n\\ para todos los atributos en el set de datos\end{tabular}} \\ \hline
			\multicolumn{1}{|l|}{}                                                                                                          & \multicolumn{1}{l|}{Sistema genera respuesta en formato JSON}                                                                                       \\ \hline
			\multicolumn{1}{|l|}{}                                                                                                          & \multicolumn{1}{l|}{\begin{tabular}[c]{@{}l@{}}Sistema recibe respuesta y genera gr�fico\\ resumen\end{tabular}}                                    \\ \hline
			\multicolumn{1}{|l|}{}                                                                                                          & \multicolumn{1}{l|}{\begin{tabular}[c]{@{}l@{}}Sistema notifica proceso generado de manera\\ correcta\end{tabular}}                                 \\ \hline
			\multicolumn{1}{|l|}{}                                                                                                          & \multicolumn{1}{l|}{\begin{tabular}[c]{@{}l@{}}Sistema despliega tabla resumen con el matriz\\ de correlaci�n generado\end{tabular}}                \\ \hline
			\multicolumn{1}{|l|}{\textbf{Referencias Cruzadas}}                                                                             & \multicolumn{1}{l|}{CU12, CU25}                                                                                                                     \\ \hline
			\multicolumn{1}{|l|}{\textbf{Cursos alternativos}}                                                                              & \multicolumn{1}{l|}{--}                                                                                                                             \\ \hline
			&                                                                                                                                                     \\ \hline
			\multicolumn{2}{|c|}{\textit{\textbf{Caso de Uso: CU37}}}                                                                                                                                                                                                                             \\ \hline
			\multicolumn{1}{|l|}{\textbf{Nombre}}                                                                                           & \multicolumn{1}{l|}{\begin{tabular}[c]{@{}l@{}}Estimar Matriz de correlaci�n de los atributos\\ en forma de heat map\end{tabular}}                  \\ \hline
			\multicolumn{1}{|l|}{\textbf{Actores}}                                                                                          & \multicolumn{1}{l|}{AP, U}                                                                                                                          \\ \hline
			\multicolumn{1}{|l|}{\textbf{Funciones Asociadas}}                                                                              & \multicolumn{1}{l|}{}                                                                                                                               \\ \hline
			\multicolumn{1}{|l|}{\textbf{Objetivo}}                                                                                         & \multicolumn{1}{l|}{\begin{tabular}[c]{@{}l@{}}Estimar matriz de correlaci�n en relaci�n a\\ la muestra visualizando heat map\end{tabular}}         \\ \hline
			\multicolumn{1}{|l|}{\textbf{Pre Condiciones}}                                                                                  & \multicolumn{1}{l|}{\begin{tabular}[c]{@{}l@{}}Solicitud de heat map para set de datos de\\ inter�s generada\end{tabular}}                          \\ \hline
			\multicolumn{1}{|l|}{\textbf{Post Condiciones}}                                                                                 & \multicolumn{1}{l|}{\begin{tabular}[c]{@{}l@{}}Solicitud de heat map para set de datos de\\ inter�s respondida\end{tabular}}                        \\ \hline
			\multicolumn{1}{|c|}{\textit{\textbf{Acciones del Actor}}}                                                                      & \multicolumn{1}{c|}{\textit{\textbf{Acciones del Sistema}}}                                                                                         \\ \hline
			\multicolumn{1}{|l|}{\begin{tabular}[c]{@{}l@{}}Usuario accede a la secci�n estad�sticas\\ en set de datos\end{tabular}}        & \multicolumn{1}{l|}{}                                                                                                                               \\ \hline
			\multicolumn{1}{|l|}{\begin{tabular}[c]{@{}l@{}}Usuario solicita visualizar matriz de\\ correlaci�n de la muestra\end{tabular}} & \multicolumn{1}{l|}{}                                                                                                                               \\ \hline
			\multicolumn{1}{|l|}{}                                                                                                          & \multicolumn{1}{l|}{Sistema recibe solicitud}                                                                                                       \\ \hline
			\multicolumn{1}{|l|}{}                                                                                                          & \multicolumn{1}{l|}{Sistema procesa la matriz de informaci�n}                                                                                       \\ \hline
			\multicolumn{1}{|l|}{}                                                                                                          & \multicolumn{1}{l|}{\begin{tabular}[c]{@{}l@{}}Sistema estima los matriz de correlaci�n\\ para todos los atributos en el set de datos\end{tabular}} \\ \hline
			\multicolumn{1}{|l|}{}                                                                                                          & \multicolumn{1}{l|}{Sistema genera respuesta en formato JSON}                                                                                       \\ \hline
			\multicolumn{1}{|l|}{}                                                                                                          & \multicolumn{1}{l|}{\begin{tabular}[c]{@{}l@{}}Sistema recibe respuesta y genera\\ gr�fico resumen\end{tabular}}                                    \\ \hline
			\multicolumn{1}{|l|}{}                                                                                                          & \multicolumn{1}{l|}{\begin{tabular}[c]{@{}l@{}}Sistema notifica proceso generado de\\ manera correcta\end{tabular}}                                 \\ \hline
			\multicolumn{1}{|l|}{}                                                                                                          & \multicolumn{1}{l|}{\begin{tabular}[c]{@{}l@{}}Sistema despliega tabla resumen con el\\ matriz de correlaci�n generado\end{tabular}}                \\ \hline
			\multicolumn{1}{|l|}{\textbf{Referencias Cruzadas}}                                                                             & \multicolumn{1}{l|}{CU12, CU26}                                                                                                                     \\ \hline
			\multicolumn{1}{|l|}{\textbf{Cursos alternativos}}                                                                              & \multicolumn{1}{l|}{--}                                                                                                                             \\ \hline
			&                                                                                                                                                     \\ \hline
			\multicolumn{2}{|c|}{\textit{\textbf{Caso de Uso: CU38}}}                                                                                                                                                                                                                             \\ \hline
			\multicolumn{1}{|l|}{\textbf{Nombre}}                                                                                           & \multicolumn{1}{l|}{Estimar scatter plot de los atributos}                                                                                          \\ \hline
			\multicolumn{1}{|l|}{\textbf{Actores}}                                                                                          & \multicolumn{1}{l|}{AP, U}                                                                                                                          \\ \hline
			\multicolumn{1}{|l|}{\textbf{Funciones Asociadas}}                                                                              & \multicolumn{1}{l|}{}                                                                                                                               \\ \hline
			\multicolumn{1}{|l|}{\textbf{Objetivo}}                                                                                         & \multicolumn{1}{l|}{Estimar scatter plot de los atributos}                                                                                          \\ \hline
			\multicolumn{1}{|l|}{\textbf{Pre Condiciones}}                                                                                  & \multicolumn{1}{l|}{\begin{tabular}[c]{@{}l@{}}Solicitud de scatter plot para set de\\ datos de inter�s generada\end{tabular}}                      \\ \hline
			\multicolumn{1}{|l|}{\textbf{Post Condiciones}}                                                                                 & \multicolumn{1}{l|}{\begin{tabular}[c]{@{}l@{}}Solicitud de scatter plot para set de\\ datos de inter�s respondida\end{tabular}}                    \\ \hline
			\multicolumn{1}{|c|}{\textit{\textbf{Acciones del Actor}}}                                                                      & \multicolumn{1}{c|}{\textit{\textbf{Acciones del Sistema}}}                                                                                         \\ \hline
			\multicolumn{1}{|l|}{\begin{tabular}[c]{@{}l@{}}Usuario accede a la secci�n estad�sticas\\ en set de datos\end{tabular}}        & \multicolumn{1}{l|}{}                                                                                                                               \\ \hline
			\multicolumn{1}{|l|}{Usuario solicita scatter plot de la muestra}                                                               & \multicolumn{1}{l|}{}                                                                                                                               \\ \hline
			\multicolumn{1}{|l|}{}                                                                                                          & \multicolumn{1}{l|}{Sistema recibe solicitud}                                                                                                       \\ \hline
			\multicolumn{1}{|l|}{}                                                                                                          & \multicolumn{1}{l|}{Sistema procesa la matriz de informaci�n}                                                                                       \\ \hline
			\multicolumn{1}{|l|}{}                                                                                                          & \multicolumn{1}{l|}{\begin{tabular}[c]{@{}l@{}}Sistema estima los scatter plot para todos\\ los atributos en el set de datos\end{tabular}}          \\ \hline
			\multicolumn{1}{|l|}{}                                                                                                          & \multicolumn{1}{l|}{Sistema genera respuesta en formato JSON}                                                                                       \\ \hline
			\multicolumn{1}{|l|}{}                                                                                                          & \multicolumn{1}{l|}{Sistema recibe respuesta y genera gr�fico resumen}                                                                              \\ \hline
			\multicolumn{1}{|l|}{}                                                                                                          & \multicolumn{1}{l|}{Sistema notifica proceso generado de manera correcta}                                                                           \\ \hline
			\multicolumn{1}{|l|}{}                                                                                                          & \multicolumn{1}{l|}{\begin{tabular}[c]{@{}l@{}}Sistema despliega tabla resumen con el matriz\\ de correlaci�n generado\end{tabular}}                \\ \hline
			\multicolumn{1}{|l|}{\textbf{Referencias Cruzadas}}                                                                             & \multicolumn{1}{l|}{CU12, CU26}                                                                                                                     \\ \hline
			\multicolumn{1}{|l|}{\textbf{Cursos alternativos}}                                                                              & \multicolumn{1}{l|}{--}                                                                                                                             \\ \hline
			\caption{Casos de uso relacionados al m�dulo de an�lisis estad�stico}
			\label{CU08}\\
			\end{longtable}
			
	\subsection{Casos de uso asociados al m�dulo de revisi�n de los datos de entrada}
	
	Estos casos de uso est�n estrechamente relacionados a la validaci�n de los set de datos de entrada que ingrese el usuario.
	
	% Please add the following required packages to your document preamble:
	% \usepackage{longtable}
	% Note: It may be necessary to compile the document several times to get a multi-page table to line up properly
	\begin{longtable}{|l|l|}
		\hline
		\multicolumn{2}{|c|}{\textit{\textbf{Caso de Uso: CU39}}}                                                                                                                                                        \\ \hline
		\endfirsthead
		%
		\endhead
		%
		\textbf{Nombre}                                                                                 & Revisar correcto estado set de datos                                                                           \\ \hline
		\textbf{Actores}                                                                                & AP, U                                                                                                          \\ \hline
		\textbf{Funciones Asociadas}                                                                    &                                                                                                                \\ \hline
		\textbf{Objetivo}                                                                               & Revisar correcto estado set de datos                                                                           \\ \hline
		\textbf{Pre Condiciones}                                                                        & Set de datos no revisado                                                                                       \\ \hline
		\textbf{Post Condiciones}                                                                       & Set de datos revisado y solicitud respondida                                                                   \\ \hline
		\multicolumn{1}{|c|}{\textit{\textbf{Acciones del Actor}}}                                      & \multicolumn{1}{c|}{\textit{\textbf{Acciones del Sistema}}}                                                    \\ \hline
		\begin{tabular}[c]{@{}l@{}}Usuario sube un set de datos a su \\ \\ �rea de trabajo\end{tabular} &                                                                                                                \\ \hline
		& Sistema recibe set de datos                                                                                    \\ \hline
		& Sistema inicia revisi�n del set de datos                                                                       \\ \hline
		& Sistema revisa elementos nulos                                                                                 \\ \hline
		& Sistema revisa elementos incorrectos                                                                           \\ \hline
		& \begin{tabular}[c]{@{}l@{}}Sistema revisa disconformidades en la\\ matriz entregada\end{tabular}               \\ \hline
		& \begin{tabular}[c]{@{}l@{}}Sistema revisa existencia de atributos\\ con variable discreta\end{tabular}         \\ \hline
		& Sistema notifica el resultado de la revisi�n                                                                   \\ \hline
		& Sistema genera resumen del proceso                                                                             \\ \hline
		& \begin{tabular}[c]{@{}l@{}}Sistema aloja set de datos en �rea de \\ \\ trabajo de usuario\end{tabular}         \\ \hline
		\textbf{Referencias Cruzadas}                                                                   & CU41                                                                                                           \\ \hline
		\textbf{Cursos alternativos}                                                                    & \begin{tabular}[c]{@{}l@{}}Si set de datos no se acepta, se procede a\\ eliminar el archivo, CU40\end{tabular} \\ \hline
		\caption{Casos de uso relacionados a la revisi�n del set de datos}
		\label{CU09}\\
		\end{longtable}
		
		
	\subsection{Casos de uso asociados al m�dulo de clustering}
	
	Estos casos de uso cumplen con las funciones relacionadas a la aplicaci�n de algoritmos de clustering en set de datos.
	
	% Please add the following required packages to your document preamble:
	% \usepackage{longtable}
	% Note: It may be necessary to compile the document several times to get a multi-page table to line up properly
	\begin{longtable}{ll}
		\hline
		\multicolumn{2}{|c|}{\textit{\textbf{Caso de Uso: CU42}}}                                                                                                                                                                                                                           \\ \hline
		\endfirsthead
		%
		\endhead
		%
		\multicolumn{1}{|l|}{\textbf{Nombre}}                                                                                                & \multicolumn{1}{l|}{Implementar algoritmo K-Means}                                                                                           \\ \hline
		\multicolumn{1}{|l|}{\textbf{Actores}}                                                                                               & \multicolumn{1}{l|}{SCL, U}                                                                                                                  \\ \hline
		\multicolumn{1}{|l|}{\textbf{Funciones Asociadas}}                                                                                   & \multicolumn{1}{l|}{}                                                                                                                        \\ \hline
		\multicolumn{1}{|l|}{\textbf{Objetivo}}                                                                                              & \multicolumn{1}{l|}{Implementar algoritmo K-Means}                                                                                           \\ \hline
		\multicolumn{1}{|l|}{\textbf{Pre Condiciones}}                                                                                       & \multicolumn{1}{l|}{Solicitud de  k-means generada}                                                                                          \\ \hline
		\multicolumn{1}{|l|}{\textbf{Post Condiciones}}                                                                                      & \multicolumn{1}{l|}{Solicitud de k-means respondida}                                                                                         \\ \hline
		\multicolumn{1}{|c|}{\textit{\textbf{Acciones del Actor}}}                                                                           & \multicolumn{1}{c|}{\textit{\textbf{Acciones del Sistema}}}                                                                                  \\ \hline
		\multicolumn{1}{|l|}{Usuario accede a m�dulo de clustering}                                                                          & \multicolumn{1}{l|}{}                                                                                                                        \\ \hline
		\multicolumn{1}{|l|}{\begin{tabular}[c]{@{}l@{}}Usuario solicita aplicar clustering\\ con K-Means\end{tabular}}                      & \multicolumn{1}{l|}{}                                                                                                                        \\ \hline
		\multicolumn{1}{|l|}{}                                                                                                               & \multicolumn{1}{l|}{\begin{tabular}[c]{@{}l@{}}Sistema recibe solicitud y despliega\\ formulario con los par�metros necesarios\end{tabular}} \\ \hline
		\multicolumn{1}{|l|}{Usuario completa el formulario}                                                                                 & \multicolumn{1}{l|}{}                                                                                                                        \\ \hline
		\multicolumn{1}{|l|}{}                                                                                                               & \multicolumn{1}{l|}{Sistema recibe el formulario}                                                                                            \\ \hline
		\multicolumn{1}{|l|}{}                                                                                                               & \multicolumn{1}{l|}{\begin{tabular}[c]{@{}l@{}}Sistema procesa la data y env�a\\ para aplicaci�n de algoritmo\end{tabular}}                  \\ \hline
		\multicolumn{1}{|l|}{\begin{tabular}[c]{@{}l@{}}SCL recibe solicitud y aplica algoritmo\\ con par�metros seleccionados\end{tabular}} & \multicolumn{1}{l|}{}                                                                                                                        \\ \hline
		\multicolumn{1}{|l|}{SCL genera respuesta de resultados}                                                                             & \multicolumn{1}{l|}{}                                                                                                                        \\ \hline
		\multicolumn{1}{|l|}{SCL genera evaluaci�n de resultados}                                                                            & \multicolumn{1}{l|}{}                                                                                                                        \\ \hline
		\multicolumn{1}{|l|}{}                                                                                                               & \multicolumn{1}{l|}{\begin{tabular}[c]{@{}l@{}}Sistema recibe respuestas y\\ resultados generados\end{tabular}}                              \\ \hline
		\multicolumn{1}{|l|}{}                                                                                                               & \multicolumn{1}{l|}{Sistema procesa la data}                                                                                                 \\ \hline
		\multicolumn{1}{|l|}{}                                                                                                               & \multicolumn{1}{l|}{Sistema despliega el resultado en pantalla}                                                                              \\ \hline
		\multicolumn{1}{|l|}{}                                                                                                               & \multicolumn{1}{l|}{\begin{tabular}[c]{@{}l@{}}Sistema notifica el correcto\\ funcionamiento del algoritmo\end{tabular}}                     \\ \hline
		\multicolumn{1}{|l|}{\textbf{Referencias Cruzadas}}                                                                                  & \multicolumn{1}{l|}{--}                                                                                                                      \\ \hline
		\multicolumn{1}{|l|}{\textbf{Cursos alternativos}}                                                                                   & \multicolumn{1}{l|}{\begin{tabular}[c]{@{}l@{}}No se puede aplicar algoritmo,\\ se notifica por las v�as establecidas\end{tabular}}          \\ \hline
		&                                                                                                                                              \\ \hline
		\multicolumn{2}{|c|}{\textit{\textbf{Caso de Uso: CU43}}}                                                                                                                                                                                                                           \\ \hline
		\multicolumn{1}{|l|}{\textbf{Nombre}}                                                                                                & \multicolumn{1}{l|}{Implementar algoritmo Mean Shift}                                                                                        \\ \hline
		\multicolumn{1}{|l|}{\textbf{Actores}}                                                                                               & \multicolumn{1}{l|}{SCL, U}                                                                                                                  \\ \hline
		\multicolumn{1}{|l|}{\textbf{Funciones Asociadas}}                                                                                   & \multicolumn{1}{l|}{}                                                                                                                        \\ \hline
		\multicolumn{1}{|l|}{\textbf{Objetivo}}                                                                                              & \multicolumn{1}{l|}{Implementar algoritmo Mean Shift}                                                                                        \\ \hline
		\multicolumn{1}{|l|}{\textbf{Pre Condiciones}}                                                                                       & \multicolumn{1}{l|}{Solicitud de  Mean Shift generada}                                                                                       \\ \hline
		\multicolumn{1}{|l|}{\textbf{Post Condiciones}}                                                                                      & \multicolumn{1}{l|}{Solicitud de Mean Shift respondida}                                                                                      \\ \hline
		\multicolumn{1}{|c|}{\textit{\textbf{Acciones del Actor}}}                                                                           & \multicolumn{1}{c|}{\textit{\textbf{Acciones del Sistema}}}                                                                                  \\ \hline
		\multicolumn{1}{|l|}{Usuario accede a m�dulo de clustering}                                                                          & \multicolumn{1}{l|}{}                                                                                                                        \\ \hline
		\multicolumn{1}{|l|}{\begin{tabular}[c]{@{}l@{}}Usuario solicita aplicar clustering con\\ Mean Shift\end{tabular}}                   & \multicolumn{1}{l|}{}                                                                                                                        \\ \hline
		\multicolumn{1}{|l|}{}                                                                                                               & \multicolumn{1}{l|}{\begin{tabular}[c]{@{}l@{}}Sistema recibe solicitud y despliega\\ formulario con los par�metros necesarios\end{tabular}} \\ \hline
		\multicolumn{1}{|l|}{Usuario completa el formulario}                                                                                 & \multicolumn{1}{l|}{}                                                                                                                        \\ \hline
		\multicolumn{1}{|l|}{}                                                                                                               & \multicolumn{1}{l|}{Sistema recibe el formulario}                                                                                            \\ \hline
		\multicolumn{1}{|l|}{}                                                                                                               & \multicolumn{1}{l|}{\begin{tabular}[c]{@{}l@{}}Sistema procesa la data y env�a para\\ aplicaci�n de algoritmo\end{tabular}}                  \\ \hline
		\multicolumn{1}{|l|}{\begin{tabular}[c]{@{}l@{}}SCL recibe solicitud y aplica algoritmo\\ con par�metros seleccionados\end{tabular}} & \multicolumn{1}{l|}{}                                                                                                                        \\ \hline
		\multicolumn{1}{|l|}{SCL genera respuesta de resultados}                                                                             & \multicolumn{1}{l|}{}                                                                                                                        \\ \hline
		\multicolumn{1}{|l|}{SCL genera evaluaci�n de resultados}                                                                            & \multicolumn{1}{l|}{}                                                                                                                        \\ \hline
		\multicolumn{1}{|l|}{}                                                                                                               & \multicolumn{1}{l|}{\begin{tabular}[c]{@{}l@{}}Sistema recibe respuestas y resultados\\ generados\end{tabular}}                              \\ \hline
		\multicolumn{1}{|l|}{}                                                                                                               & \multicolumn{1}{l|}{Sistema procesa la data}                                                                                                 \\ \hline
		\multicolumn{1}{|l|}{}                                                                                                               & \multicolumn{1}{l|}{Sistema despliega el resultado en pantalla}                                                                              \\ \hline
		\multicolumn{1}{|l|}{}                                                                                                               & \multicolumn{1}{l|}{\begin{tabular}[c]{@{}l@{}}Sistema notifica el correcto funcionamiento\\ del algoritmo\end{tabular}}                     \\ \hline
		\multicolumn{1}{|l|}{\textbf{Referencias Cruzadas}}                                                                                  & \multicolumn{1}{l|}{--}                                                                                                                      \\ \hline
		\multicolumn{1}{|l|}{\textbf{Cursos alternativos}}                                                                                   & \multicolumn{1}{l|}{\begin{tabular}[c]{@{}l@{}}No se puede aplicar algoritmo,\\ se notifica por las v�as establecidas\end{tabular}}          \\ \hline
		&                                                                                                                                              \\ \hline
		\multicolumn{2}{|c|}{\textit{\textbf{Caso de Uso: CU44}}}                                                                                                                                                                                                                           \\ \hline
		\multicolumn{1}{|l|}{\textbf{Nombre}}                                                                                                & \multicolumn{1}{l|}{\begin{tabular}[c]{@{}l@{}}Implementar algoritmo Affinity\\ Propagation\end{tabular}}                                    \\ \hline
		\multicolumn{1}{|l|}{\textbf{Actores}}                                                                                               & \multicolumn{1}{l|}{SCL, U}                                                                                                                  \\ \hline
		\multicolumn{1}{|l|}{\textbf{Funciones Asociadas}}                                                                                   & \multicolumn{1}{l|}{}                                                                                                                        \\ \hline
		\multicolumn{1}{|l|}{\textbf{Objetivo}}                                                                                              & \multicolumn{1}{l|}{\begin{tabular}[c]{@{}l@{}}Implementar algoritmo Affinity\\ Propagation\end{tabular}}                                    \\ \hline
		\multicolumn{1}{|l|}{\textbf{Pre Condiciones}}                                                                                       & \multicolumn{1}{l|}{\begin{tabular}[c]{@{}l@{}}Solicitud de  Affinity\\ Propagation generada\end{tabular}}                                   \\ \hline
		\multicolumn{1}{|l|}{\textbf{Post Condiciones}}                                                                                      & \multicolumn{1}{l|}{\begin{tabular}[c]{@{}l@{}}Solicitud de Affinity\\ Propagation respondida\end{tabular}}                                  \\ \hline
		\multicolumn{1}{|c|}{\textit{\textbf{Acciones del Actor}}}                                                                           & \multicolumn{1}{c|}{\textit{\textbf{Acciones del Sistema}}}                                                                                  \\ \hline
		\multicolumn{1}{|l|}{Usuario accede a m�dulo de clustering}                                                                          & \multicolumn{1}{l|}{}                                                                                                                        \\ \hline
		\multicolumn{1}{|l|}{\begin{tabular}[c]{@{}l@{}}Usuario solicita aplicar clustering con\\ Affinity Propagation\end{tabular}}         & \multicolumn{1}{l|}{}                                                                                                                        \\ \hline
		\multicolumn{1}{|l|}{}                                                                                                               & \multicolumn{1}{l|}{\begin{tabular}[c]{@{}l@{}}Sistema recibe solicitud y despliega\\ formulario con los par�metros necesarios\end{tabular}} \\ \hline
		\multicolumn{1}{|l|}{Usuario completa el formulario}                                                                                 & \multicolumn{1}{l|}{}                                                                                                                        \\ \hline
		\multicolumn{1}{|l|}{}                                                                                                               & \multicolumn{1}{l|}{Sistema recibe el formulario}                                                                                            \\ \hline
		\multicolumn{1}{|l|}{}                                                                                                               & \multicolumn{1}{l|}{\begin{tabular}[c]{@{}l@{}}Sistema procesa la data y env�a\\ para aplicaci�n de algoritmo\end{tabular}}                  \\ \hline
		\multicolumn{1}{|l|}{\begin{tabular}[c]{@{}l@{}}SCL recibe solicitud y aplica algoritmo\\ con par�metros seleccionados\end{tabular}} & \multicolumn{1}{l|}{}                                                                                                                        \\ \hline
		\multicolumn{1}{|l|}{SCL genera respuesta de resultados}                                                                             & \multicolumn{1}{l|}{}                                                                                                                        \\ \hline
		\multicolumn{1}{|l|}{SCL genera evaluaci�n de resultados}                                                                            & \multicolumn{1}{l|}{}                                                                                                                        \\ \hline
		\multicolumn{1}{|l|}{}                                                                                                               & \multicolumn{1}{l|}{\begin{tabular}[c]{@{}l@{}}Sistema recibe respuestas y\\ resultados generados\end{tabular}}                              \\ \hline
		\multicolumn{1}{|l|}{}                                                                                                               & \multicolumn{1}{l|}{Sistema procesa la data}                                                                                                 \\ \hline
		\multicolumn{1}{|l|}{}                                                                                                               & \multicolumn{1}{l|}{\begin{tabular}[c]{@{}l@{}}Sistema despliega el resultado en\\ pantalla\end{tabular}}                                    \\ \hline
		\multicolumn{1}{|l|}{}                                                                                                               & \multicolumn{1}{l|}{\begin{tabular}[c]{@{}l@{}}Sistema notifica el correcto\\ funcionamiento del algoritmo\end{tabular}}                     \\ \hline
		\multicolumn{1}{|l|}{\textbf{Referencias Cruzadas}}                                                                                  & \multicolumn{1}{l|}{--}                                                                                                                      \\ \hline
		\multicolumn{1}{|l|}{\textbf{Cursos alternativos}}                                                                                   & \multicolumn{1}{l|}{\begin{tabular}[c]{@{}l@{}}No se puede aplicar algoritmo,\\ se notifica por las v�as establecidas\end{tabular}}          \\ \hline
		&                                                                                                                                              \\ \hline
		\multicolumn{2}{|c|}{\textit{\textbf{Caso de Uso: CU45}}}                                                                                                                                                                                                                           \\ \hline
		\multicolumn{1}{|l|}{\textbf{Nombre}}                                                                                                & \multicolumn{1}{l|}{Implementar algoritmo DBScan}                                                                                            \\ \hline
		\multicolumn{1}{|l|}{\textbf{Actores}}                                                                                               & \multicolumn{1}{l|}{SCL, U}                                                                                                                  \\ \hline
		\multicolumn{1}{|l|}{\textbf{Funciones Asociadas}}                                                                                   & \multicolumn{1}{l|}{}                                                                                                                        \\ \hline
		\multicolumn{1}{|l|}{\textbf{Objetivo}}                                                                                              & \multicolumn{1}{l|}{Implementar algoritmo DBScan}                                                                                            \\ \hline
		\multicolumn{1}{|l|}{\textbf{Pre Condiciones}}                                                                                       & \multicolumn{1}{l|}{Solicitud de DBScan generada}                                                                                            \\ \hline
		\multicolumn{1}{|l|}{\textbf{Post Condiciones}}                                                                                      & \multicolumn{1}{l|}{Solicitud de DBScan respondida}                                                                                          \\ \hline
		\multicolumn{1}{|c|}{\textit{\textbf{Acciones del Actor}}}                                                                           & \multicolumn{1}{c|}{\textit{\textbf{Acciones del Sistema}}}                                                                                  \\ \hline
		\multicolumn{1}{|l|}{Usuario accede a m�dulo de clustering}                                                                          & \multicolumn{1}{l|}{}                                                                                                                        \\ \hline
		\multicolumn{1}{|l|}{\begin{tabular}[c]{@{}l@{}}Usuario solicita aplicar clustering con\\ DBScan\end{tabular}}                       & \multicolumn{1}{l|}{}                                                                                                                        \\ \hline
		\multicolumn{1}{|l|}{}                                                                                                               & \multicolumn{1}{l|}{\begin{tabular}[c]{@{}l@{}}Sistema recibe solicitud y despliega\\ formulario con los par�metros necesarios\end{tabular}} \\ \hline
		\multicolumn{1}{|l|}{Usuario completa el formulario}                                                                                 & \multicolumn{1}{l|}{}                                                                                                                        \\ \hline
		\multicolumn{1}{|l|}{}                                                                                                               & \multicolumn{1}{l|}{Sistema recibe el formulario}                                                                                            \\ \hline
		\multicolumn{1}{|l|}{}                                                                                                               & \multicolumn{1}{l|}{\begin{tabular}[c]{@{}l@{}}Sistema procesa la data y env�a\\ para aplicaci�n de algoritmo\end{tabular}}                  \\ \hline
		\multicolumn{1}{|l|}{\begin{tabular}[c]{@{}l@{}}SCL recibe solicitud y aplica algoritmo\\ con par�metros seleccionados\end{tabular}} & \multicolumn{1}{l|}{}                                                                                                                        \\ \hline
		\multicolumn{1}{|l|}{SCL genera respuesta de resultados}                                                                             & \multicolumn{1}{l|}{}                                                                                                                        \\ \hline
		\multicolumn{1}{|l|}{SCL genera evaluaci�n de resultados}                                                                            & \multicolumn{1}{l|}{}                                                                                                                        \\ \hline
		\multicolumn{1}{|l|}{}                                                                                                               & \multicolumn{1}{l|}{\begin{tabular}[c]{@{}l@{}}Sistema recibe respuestas y\\ resultados generados\end{tabular}}                              \\ \hline
		\multicolumn{1}{|l|}{}                                                                                                               & \multicolumn{1}{l|}{Sistema procesa la data}                                                                                                 \\ \hline
		\multicolumn{1}{|l|}{}                                                                                                               & \multicolumn{1}{l|}{\begin{tabular}[c]{@{}l@{}}Sistema despliega el resultado\\ en pantalla\end{tabular}}                                    \\ \hline
		\multicolumn{1}{|l|}{}                                                                                                               & \multicolumn{1}{l|}{\begin{tabular}[c]{@{}l@{}}Sistema notifica el correcto\\ funcionamiento del algoritmo\end{tabular}}                     \\ \hline
		\multicolumn{1}{|l|}{\textbf{Referencias Cruzadas}}                                                                                  & \multicolumn{1}{l|}{--}                                                                                                                      \\ \hline
		\multicolumn{1}{|l|}{\textbf{Cursos alternativos}}                                                                                   & \multicolumn{1}{l|}{\begin{tabular}[c]{@{}l@{}}No se puede aplicar algoritmo,\\ se notifica por las v�as establecidas\end{tabular}}          \\ \hline
		&                                                                                                                                              \\ \hline
		\multicolumn{2}{|c|}{\textit{\textbf{Caso de Uso: CU46}}}                                                                                                                                                                                                                           \\ \hline
		\multicolumn{1}{|l|}{\textbf{Nombre}}                                                                                                & \multicolumn{1}{l|}{Implementar algoritmo Aglomerativos}                                                                                     \\ \hline
		\multicolumn{1}{|l|}{\textbf{Actores}}                                                                                               & \multicolumn{1}{l|}{SCL, U}                                                                                                                  \\ \hline
		\multicolumn{1}{|l|}{\textbf{Funciones Asociadas}}                                                                                   & \multicolumn{1}{l|}{}                                                                                                                        \\ \hline
		\multicolumn{1}{|l|}{\textbf{Objetivo}}                                                                                              & \multicolumn{1}{l|}{Implementar algoritmo Aglomerativos}                                                                                     \\ \hline
		\multicolumn{1}{|l|}{\textbf{Pre Condiciones}}                                                                                       & \multicolumn{1}{l|}{Solicitud de Aglomerativos generada}                                                                                     \\ \hline
		\multicolumn{1}{|l|}{\textbf{Post Condiciones}}                                                                                      & \multicolumn{1}{l|}{Solicitud de Aglomerativos respondida}                                                                                   \\ \hline
		\multicolumn{1}{|c|}{\textit{\textbf{Acciones del Actor}}}                                                                           & \multicolumn{1}{c|}{\textit{\textbf{Acciones del Sistema}}}                                                                                  \\ \hline
		\multicolumn{1}{|l|}{Usuario accede a m�dulo de clustering}                                                                          & \multicolumn{1}{l|}{}                                                                                                                        \\ \hline
		\multicolumn{1}{|l|}{\begin{tabular}[c]{@{}l@{}}Usuario solicita aplicar clustering con\\ Aglomerativos\end{tabular}}                & \multicolumn{1}{l|}{}                                                                                                                        \\ \hline
		\multicolumn{1}{|l|}{}                                                                                                               & \multicolumn{1}{l|}{\begin{tabular}[c]{@{}l@{}}Sistema recibe solicitud y despliega\\ formulario con los par�metros necesarios\end{tabular}} \\ \hline
		\multicolumn{1}{|l|}{Usuario completa el formulario}                                                                                 & \multicolumn{1}{l|}{}                                                                                                                        \\ \hline
		\multicolumn{1}{|l|}{}                                                                                                               & \multicolumn{1}{l|}{Sistema recibe el formulario}                                                                                            \\ \hline
		\multicolumn{1}{|l|}{}                                                                                                               & \multicolumn{1}{l|}{\begin{tabular}[c]{@{}l@{}}Sistema procesa la data y env�a para\\ aplicaci�n de algoritmo\end{tabular}}                  \\ \hline
		\multicolumn{1}{|l|}{\begin{tabular}[c]{@{}l@{}}SCL recibe solicitud y aplica algoritmo\\ con par�metros seleccionados\end{tabular}} & \multicolumn{1}{l|}{}                                                                                                                        \\ \hline
		\multicolumn{1}{|l|}{SCL genera respuesta de resultados}                                                                             & \multicolumn{1}{l|}{}                                                                                                                        \\ \hline
		\multicolumn{1}{|l|}{SCL genera evaluaci�n de resultados}                                                                            & \multicolumn{1}{l|}{}                                                                                                                        \\ \hline
		\multicolumn{1}{|l|}{}                                                                                                               & \multicolumn{1}{l|}{\begin{tabular}[c]{@{}l@{}}Sistema recibe respuestas y\\ resultados generados\end{tabular}}                              \\ \hline
		\multicolumn{1}{|l|}{}                                                                                                               & \multicolumn{1}{l|}{Sistema procesa la data}                                                                                                 \\ \hline
		\multicolumn{1}{|l|}{}                                                                                                               & \multicolumn{1}{l|}{Sistema despliega el resultado en pantalla}                                                                              \\ \hline
		\multicolumn{1}{|l|}{}                                                                                                               & \multicolumn{1}{l|}{\begin{tabular}[c]{@{}l@{}}Sistema notifica el correcto funcionamiento\\ del algoritmo\end{tabular}}                     \\ \hline
		\multicolumn{1}{|l|}{\textbf{Referencias Cruzadas}}                                                                                  & \multicolumn{1}{l|}{--}                                                                                                                      \\ \hline
		\multicolumn{1}{|l|}{\textbf{Cursos alternativos}}                                                                                   & \multicolumn{1}{l|}{\begin{tabular}[c]{@{}l@{}}No se puede aplicar algoritmo,\\ se notifica por las v�as establecidas\end{tabular}}          \\ \hline
		&                                                                                                                                              \\ \hline
		\multicolumn{2}{|c|}{\textit{\textbf{Caso de Uso: CU48}}}                                                                                                                                                                                                                           \\ \hline
		\multicolumn{1}{|l|}{\textbf{Nombre}}                                                                                                & \multicolumn{1}{l|}{Implementar algoritmo jerarquizado}                                                                                      \\ \hline
		\multicolumn{1}{|l|}{\textbf{Actores}}                                                                                               & \multicolumn{1}{l|}{SCL, U}                                                                                                                  \\ \hline
		\multicolumn{1}{|l|}{\textbf{Funciones Asociadas}}                                                                                   & \multicolumn{1}{l|}{}                                                                                                                        \\ \hline
		\multicolumn{1}{|l|}{\textbf{Objetivo}}                                                                                              & \multicolumn{1}{l|}{Implementar algoritmo jerarquizado}                                                                                      \\ \hline
		\multicolumn{1}{|l|}{\textbf{Pre Condiciones}}                                                                                       & \multicolumn{1}{l|}{Solicitud de jerarquizado generada}                                                                                      \\ \hline
		\multicolumn{1}{|l|}{\textbf{Post Condiciones}}                                                                                      & \multicolumn{1}{l|}{Solicitud de jerarquizado respondida}                                                                                    \\ \hline
		\multicolumn{1}{|c|}{\textit{\textbf{Acciones del Actor}}}                                                                           & \multicolumn{1}{c|}{\textit{\textbf{Acciones del Sistema}}}                                                                                  \\ \hline
		\multicolumn{1}{|l|}{Usuario accede a m�dulo de clustering}                                                                          & \multicolumn{1}{l|}{}                                                                                                                        \\ \hline
		\multicolumn{1}{|l|}{\begin{tabular}[c]{@{}l@{}}Usuario solicita aplicar clustering con\\ jerarquizado\end{tabular}}                 & \multicolumn{1}{l|}{}                                                                                                                        \\ \hline
		\multicolumn{1}{|l|}{}                                                                                                               & \multicolumn{1}{l|}{\begin{tabular}[c]{@{}l@{}}Sistema recibe solicitud y despliega\\ formulario con los par�metros necesarios\end{tabular}} \\ \hline
		\multicolumn{1}{|l|}{Usuario completa el formulario}                                                                                 & \multicolumn{1}{l|}{}                                                                                                                        \\ \hline
		\multicolumn{1}{|l|}{}                                                                                                               & \multicolumn{1}{l|}{Sistema recibe el formulario}                                                                                            \\ \hline
		\multicolumn{1}{|l|}{}                                                                                                               & \multicolumn{1}{l|}{\begin{tabular}[c]{@{}l@{}}Sistema procesa la data y env�a para\\ aplicaci�n de algoritmo\end{tabular}}                  \\ \hline
		\multicolumn{1}{|l|}{\begin{tabular}[c]{@{}l@{}}SCL recibe solicitud y aplica algoritmo\\ con par�metros seleccionados\end{tabular}} & \multicolumn{1}{l|}{}                                                                                                                        \\ \hline
		\multicolumn{1}{|l|}{SCL genera respuesta de resultados}                                                                             & \multicolumn{1}{l|}{}                                                                                                                        \\ \hline
		\multicolumn{1}{|l|}{SCL genera evaluaci�n de resultados}                                                                            & \multicolumn{1}{l|}{}                                                                                                                        \\ \hline
		\multicolumn{1}{|l|}{}                                                                                                               & \multicolumn{1}{l|}{Sistema recibe respuestas y resultados generados}                                                                        \\ \hline
		\multicolumn{1}{|l|}{}                                                                                                               & \multicolumn{1}{l|}{Sistema procesa la data}                                                                                                 \\ \hline
		\multicolumn{1}{|l|}{}                                                                                                               & \multicolumn{1}{l|}{Sistema despliega el resultado en pantalla}                                                                              \\ \hline
		\multicolumn{1}{|l|}{}                                                                                                               & \multicolumn{1}{l|}{\begin{tabular}[c]{@{}l@{}}Sistema notifica el correcto\\ funcionamiento del algoritmo\end{tabular}}                     \\ \hline
		\multicolumn{1}{|l|}{\textbf{Referencias Cruzadas}}                                                                                  & \multicolumn{1}{l|}{--}                                                                                                                      \\ \hline
		\multicolumn{1}{|l|}{\textbf{Cursos alternativos}}                                                                                   & \multicolumn{1}{l|}{\begin{tabular}[c]{@{}l@{}}No se puede aplicar algoritmo,\\ se notifica por las v�as establecidas\end{tabular}}          \\ \hline
		&                                                                                                                                              \\ \hline
		\multicolumn{2}{|c|}{\textit{\textbf{Caso de Uso: CU49}}}                                                                                                                                                                                                                           \\ \hline
		\multicolumn{1}{|l|}{\textbf{Nombre}}                                                                                                & \multicolumn{1}{l|}{Implementar algoritmo SOM}                                                                                               \\ \hline
		\multicolumn{1}{|l|}{\textbf{Actores}}                                                                                               & \multicolumn{1}{l|}{SCL, U}                                                                                                                  \\ \hline
		\multicolumn{1}{|l|}{\textbf{Funciones Asociadas}}                                                                                   & \multicolumn{1}{l|}{}                                                                                                                        \\ \hline
		\multicolumn{1}{|l|}{\textbf{Objetivo}}                                                                                              & \multicolumn{1}{l|}{Implementar algoritmo SOM}                                                                                               \\ \hline
		\multicolumn{1}{|l|}{\textbf{Pre Condiciones}}                                                                                       & \multicolumn{1}{l|}{Solicitud de SOM generada}                                                                                               \\ \hline
		\multicolumn{1}{|l|}{\textbf{Post Condiciones}}                                                                                      & \multicolumn{1}{l|}{Solicitud de SOM respondida}                                                                                             \\ \hline
		\multicolumn{1}{|c|}{\textit{\textbf{Acciones del Actor}}}                                                                           & \multicolumn{1}{c|}{\textit{\textbf{Acciones del Sistema}}}                                                                                  \\ \hline
		\multicolumn{1}{|l|}{Usuario accede a m�dulo de clustering}                                                                          & \multicolumn{1}{l|}{}                                                                                                                        \\ \hline
		\multicolumn{1}{|l|}{Usuario solicita aplicar clustering con SOM}                                                                    & \multicolumn{1}{l|}{}                                                                                                                        \\ \hline
		\multicolumn{1}{|l|}{}                                                                                                               & \multicolumn{1}{l|}{\begin{tabular}[c]{@{}l@{}}Sistema recibe solicitud y despliega\\ formulario con los par�metros necesarios\end{tabular}} \\ \hline
		\multicolumn{1}{|l|}{Usuario completa el formulario}                                                                                 & \multicolumn{1}{l|}{}                                                                                                                        \\ \hline
		\multicolumn{1}{|l|}{}                                                                                                               & \multicolumn{1}{l|}{Sistema recibe el formulario}                                                                                            \\ \hline
		\multicolumn{1}{|l|}{}                                                                                                               & \multicolumn{1}{l|}{\begin{tabular}[c]{@{}l@{}}Sistema procesa la data y env�a\\ para aplicaci�n de algoritmo\end{tabular}}                  \\ \hline
		\multicolumn{1}{|l|}{\begin{tabular}[c]{@{}l@{}}SCL recibe solicitud y aplica algoritmo\\ con par�metros seleccionados\end{tabular}} & \multicolumn{1}{l|}{}                                                                                                                        \\ \hline
		\multicolumn{1}{|l|}{SCL genera respuesta de resultados}                                                                             & \multicolumn{1}{l|}{}                                                                                                                        \\ \hline
		\multicolumn{1}{|l|}{SCL genera evaluaci�n de resultados}                                                                            & \multicolumn{1}{l|}{}                                                                                                                        \\ \hline
		\multicolumn{1}{|l|}{}                                                                                                               & \multicolumn{1}{l|}{\begin{tabular}[c]{@{}l@{}}Sistema recibe respuestas y\\ resultados generados\end{tabular}}                              \\ \hline
		\multicolumn{1}{|l|}{}                                                                                                               & \multicolumn{1}{l|}{Sistema procesa la data}                                                                                                 \\ \hline
		\multicolumn{1}{|l|}{}                                                                                                               & \multicolumn{1}{l|}{Sistema despliega el resultado en pantalla}                                                                              \\ \hline
		\multicolumn{1}{|l|}{}                                                                                                               & \multicolumn{1}{l|}{\begin{tabular}[c]{@{}l@{}}Sistema notifica el correcto\\ funcionamiento del algoritmo\end{tabular}}                     \\ \hline
		\multicolumn{1}{|l|}{\textbf{Referencias Cruzadas}}                                                                                  & \multicolumn{1}{l|}{--}                                                                                                                      \\ \hline
		\multicolumn{1}{|l|}{\textbf{Cursos alternativos}}                                                                                   & \multicolumn{1}{l|}{\begin{tabular}[c]{@{}l@{}}No se puede aplicar algoritmo,\\ se notifica por las v�as establecidas\end{tabular}}          \\ \hline
		&                                                                                                                                              \\ \hline
		\multicolumn{2}{|c|}{\textit{\textbf{Caso de Uso: CU50}}}                                                                                                                                                                                                                           \\ \hline
		\multicolumn{1}{|l|}{\textbf{Nombre}}                                                                                                & \multicolumn{1}{l|}{Evaluar las particiones generadas}                                                                                       \\ \hline
		\multicolumn{1}{|l|}{\textbf{Actores}}                                                                                               & \multicolumn{1}{l|}{SCL, U}                                                                                                                  \\ \hline
		\multicolumn{1}{|l|}{\textbf{Funciones Asociadas}}                                                                                   & \multicolumn{1}{l|}{}                                                                                                                        \\ \hline
		\multicolumn{1}{|l|}{\textbf{Objetivo}}                                                                                              & \multicolumn{1}{l|}{Evaluar las particiones generadas}                                                                                       \\ \hline
		\multicolumn{1}{|l|}{\textbf{Pre Condiciones}}                                                                                       & \multicolumn{1}{l|}{\begin{tabular}[c]{@{}l@{}}Particiones generadas mediante\\ Clustering sin evaluaci�n\end{tabular}}                      \\ \hline
		\multicolumn{1}{|l|}{\textbf{Post Condiciones}}                                                                                      & \multicolumn{1}{l|}{\begin{tabular}[c]{@{}l@{}}Particiones generadas mediante\\ Clustering evaluadas de manera\\ correcta\end{tabular}}      \\ \hline
		\multicolumn{1}{|c|}{\textit{\textbf{Acciones del Actor}}}                                                                           & \multicolumn{1}{c|}{\textit{\textbf{Acciones del Sistema}}}                                                                                  \\ \hline
		\multicolumn{1}{|l|}{Usuario accede a m�dulo de clustering}                                                                          & \multicolumn{1}{l|}{}                                                                                                                        \\ \hline
		\multicolumn{1}{|l|}{\begin{tabular}[c]{@{}l@{}}Usuario solicita aplicar clustering con\\ algunos de los algoritmos\end{tabular}}    & \multicolumn{1}{l|}{}                                                                                                                        \\ \hline
		\multicolumn{1}{|l|}{}                                                                                                               & \multicolumn{1}{l|}{Sistema recibe la solicitud}                                                                                             \\ \hline
		\multicolumn{1}{|l|}{}                                                                                                               & \multicolumn{1}{l|}{\begin{tabular}[c]{@{}l@{}}Sistema procesa la solicitud\\ y la env�a al SCL\end{tabular}}                                \\ \hline
		\multicolumn{1}{|l|}{\begin{tabular}[c]{@{}l@{}}SCL recibe solicitud y aplica algoritmo\\ con par�metros seleccionados\end{tabular}} & \multicolumn{1}{l|}{}                                                                                                                        \\ \hline
		\multicolumn{1}{|l|}{\begin{tabular}[c]{@{}l@{}}SCL eval�a las particiones mediante\\ calinkski-harabazz\end{tabular}}               & \multicolumn{1}{l|}{}                                                                                                                        \\ \hline
		\multicolumn{1}{|l|}{\begin{tabular}[c]{@{}l@{}}SCL eval�a las particiones mediante\\ coeficiente de siluetas\end{tabular}}          & \multicolumn{1}{l|}{}                                                                                                                        \\ \hline
		\multicolumn{1}{|l|}{\begin{tabular}[c]{@{}l@{}}SCL eval�a las particiones mediante\\ an�lisis estad�sticos\end{tabular}}            & \multicolumn{1}{l|}{}                                                                                                                        \\ \hline
		\multicolumn{1}{|l|}{SCL reporta resultados}                                                                                         & \multicolumn{1}{l|}{}                                                                                                                        \\ \hline
		\multicolumn{1}{|l|}{}                                                                                                               & \multicolumn{1}{l|}{Sistema recibe los resultados}                                                                                           \\ \hline
		\multicolumn{1}{|l|}{}                                                                                                               & \multicolumn{1}{l|}{\begin{tabular}[c]{@{}l@{}}Sistema despliega resultados en\\ interfaz mediante uso de data table\end{tabular}}           \\ \hline
		\multicolumn{1}{|l|}{}                                                                                                               & \multicolumn{1}{l|}{\begin{tabular}[c]{@{}l@{}}Sistema despliega datatable y\\ expone los resultados\end{tabular}}                           \\ \hline
		\multicolumn{1}{|l|}{}                                                                                                               & \multicolumn{1}{l|}{\begin{tabular}[c]{@{}l@{}}Sistema notifica el correcto\\ t�rmino de proceso\end{tabular}}                               \\ \hline
		\multicolumn{1}{|l|}{\textbf{Referencias Cruzadas}}                                                                                  & \multicolumn{1}{l|}{CU25}                                                                                                                    \\ \hline
		\multicolumn{1}{|l|}{\textbf{Cursos alternativos}}                                                                                   & \multicolumn{1}{l|}{\begin{tabular}[c]{@{}l@{}}No se puede aplicar algoritmo,\\ se notifica por las v�as establecidas\end{tabular}}          \\ \hline
		&                                                                                                                                              \\ \hline
		\multicolumn{2}{|c|}{\textit{\textbf{Caso de Uso: CU51}}}                                                                                                                                                                                                                           \\ \hline
		\multicolumn{1}{|l|}{\textbf{Nombre}}                                                                                                & \multicolumn{1}{l|}{\begin{tabular}[c]{@{}l@{}}Generar fase exploratoria de algoritmos\\ de clustering\end{tabular}}                         \\ \hline
		\multicolumn{1}{|l|}{\textbf{Actores}}                                                                                               & \multicolumn{1}{l|}{SC, U}                                                                                                                   \\ \hline
		\multicolumn{1}{|l|}{\textbf{Funciones Asociadas}}                                                                                   & \multicolumn{1}{l|}{}                                                                                                                        \\ \hline
		\multicolumn{1}{|l|}{\textbf{Objetivo}}                                                                                              & \multicolumn{1}{l|}{\begin{tabular}[c]{@{}l@{}}Generar fase exploratoria de algoritmos\\ de clustering\end{tabular}}                         \\ \hline
		\multicolumn{1}{|l|}{\textbf{Pre Condiciones}}                                                                                       & \multicolumn{1}{l|}{\begin{tabular}[c]{@{}l@{}}Solicitud de fase exploratoria en algoritmos\\ de clustering generada\end{tabular}}           \\ \hline
		\multicolumn{1}{|l|}{\textbf{Post Condiciones}}                                                                                      & \multicolumn{1}{l|}{\begin{tabular}[c]{@{}l@{}}Solicitud de fase exploratoria en algoritmos\\ de clustering procesada\end{tabular}}          \\ \hline
		\multicolumn{1}{|c|}{\textit{\textbf{Acciones del Actor}}}                                                                           & \multicolumn{1}{c|}{\textit{\textbf{Acciones del Sistema}}}                                                                                  \\ \hline
		\multicolumn{1}{|l|}{\begin{tabular}[c]{@{}l@{}}Usuario accede a m�dulo de\\ clustering\end{tabular}}                                & \multicolumn{1}{l|}{}                                                                                                                        \\ \hline
		\multicolumn{1}{|l|}{\begin{tabular}[c]{@{}l@{}}Usuario accede a fase\\ exploratoria\end{tabular}}                                   & \multicolumn{1}{l|}{}                                                                                                                        \\ \hline
		\multicolumn{1}{|l|}{}                                                                                                               & \multicolumn{1}{l|}{Sistema recibe solicitud}                                                                                                \\ \hline
		\multicolumn{1}{|l|}{}                                                                                                               & \multicolumn{1}{l|}{Sistema prepara el Job}                                                                                                  \\ \hline
		\multicolumn{1}{|l|}{}                                                                                                               & \multicolumn{1}{l|}{Sistema lanza job}                                                                                                       \\ \hline
		\multicolumn{1}{|l|}{}                                                                                                               & \multicolumn{1}{l|}{\begin{tabular}[c]{@{}l@{}}Sistema notifica al usuario que\\ el job ha sido lanzado\end{tabular}}                        \\ \hline
		\multicolumn{1}{|l|}{SC eval�a el estado del job}                                                                                    & \multicolumn{1}{l|}{}                                                                                                                        \\ \hline
		\multicolumn{1}{|l|}{SC finaliza job}                                                                                                & \multicolumn{1}{l|}{}                                                                                                                        \\ \hline
		\multicolumn{1}{|l|}{SC notifica finalizaci�n de job}                                                                                & \multicolumn{1}{l|}{}                                                                                                                        \\ \hline
		\multicolumn{1}{|l|}{}                                                                                                               & \multicolumn{1}{l|}{Sistema recibe notificaci�n}                                                                                             \\ \hline
		\multicolumn{1}{|l|}{}                                                                                                               & \multicolumn{1}{l|}{\begin{tabular}[c]{@{}l@{}}Sistema prepara resumen de resultados\\ y particiones generadas\end{tabular}}                 \\ \hline
		\multicolumn{1}{|l|}{}                                                                                                               & \multicolumn{1}{l|}{\begin{tabular}[c]{@{}l@{}}Sistema despliega datatable y expone\\ los resultados\end{tabular}}                           \\ \hline
		\multicolumn{1}{|l|}{}                                                                                                               & \multicolumn{1}{l|}{Sistema notifica al usuario el resultado final}                                                                          \\ \hline
		\multicolumn{1}{|l|}{\textbf{Referencias Cruzadas}}                                                                                  & \multicolumn{1}{l|}{CU52, CU53, CU25}                                                                                                        \\ \hline
		\multicolumn{1}{|l|}{\textbf{Cursos alternativos}}                                                                                   & \multicolumn{1}{l|}{\begin{tabular}[c]{@{}l@{}}No se puede aplicar algoritmo, se notifica\\ por las v�as establecidas\end{tabular}}          \\ \hline
		\caption{Casos de uso relacionados a los m�dulos de clustering}
		\label{CU10}\\
		\end{longtable}
	
	\subsection{Casos de uso relacionados al lanzamiento de Jobs}
	
	Los jobs representan acciones en el sistema que ser�n ejecutadas en segundo plano y ser� notificado el t�rmino de cada procesos.
	
	% Please add the following required packages to your document preamble:
	% \usepackage{longtable}
	% Note: It may be necessary to compile the document several times to get a multi-page table to line up properly
	\begin{longtable}{ll}
		\hline
		\multicolumn{2}{|c|}{\textit{\textbf{Caso de Uso: CU52}}}                                                                                                                                                                                                                  \\ \hline
		\endfirsthead
		%
		\endhead
		%
		\multicolumn{1}{|l|}{\textbf{Nombre}}                                                                                               & \multicolumn{1}{l|}{Crear un nuevo Job}                                                                                              \\ \hline
		\multicolumn{1}{|l|}{\textbf{Actores}}                                                                                              & \multicolumn{1}{l|}{SC}                                                                                                              \\ \hline
		\multicolumn{1}{|l|}{\textbf{Funciones Asociadas}}                                                                                  & \multicolumn{1}{l|}{}                                                                                                                \\ \hline
		\multicolumn{1}{|l|}{\textbf{Objetivo}}                                                                                             & \multicolumn{1}{l|}{Crear un nuevo Job}                                                                                              \\ \hline
		\multicolumn{1}{|l|}{\textbf{Pre Condiciones}}                                                                                      & \multicolumn{1}{l|}{Job no creado}                                                                                                   \\ \hline
		\multicolumn{1}{|l|}{\textbf{Post Condiciones}}                                                                                     & \multicolumn{1}{l|}{\begin{tabular}[c]{@{}l@{}}Job creado y lanzado al\\ sistema de colas\end{tabular}}                              \\ \hline
		\multicolumn{1}{|c|}{\textit{\textbf{Acciones del Actor}}}                                                                          & \multicolumn{1}{c|}{\textit{\textbf{Acciones del Sistema}}}                                                                          \\ \hline
		\multicolumn{1}{|l|}{SC recibe petici�n de crear nuevo job}                                                                         & \multicolumn{1}{l|}{}                                                                                                                \\ \hline
		\multicolumn{1}{|l|}{SC recibe la data}                                                                                             & \multicolumn{1}{l|}{\begin{tabular}[c]{@{}l@{}}Sistema procesa la informaci�n y\\ crea el job\end{tabular}}                          \\ \hline
		\multicolumn{1}{|l|}{}                                                                                                              & \multicolumn{1}{l|}{Sistema encola el job}                                                                                           \\ \hline
		\multicolumn{1}{|l|}{}                                                                                                              & \multicolumn{1}{l|}{\begin{tabular}[c]{@{}l@{}}Sistema almacena en sistema\\ persistente los atributos del job\end{tabular}}         \\ \hline
		\multicolumn{1}{|l|}{}                                                                                                              & \multicolumn{1}{l|}{\begin{tabular}[c]{@{}l@{}}Sistema notifica mediante correo\\ electr�nico el estado del job\end{tabular}}        \\ \hline
		\multicolumn{1}{|l|}{}                                                                                                              & \multicolumn{1}{l|}{\begin{tabular}[c]{@{}l@{}}Sistema habilita item para evaluar\\ avance del job\end{tabular}}                     \\ \hline
		\multicolumn{1}{|l|}{\textbf{Referencias Cruzadas}}                                                                                 & \multicolumn{1}{l|}{CU25, CU02, CU21}                                                                                                \\ \hline
		\multicolumn{1}{|l|}{\textbf{Cursos alternativos}}                                                                                  & \multicolumn{1}{l|}{--}                                                                                                              \\ \hline
		&                                                                                                                                      \\ \hline
		\multicolumn{2}{|c|}{\textit{\textbf{Caso de Uso: CU53}}}                                                                                                                                                                                                                  \\ \hline
		\multicolumn{1}{|l|}{\textbf{Nombre}}                                                                                               & \multicolumn{1}{l|}{Finalizar un Job}                                                                                                \\ \hline
		\multicolumn{1}{|l|}{\textbf{Actores}}                                                                                              & \multicolumn{1}{l|}{SC}                                                                                                              \\ \hline
		\multicolumn{1}{|l|}{\textbf{Funciones Asociadas}}                                                                                  & \multicolumn{1}{l|}{}                                                                                                                \\ \hline
		\multicolumn{1}{|l|}{\textbf{Objetivo}}                                                                                             & \multicolumn{1}{l|}{Finalizar un Job}                                                                                                \\ \hline
		\multicolumn{1}{|l|}{\textbf{Pre Condiciones}}                                                                                      & \multicolumn{1}{l|}{\begin{tabular}[c]{@{}l@{}}Job en estado de ejecuci�n\\ o en espera\end{tabular}}                                \\ \hline
		\multicolumn{1}{|l|}{\textbf{Post Condiciones}}                                                                                     & \multicolumn{1}{l|}{Job en estado finalizado}                                                                                        \\ \hline
		\multicolumn{1}{|c|}{\textit{\textbf{Acciones del Actor}}}                                                                          & \multicolumn{1}{c|}{\textit{\textbf{Acciones del Sistema}}}                                                                          \\ \hline
		\multicolumn{1}{|l|}{SC recibe petici�n de finalizar job}                                                                           & \multicolumn{1}{l|}{}                                                                                                                \\ \hline
		\multicolumn{1}{|l|}{SC recibe la data}                                                                                             & \multicolumn{1}{l|}{\begin{tabular}[c]{@{}l@{}}Sistema procesa la informaci�n\\ y finaliza el job\end{tabular}}                      \\ \hline
		\multicolumn{1}{|l|}{}                                                                                                              & \multicolumn{1}{l|}{Sistema des encola el job}                                                                                       \\ \hline
		\multicolumn{1}{|l|}{}                                                                                                              & \multicolumn{1}{l|}{\begin{tabular}[c]{@{}l@{}}Sistema almacena en sistema\\ persistente los atributos del job\end{tabular}}         \\ \hline
		\multicolumn{1}{|l|}{}                                                                                                              & \multicolumn{1}{l|}{\begin{tabular}[c]{@{}l@{}}Sistema notifica mediante correo\\ electr�nico el estado del job\end{tabular}}        \\ \hline
		\multicolumn{1}{|l|}{\textbf{Referencias Cruzadas}}                                                                                 & \multicolumn{1}{l|}{CU25, CU02}                                                                                                      \\ \hline
		\multicolumn{1}{|l|}{\textbf{Cursos alternativos}}                                                                                  & \multicolumn{1}{l|}{--}                                                                                                              \\ \hline
		&                                                                                                                                      \\ \hline
		\multicolumn{2}{|c|}{\textit{\textbf{Caso de Uso: CU75}}}                                                                                                                                                                                                                  \\ \hline
		\multicolumn{1}{|l|}{Nombre}                                                                                                        & \multicolumn{1}{l|}{Editar un Job}                                                                                                   \\ \hline
		\multicolumn{1}{|l|}{Actores}                                                                                                       & \multicolumn{1}{l|}{SC}                                                                                                              \\ \hline
		\multicolumn{1}{|l|}{Funciones Asociadas}                                                                                           & \multicolumn{1}{l|}{}                                                                                                                \\ \hline
		\multicolumn{1}{|l|}{Objetivo}                                                                                                      & \multicolumn{1}{l|}{Editar un Job}                                                                                                   \\ \hline
		\multicolumn{1}{|l|}{Pre Condiciones}                                                                                               & \multicolumn{1}{l|}{\begin{tabular}[c]{@{}l@{}}Job en estado de ejecuci�n o\\ en espera\end{tabular}}                                \\ \hline
		\multicolumn{1}{|l|}{Post Condiciones}                                                                                              & \multicolumn{1}{l|}{\begin{tabular}[c]{@{}l@{}}Job con cambio de estado\\ generado\end{tabular}}                                     \\ \hline
		\multicolumn{1}{|l|}{Acciones del Actor}                                                                                            & \multicolumn{1}{l|}{Acciones del Sistema}                                                                                            \\ \hline
		\multicolumn{1}{|l|}{\begin{tabular}[c]{@{}l@{}}SC recibe petici�n de editar estado\\ de job\end{tabular}}                          & \multicolumn{1}{l|}{}                                                                                                                \\ \hline
		\multicolumn{1}{|l|}{SC recibe la data}                                                                                             & \multicolumn{1}{l|}{\begin{tabular}[c]{@{}l@{}}Sistema procesa la informaci�n\\ y edita el job\end{tabular}}                         \\ \hline
		\multicolumn{1}{|l|}{}                                                                                                              & \multicolumn{1}{l|}{\begin{tabular}[c]{@{}l@{}}Sistema cambia estado de proceso\\ en gestor de colas\end{tabular}}                   \\ \hline
		\multicolumn{1}{|l|}{}                                                                                                              & \multicolumn{1}{l|}{\begin{tabular}[c]{@{}l@{}}Sistema almacena en sistema\\ persistente los atributos del job\end{tabular}}         \\ \hline
		\multicolumn{1}{|l|}{}                                                                                                              & \multicolumn{1}{l|}{\begin{tabular}[c]{@{}l@{}}Sistema notifica mediante correo\\ electr�nico el estado del job\end{tabular}}        \\ \hline
		\multicolumn{1}{|l|}{\textbf{Referencias Cruzadas}}                                                                                 & \multicolumn{1}{l|}{CU25, CU02}                                                                                                      \\ \hline
		\multicolumn{1}{|l|}{\textbf{Cursos alternativos}}                                                                                  & \multicolumn{1}{l|}{--}                                                                                                              \\ \hline
		&                                                                                                                                      \\ \hline
		\multicolumn{2}{|c|}{\textit{\textbf{Caso de Uso: CU76}}}                                                                                                                                                                                                                  \\ \hline
		\multicolumn{1}{|l|}{\textbf{Nombre}}                                                                                               & \multicolumn{1}{l|}{Mostrar estado de cola}                                                                                          \\ \hline
		\multicolumn{1}{|l|}{\textbf{Actores}}                                                                                              & \multicolumn{1}{l|}{SC}                                                                                                              \\ \hline
		\multicolumn{1}{|l|}{\textbf{Funciones Asociadas}}                                                                                  & \multicolumn{1}{l|}{}                                                                                                                \\ \hline
		\multicolumn{1}{|l|}{\textbf{Objetivo}}                                                                                             & \multicolumn{1}{l|}{Mostrar estado de cola}                                                                                          \\ \hline
		\multicolumn{1}{|l|}{\textbf{Pre Condiciones}}                                                                                      & \multicolumn{1}{l|}{\begin{tabular}[c]{@{}l@{}}Mostrar el estado de los trabajos\\ en el sistema de colas\end{tabular}}              \\ \hline
		\multicolumn{1}{|l|}{\textbf{Post Condiciones}}                                                                                     & \multicolumn{1}{l|}{\begin{tabular}[c]{@{}l@{}}Solicitud de visualizaci�n de estado\\ de colas en espera\end{tabular}}               \\ \hline
		\multicolumn{1}{|c|}{\textit{\textbf{Acciones del Actor}}}                                                                          & \multicolumn{1}{c|}{\textit{\textbf{Acciones del Sistema}}}                                                                          \\ \hline
		\multicolumn{1}{|l|}{\begin{tabular}[c]{@{}l@{}}SC recibe petici�n de mostrar estado\\ de colas\end{tabular}}                       & \multicolumn{1}{l|}{}                                                                                                                \\ \hline
		\multicolumn{1}{|l|}{SC recupera el estado de los jobs}                                                                             & \multicolumn{1}{l|}{}                                                                                                                \\ \hline
		\multicolumn{1}{|l|}{\begin{tabular}[c]{@{}l@{}}SC procesa la informaci�n en\\ formato JSON\end{tabular}}                           & \multicolumn{1}{l|}{}                                                                                                                \\ \hline
		\multicolumn{1}{|l|}{SC env�a respuesta}                                                                                            & \multicolumn{1}{l|}{}                                                                                                                \\ \hline
		\multicolumn{1}{|l|}{}                                                                                                              & \multicolumn{1}{l|}{Sistema recibe respuesta}                                                                                        \\ \hline
		\multicolumn{1}{|l|}{}                                                                                                              & \multicolumn{1}{l|}{\begin{tabular}[c]{@{}l@{}}Sistema crea gr�fico resumen con\\ los procesos y sus estados\end{tabular}}           \\ \hline
		\multicolumn{1}{|l|}{}                                                                                                              & \multicolumn{1}{l|}{\begin{tabular}[c]{@{}l@{}}Sistema crea datatable con el resumen\\ de los procesos\end{tabular}}                 \\ \hline
		\multicolumn{1}{|l|}{}                                                                                                              & \multicolumn{1}{l|}{\begin{tabular}[c]{@{}l@{}}Sistema despliega los resultados\\ en la vista\end{tabular}}                          \\ \hline
		\multicolumn{1}{|l|}{\textbf{Referencias Cruzadas}}                                                                                 & \multicolumn{1}{l|}{CU25, CU02, CU26}                                                                                                \\ \hline
		\multicolumn{1}{|l|}{\textbf{Cursos alternativos}}                                                                                  & \multicolumn{1}{l|}{--}                                                                                                              \\ \hline
		&                                                                                                                                      \\ \hline
		\multicolumn{2}{|c|}{\textit{\textbf{Caso de Uso: CU77}}}                                                                                                                                                                                                                  \\ \hline
		\multicolumn{1}{|l|}{\textbf{Nombre}}                                                                                               & \multicolumn{1}{l|}{Notificar estados de procesos}                                                                                   \\ \hline
		\multicolumn{1}{|l|}{\textbf{Actores}}                                                                                              & \multicolumn{1}{l|}{SC}                                                                                                              \\ \hline
		\multicolumn{1}{|l|}{\textbf{Funciones Asociadas}}                                                                                  & \multicolumn{1}{l|}{}                                                                                                                \\ \hline
		\multicolumn{1}{|l|}{\textbf{Objetivo}}                                                                                             & \multicolumn{1}{l|}{Notificar estados de procesos}                                                                                   \\ \hline
		\multicolumn{1}{|l|}{\textbf{Pre Condiciones}}                                                                                      & \multicolumn{1}{l|}{\begin{tabular}[c]{@{}l@{}}Notificar v�a correo electr�nico el\\ estado de un proceso\end{tabular}}              \\ \hline
		\multicolumn{1}{|l|}{\textbf{Post Condiciones}}                                                                                     & \multicolumn{1}{l|}{\begin{tabular}[c]{@{}l@{}}Solicitud de notificaci�n de estado \\ \\ de Job\end{tabular}}                        \\ \hline
		\multicolumn{1}{|c|}{\textit{\textbf{Acciones del Actor}}}                                                                          & \multicolumn{1}{c|}{\textit{\textbf{Acciones del Sistema}}}                                                                          \\ \hline
		\multicolumn{1}{|l|}{\begin{tabular}[c]{@{}l@{}}Sistema recibe solicitud de notificaci�n de\\ estado de proceso\end{tabular}}       & \multicolumn{1}{l|}{}                                                                                                                \\ \hline
		\multicolumn{1}{|l|}{Sistema chequea el estado y lo registra}                                                                       & \multicolumn{1}{l|}{}                                                                                                                \\ \hline
		\multicolumn{1}{|l|}{Sistema env�a respuesta en formato JSON}                                                                       & \multicolumn{1}{l|}{}                                                                                                                \\ \hline
		\multicolumn{1}{|l|}{}                                                                                                              & \multicolumn{1}{l|}{Sistema recibe respuesta}                                                                                        \\ \hline
		\multicolumn{1}{|l|}{}                                                                                                              & \multicolumn{1}{l|}{\begin{tabular}[c]{@{}l@{}}Sistema env�a petici�n de notificaci�n\\ al gestor de correos\end{tabular}}           \\ \hline
		\multicolumn{1}{|l|}{}                                                                                                              & \multicolumn{1}{l|}{Sistema genera correo y notifica al usuario}                                                                     \\ \hline
		\multicolumn{1}{|l|}{}                                                                                                              & \multicolumn{1}{l|}{\begin{tabular}[c]{@{}l@{}}Sistema registra el estado del proceso en\\ almacenamiento persistente\end{tabular}}  \\ \hline
		\multicolumn{1}{|l|}{}                                                                                                              & \multicolumn{1}{l|}{\begin{tabular}[c]{@{}l@{}}Sistema registra el estado del proceso en\\ sistema Log\end{tabular}}                 \\ \hline
		\multicolumn{1}{|l|}{\textbf{Referencias Cruzadas}}                                                                                 & \multicolumn{1}{l|}{CU02, CU21}                                                                                                      \\ \hline
		\multicolumn{1}{|l|}{\textbf{Cursos alternativos}}                                                                                  & \multicolumn{1}{l|}{--}                                                                                                              \\ \hline
		&                                                                                                                                      \\ \hline
		\multicolumn{2}{|c|}{\textit{\textbf{Caso de Uso: CU78}}}                                                                                                                                                                                                                  \\ \hline
		\multicolumn{1}{|l|}{\textbf{Nombre}}                                                                                               & \multicolumn{1}{l|}{Notificar finalizaci�n de Job}                                                                                   \\ \hline
		\multicolumn{1}{|l|}{\textbf{Actores}}                                                                                              & \multicolumn{1}{l|}{SC}                                                                                                              \\ \hline
		\multicolumn{1}{|l|}{\textbf{Funciones Asociadas}}                                                                                  & \multicolumn{1}{l|}{}                                                                                                                \\ \hline
		\multicolumn{1}{|l|}{\textbf{Objetivo}}                                                                                             & \multicolumn{1}{l|}{Notificar finalizaci�n de Job}                                                                                   \\ \hline
		\multicolumn{1}{|l|}{\textbf{Pre Condiciones}}                                                                                      & \multicolumn{1}{l|}{\begin{tabular}[c]{@{}l@{}}Notificar v�a correo electr�nico la\\ finalizaci�n de un proceso\end{tabular}}        \\ \hline
		\multicolumn{1}{|l|}{\textbf{Post Condiciones}}                                                                                     & \multicolumn{1}{l|}{\begin{tabular}[c]{@{}l@{}}Solicitud de notificaci�n de estado\\ de Job\end{tabular}}                            \\ \hline
		\multicolumn{1}{|c|}{\textit{\textbf{Acciones del Actor}}}                                                                          & \multicolumn{1}{c|}{\textit{\textbf{Acciones del Sistema}}}                                                                          \\ \hline
		\multicolumn{1}{|l|}{\begin{tabular}[c]{@{}l@{}}Sistema recibe solicitud de notificaci�n de\\ finalizaci�n de proceso\end{tabular}} & \multicolumn{1}{l|}{}                                                                                                                \\ \hline
		\multicolumn{1}{|l|}{Sistema chequea el estado y lo registra}                                                                       & \multicolumn{1}{l|}{}                                                                                                                \\ \hline
		\multicolumn{1}{|l|}{Sistema env�a respuesta en formato JSON}                                                                       & \multicolumn{1}{l|}{}                                                                                                                \\ \hline
		\multicolumn{1}{|l|}{SC finaliza job}                                                                                               & \multicolumn{1}{l|}{}                                                                                                                \\ \hline
		\multicolumn{1}{|l|}{}                                                                                                              & \multicolumn{1}{l|}{Sistema recibe respuesta}                                                                                        \\ \hline
		\multicolumn{1}{|l|}{}                                                                                                              & \multicolumn{1}{l|}{\begin{tabular}[c]{@{}l@{}}Sistema env�a petici�n de notificaci�n\\ al gestor de correos\end{tabular}}           \\ \hline
		\multicolumn{1}{|l|}{}                                                                                                              & \multicolumn{1}{l|}{Sistema genera correo y notifica al usuario}                                                                     \\ \hline
		\multicolumn{1}{|l|}{}                                                                                                              & \multicolumn{1}{l|}{\begin{tabular}[c]{@{}l@{}}Sistema registra el estado del proceso en\\ almacenamiento persistente\end{tabular}}  \\ \hline
		\multicolumn{1}{|l|}{}                                                                                                              & \multicolumn{1}{l|}{\begin{tabular}[c]{@{}l@{}}Sistema registra el estado del proceso en\\ sistema Log\end{tabular}}                 \\ \hline
		\multicolumn{1}{|l|}{\textbf{Referencias Cruzadas}}                                                                                 & \multicolumn{1}{l|}{CU02, CU21}                                                                                                      \\ \hline
		\multicolumn{1}{|l|}{\textbf{Cursos alternativos}}                                                                                  & \multicolumn{1}{l|}{--}                                                                                                              \\ \hline
		&                                                                                                                                      \\ \hline
		\multicolumn{2}{|c|}{\textit{\textbf{Caso de Uso: CU79}}}                                                                                                                                                                                                                  \\ \hline
		\multicolumn{1}{|l|}{\textbf{Nombre}}                                                                                               & \multicolumn{1}{l|}{Revisar procesos en cola}                                                                                        \\ \hline
		\multicolumn{1}{|l|}{\textbf{Actores}}                                                                                              & \multicolumn{1}{l|}{SC}                                                                                                              \\ \hline
		\multicolumn{1}{|l|}{\textbf{Funciones Asociadas}}                                                                                  & \multicolumn{1}{l|}{}                                                                                                                \\ \hline
		\multicolumn{1}{|l|}{\textbf{Objetivo}}                                                                                             & \multicolumn{1}{l|}{Revisar procesos en cola}                                                                                        \\ \hline
		\multicolumn{1}{|l|}{\textbf{Pre Condiciones}}                                                                                      & \multicolumn{1}{l|}{\begin{tabular}[c]{@{}l@{}}Revisar procesos en cola y determinar\\ si es necesario cambiar estados\end{tabular}} \\ \hline
		\multicolumn{1}{|l|}{\textbf{Post Condiciones}}                                                                                     & \multicolumn{1}{l|}{Revisi�n de procesos requerida}                                                                                  \\ \hline
		\multicolumn{1}{|c|}{\textit{\textbf{Acciones del Actor}}}                                                                          & \multicolumn{1}{c|}{\textit{\textbf{Acciones del Sistema}}}                                                                          \\ \hline
		\multicolumn{1}{|l|}{SC recibe solicitud de revisar sus procesos}                                                                   & \multicolumn{1}{l|}{}                                                                                                                \\ \hline
		\multicolumn{1}{|l|}{SC revisa los estados de todos los procesos}                                                                   & \multicolumn{1}{l|}{}                                                                                                                \\ \hline
		\multicolumn{1}{|l|}{SC eval�a el tiempo de espera de cada proceso}                                                                 & \multicolumn{1}{l|}{}                                                                                                                \\ \hline
		\multicolumn{1}{|l|}{SC eval�a el tiempo de ejecuci�n de cada proceso}                                                              & \multicolumn{1}{l|}{}                                                                                                                \\ \hline
		\multicolumn{1}{|l|}{SC eval�a la prioridad de cada proceso}                                                                        & \multicolumn{1}{l|}{}                                                                                                                \\ \hline
		\multicolumn{1}{|l|}{SC determina qu� procesos finalizar}                                                                           & \multicolumn{1}{l|}{}                                                                                                                \\ \hline
		\multicolumn{1}{|l|}{SC finaliza jobs}                                                                                              & \multicolumn{1}{l|}{}                                                                                                                \\ \hline
		\multicolumn{1}{|l|}{SC genera resumen de revisi�n}                                                                                 & \multicolumn{1}{l|}{}                                                                                                                \\ \hline
		\multicolumn{1}{|l|}{}                                                                                                              & \multicolumn{1}{l|}{Sistema recibe resumen}                                                                                          \\ \hline
		\multicolumn{1}{|l|}{}                                                                                                              & \multicolumn{1}{l|}{\begin{tabular}[c]{@{}l@{}}Sistema notifica los cambios de estado\\ seg�n corresponda\end{tabular}}              \\ \hline
		\multicolumn{1}{|l|}{}                                                                                                              & \multicolumn{1}{l|}{\begin{tabular}[c]{@{}l@{}}Sistema almacena en sistema persistente\\ los atributos del job\end{tabular}}         \\ \hline
		\multicolumn{1}{|l|}{}                                                                                                              & \multicolumn{1}{l|}{\begin{tabular}[c]{@{}l@{}}Sistema registra el estado del proceso en\\ sistema Log\end{tabular}}                 \\ \hline
		\multicolumn{1}{|l|}{\textbf{Referencias Cruzadas}}                                                                                 & \multicolumn{1}{l|}{CU02, CU21}                                                                                                      \\ \hline
		\multicolumn{1}{|l|}{\textbf{Cursos alternativos}}                                                                                  & \multicolumn{1}{l|}{--}                                                                                                              \\ \hline
		\caption{Casos de uso asociados al lanzamiento de Jobs}
		\label{CU11}\\
		\end{longtable}
		
	\subsection{Diagramas de casos de uso}

\section{Diagramas de secuencia o colaboraci�n}

\section{Conceptos}

\subsection{Modelo Conceptual}

\section{Entidades}

\subsection{Modelo de Entidades}
 %hipotesis
\chapter{Dise�o}

\section{Arquitectura de Software}

\section{Diagramas de Interacci�n}

\section{Diagrama de Clases}

\section{Diagramas de Estado}

 %objetivos
\chapter{Planificaci�n}

\section{Etapas del Proyecto}

 %metodologias y desarrollos de modelos
\begin{thebibliography}{X}

\bibitem{intro1} \textit{Data Mining Curriculum}. ACM SIGKDD. 2006-04-30. Retrieved 2014-01-27.

\bibitem{intro2} Clifton, Christopher (2010). \textit{Encyclopedia Britannica: Definition of Data Mining}. Retrieved 2010-12-09.

\bibitem{intro3} Hastie, Trevor; Tibshirani, Robert; Friedman, Jerome (2009). \textit{The Elements of Statistical Learning: Data Mining, Inference, and Prediction}. Retrieved 2012-08-07.

\bibitem{intro4} Fayyad, Usama; Piatetsky-Shapiro, Gregory; Smyth, Padhraic (1996). \textit{From Data Mining to Knowledge Discovery in Databases} (PDF). Retrieved 17 December 2008.

\end{thebibliography}
%%% ambiente glosario
%\begin{glosario}
%  \item[El primer t�rmino:] Este es el significado del primer t�rmino, realmente no se bien lo que significa pero podr�a haberlo averiguado si hubiese tenido un poco mas de tiempo.
%  \item[El segundo t�rmino:] Este si se lo que significa pero me da lata escribirlo...
%\end{glosario}


%% genera las referencias
%\nocite{*}
%\bibliography{bib/b1}

%% comienzo de la parte de anexos


%% contenido del primer anexo
%% fin
\end{document}

   
