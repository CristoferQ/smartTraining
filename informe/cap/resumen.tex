
Conceptos como miner�a de datos, machine learning, big data, an�lisis estad�sticos, modelamientos matem�ticos, etc, son mencionados d�a a d�a, ya sea en el �mbito privado como p�blico, involucrando �reas como: comercio, salud, investigaci�n, transporte, etc. Lo cual denota que son tem�ticas que han adquirido mayor relevancia y su ascenso seguir� con el pasar del tiempo.

La manipulaci�n de grandes vol�menes de datos, con el fin de poder extraer informaci�n de ellos, b�squeda de patrones, evaluaciones estad�sticas, etc. Implica por parte del interesado, tener conocimientos en dichas �reas adem�s de comprender herramientas inform�ticas que le permitan dicho procedimiento. Sin embargo, dichas herramientas o son costosas, debido a la licencia que implica, o, se requiere de conocimiento inform�tico para su manipulaci�n, debido a que requiere implementar m�dulos o servicios a medida que permitan ejecutar las tareas de inter�s, lo cual deja a un n�mero importante de entidades que desean involucrarse en dicho mundo, pero no cuentan con las capacidades ni tampoco con las competencias para ello.

Dado a lo anterior y en base a la creciente demanda de desarrollo de metodolog�as que permitan aplicar data mining a procesos de datos, con el fin de extraer informaci�n y conocimiento de la misma, se propone Smart Training, como sistema web, que facilite los procesos de evaluaciones estad�sticas, b�squeda de patrones de comportamiento, desarrollo de modelos de clasificaci�n y evaluaci�n de caracter�sticas o features en el set de datos a estudiar.

